\documentclass[11pt]{article}

    \usepackage[breakable]{tcolorbox}
    \usepackage{parskip} % Stop auto-indenting (to mimic markdown behaviour)
    

    % Basic figure setup, for now with no caption control since it's done
    % automatically by Pandoc (which extracts ![](path) syntax from Markdown).
    \usepackage{graphicx}
    % Maintain compatibility with old templates. Remove in nbconvert 6.0
    \let\Oldincludegraphics\includegraphics
    % Ensure that by default, figures have no caption (until we provide a
    % proper Figure object with a Caption API and a way to capture that
    % in the conversion process - todo).
    \usepackage{caption}
    \DeclareCaptionFormat{nocaption}{}
    \captionsetup{format=nocaption,aboveskip=0pt,belowskip=0pt}

    \usepackage{float}
    \floatplacement{figure}{H} % forces figures to be placed at the correct location
    \usepackage{xcolor} % Allow colors to be defined
    \usepackage{enumerate} % Needed for markdown enumerations to work
    \usepackage{geometry} % Used to adjust the document margins
    \usepackage{amsmath} % Equations
    \usepackage{amssymb} % Equations
    \usepackage{textcomp} % defines textquotesingle
    % Hack from http://tex.stackexchange.com/a/47451/13684:
    \AtBeginDocument{%
        \def\PYZsq{\textquotesingle}% Upright quotes in Pygmentized code
    }
    \usepackage{upquote} % Upright quotes for verbatim code
    \usepackage{eurosym} % defines \euro

    \usepackage{iftex}
    \ifPDFTeX
        \usepackage[T1]{fontenc}
        \IfFileExists{alphabeta.sty}{
              \usepackage{alphabeta}
          }{
              \usepackage[mathletters]{ucs}
              \usepackage[utf8x]{inputenc}
          }
    \else
        \usepackage{fontspec}
        \usepackage{unicode-math}
    \fi

    \usepackage{fancyvrb} % verbatim replacement that allows latex
    \usepackage[Export]{adjustbox} % Used to constrain images to a maximum size
    \adjustboxset{max size={0.9\linewidth}{0.9\paperheight}}

    % The hyperref package gives us a pdf with properly built
    % internal navigation ('pdf bookmarks' for the table of contents,
    % internal cross-reference links, web links for URLs, etc.)
    \usepackage{hyperref}
    % The default LaTeX title has an obnoxious amount of whitespace. By default,
    % titling removes some of it. It also provides customization options.
    \usepackage{titling}
    \usepackage{longtable} % longtable support required by pandoc >1.10
    \usepackage{booktabs}  % table support for pandoc > 1.12.2
    \usepackage{array}     % table support for pandoc >= 2.11.3
    \usepackage{calc}      % table minipage width calculation for pandoc >= 2.11.1
    \usepackage[inline]{enumitem} % IRkernel/repr support (it uses the enumerate* environment)
    \usepackage[normalem]{ulem} % ulem is needed to support strikethroughs (\sout)
                                % normalem makes italics be italics, not underlines
    \usepackage{mathrsfs}
    

    
    % Colors for the hyperref package
    \definecolor{urlcolor}{rgb}{0,.145,.698}
    \definecolor{linkcolor}{rgb}{.71,0.21,0.01}
    \definecolor{citecolor}{rgb}{.12,.54,.11}

    % ANSI colors
    \definecolor{ansi-black}{HTML}{3E424D}
    \definecolor{ansi-black-intense}{HTML}{282C36}
    \definecolor{ansi-red}{HTML}{E75C58}
    \definecolor{ansi-red-intense}{HTML}{B22B31}
    \definecolor{ansi-green}{HTML}{00A250}
    \definecolor{ansi-green-intense}{HTML}{007427}
    \definecolor{ansi-yellow}{HTML}{DDB62B}
    \definecolor{ansi-yellow-intense}{HTML}{B27D12}
    \definecolor{ansi-blue}{HTML}{208FFB}
    \definecolor{ansi-blue-intense}{HTML}{0065CA}
    \definecolor{ansi-magenta}{HTML}{D160C4}
    \definecolor{ansi-magenta-intense}{HTML}{A03196}
    \definecolor{ansi-cyan}{HTML}{60C6C8}
    \definecolor{ansi-cyan-intense}{HTML}{258F8F}
    \definecolor{ansi-white}{HTML}{C5C1B4}
    \definecolor{ansi-white-intense}{HTML}{A1A6B2}
    \definecolor{ansi-default-inverse-fg}{HTML}{FFFFFF}
    \definecolor{ansi-default-inverse-bg}{HTML}{000000}

    % common color for the border for error outputs.
    \definecolor{outerrorbackground}{HTML}{FFDFDF}

    % commands and environments needed by pandoc snippets
    % extracted from the output of `pandoc -s`
    \providecommand{\tightlist}{%
      \setlength{\itemsep}{0pt}\setlength{\parskip}{0pt}}
    \DefineVerbatimEnvironment{Highlighting}{Verbatim}{commandchars=\\\{\}}
    % Add ',fontsize=\small' for more characters per line
    \newenvironment{Shaded}{}{}
    \newcommand{\KeywordTok}[1]{\textcolor[rgb]{0.00,0.44,0.13}{\textbf{{#1}}}}
    \newcommand{\DataTypeTok}[1]{\textcolor[rgb]{0.56,0.13,0.00}{{#1}}}
    \newcommand{\DecValTok}[1]{\textcolor[rgb]{0.25,0.63,0.44}{{#1}}}
    \newcommand{\BaseNTok}[1]{\textcolor[rgb]{0.25,0.63,0.44}{{#1}}}
    \newcommand{\FloatTok}[1]{\textcolor[rgb]{0.25,0.63,0.44}{{#1}}}
    \newcommand{\CharTok}[1]{\textcolor[rgb]{0.25,0.44,0.63}{{#1}}}
    \newcommand{\StringTok}[1]{\textcolor[rgb]{0.25,0.44,0.63}{{#1}}}
    \newcommand{\CommentTok}[1]{\textcolor[rgb]{0.38,0.63,0.69}{\textit{{#1}}}}
    \newcommand{\OtherTok}[1]{\textcolor[rgb]{0.00,0.44,0.13}{{#1}}}
    \newcommand{\AlertTok}[1]{\textcolor[rgb]{1.00,0.00,0.00}{\textbf{{#1}}}}
    \newcommand{\FunctionTok}[1]{\textcolor[rgb]{0.02,0.16,0.49}{{#1}}}
    \newcommand{\RegionMarkerTok}[1]{{#1}}
    \newcommand{\ErrorTok}[1]{\textcolor[rgb]{1.00,0.00,0.00}{\textbf{{#1}}}}
    \newcommand{\NormalTok}[1]{{#1}}
    
    % Additional commands for more recent versions of Pandoc
    \newcommand{\ConstantTok}[1]{\textcolor[rgb]{0.53,0.00,0.00}{{#1}}}
    \newcommand{\SpecialCharTok}[1]{\textcolor[rgb]{0.25,0.44,0.63}{{#1}}}
    \newcommand{\VerbatimStringTok}[1]{\textcolor[rgb]{0.25,0.44,0.63}{{#1}}}
    \newcommand{\SpecialStringTok}[1]{\textcolor[rgb]{0.73,0.40,0.53}{{#1}}}
    \newcommand{\ImportTok}[1]{{#1}}
    \newcommand{\DocumentationTok}[1]{\textcolor[rgb]{0.73,0.13,0.13}{\textit{{#1}}}}
    \newcommand{\AnnotationTok}[1]{\textcolor[rgb]{0.38,0.63,0.69}{\textbf{\textit{{#1}}}}}
    \newcommand{\CommentVarTok}[1]{\textcolor[rgb]{0.38,0.63,0.69}{\textbf{\textit{{#1}}}}}
    \newcommand{\VariableTok}[1]{\textcolor[rgb]{0.10,0.09,0.49}{{#1}}}
    \newcommand{\ControlFlowTok}[1]{\textcolor[rgb]{0.00,0.44,0.13}{\textbf{{#1}}}}
    \newcommand{\OperatorTok}[1]{\textcolor[rgb]{0.40,0.40,0.40}{{#1}}}
    \newcommand{\BuiltInTok}[1]{{#1}}
    \newcommand{\ExtensionTok}[1]{{#1}}
    \newcommand{\PreprocessorTok}[1]{\textcolor[rgb]{0.74,0.48,0.00}{{#1}}}
    \newcommand{\AttributeTok}[1]{\textcolor[rgb]{0.49,0.56,0.16}{{#1}}}
    \newcommand{\InformationTok}[1]{\textcolor[rgb]{0.38,0.63,0.69}{\textbf{\textit{{#1}}}}}
    \newcommand{\WarningTok}[1]{\textcolor[rgb]{0.38,0.63,0.69}{\textbf{\textit{{#1}}}}}
    
    
    % Define a nice break command that doesn't care if a line doesn't already
    % exist.
    \def\br{\hspace*{\fill} \\* }
    % Math Jax compatibility definitions
    \def\gt{>}
    \def\lt{<}
    \let\Oldtex\TeX
    \let\Oldlatex\LaTeX
    \renewcommand{\TeX}{\textrm{\Oldtex}}
    \renewcommand{\LaTeX}{\textrm{\Oldlatex}}
    % Document parameters
    % Document title
    \title{analisisCaos2}
    
    
    
    
    
% Pygments definitions
\makeatletter
\def\PY@reset{\let\PY@it=\relax \let\PY@bf=\relax%
    \let\PY@ul=\relax \let\PY@tc=\relax%
    \let\PY@bc=\relax \let\PY@ff=\relax}
\def\PY@tok#1{\csname PY@tok@#1\endcsname}
\def\PY@toks#1+{\ifx\relax#1\empty\else%
    \PY@tok{#1}\expandafter\PY@toks\fi}
\def\PY@do#1{\PY@bc{\PY@tc{\PY@ul{%
    \PY@it{\PY@bf{\PY@ff{#1}}}}}}}
\def\PY#1#2{\PY@reset\PY@toks#1+\relax+\PY@do{#2}}

\@namedef{PY@tok@w}{\def\PY@tc##1{\textcolor[rgb]{0.73,0.73,0.73}{##1}}}
\@namedef{PY@tok@c}{\let\PY@it=\textit\def\PY@tc##1{\textcolor[rgb]{0.24,0.48,0.48}{##1}}}
\@namedef{PY@tok@cp}{\def\PY@tc##1{\textcolor[rgb]{0.61,0.40,0.00}{##1}}}
\@namedef{PY@tok@k}{\let\PY@bf=\textbf\def\PY@tc##1{\textcolor[rgb]{0.00,0.50,0.00}{##1}}}
\@namedef{PY@tok@kp}{\def\PY@tc##1{\textcolor[rgb]{0.00,0.50,0.00}{##1}}}
\@namedef{PY@tok@kt}{\def\PY@tc##1{\textcolor[rgb]{0.69,0.00,0.25}{##1}}}
\@namedef{PY@tok@o}{\def\PY@tc##1{\textcolor[rgb]{0.40,0.40,0.40}{##1}}}
\@namedef{PY@tok@ow}{\let\PY@bf=\textbf\def\PY@tc##1{\textcolor[rgb]{0.67,0.13,1.00}{##1}}}
\@namedef{PY@tok@nb}{\def\PY@tc##1{\textcolor[rgb]{0.00,0.50,0.00}{##1}}}
\@namedef{PY@tok@nf}{\def\PY@tc##1{\textcolor[rgb]{0.00,0.00,1.00}{##1}}}
\@namedef{PY@tok@nc}{\let\PY@bf=\textbf\def\PY@tc##1{\textcolor[rgb]{0.00,0.00,1.00}{##1}}}
\@namedef{PY@tok@nn}{\let\PY@bf=\textbf\def\PY@tc##1{\textcolor[rgb]{0.00,0.00,1.00}{##1}}}
\@namedef{PY@tok@ne}{\let\PY@bf=\textbf\def\PY@tc##1{\textcolor[rgb]{0.80,0.25,0.22}{##1}}}
\@namedef{PY@tok@nv}{\def\PY@tc##1{\textcolor[rgb]{0.10,0.09,0.49}{##1}}}
\@namedef{PY@tok@no}{\def\PY@tc##1{\textcolor[rgb]{0.53,0.00,0.00}{##1}}}
\@namedef{PY@tok@nl}{\def\PY@tc##1{\textcolor[rgb]{0.46,0.46,0.00}{##1}}}
\@namedef{PY@tok@ni}{\let\PY@bf=\textbf\def\PY@tc##1{\textcolor[rgb]{0.44,0.44,0.44}{##1}}}
\@namedef{PY@tok@na}{\def\PY@tc##1{\textcolor[rgb]{0.41,0.47,0.13}{##1}}}
\@namedef{PY@tok@nt}{\let\PY@bf=\textbf\def\PY@tc##1{\textcolor[rgb]{0.00,0.50,0.00}{##1}}}
\@namedef{PY@tok@nd}{\def\PY@tc##1{\textcolor[rgb]{0.67,0.13,1.00}{##1}}}
\@namedef{PY@tok@s}{\def\PY@tc##1{\textcolor[rgb]{0.73,0.13,0.13}{##1}}}
\@namedef{PY@tok@sd}{\let\PY@it=\textit\def\PY@tc##1{\textcolor[rgb]{0.73,0.13,0.13}{##1}}}
\@namedef{PY@tok@si}{\let\PY@bf=\textbf\def\PY@tc##1{\textcolor[rgb]{0.64,0.35,0.47}{##1}}}
\@namedef{PY@tok@se}{\let\PY@bf=\textbf\def\PY@tc##1{\textcolor[rgb]{0.67,0.36,0.12}{##1}}}
\@namedef{PY@tok@sr}{\def\PY@tc##1{\textcolor[rgb]{0.64,0.35,0.47}{##1}}}
\@namedef{PY@tok@ss}{\def\PY@tc##1{\textcolor[rgb]{0.10,0.09,0.49}{##1}}}
\@namedef{PY@tok@sx}{\def\PY@tc##1{\textcolor[rgb]{0.00,0.50,0.00}{##1}}}
\@namedef{PY@tok@m}{\def\PY@tc##1{\textcolor[rgb]{0.40,0.40,0.40}{##1}}}
\@namedef{PY@tok@gh}{\let\PY@bf=\textbf\def\PY@tc##1{\textcolor[rgb]{0.00,0.00,0.50}{##1}}}
\@namedef{PY@tok@gu}{\let\PY@bf=\textbf\def\PY@tc##1{\textcolor[rgb]{0.50,0.00,0.50}{##1}}}
\@namedef{PY@tok@gd}{\def\PY@tc##1{\textcolor[rgb]{0.63,0.00,0.00}{##1}}}
\@namedef{PY@tok@gi}{\def\PY@tc##1{\textcolor[rgb]{0.00,0.52,0.00}{##1}}}
\@namedef{PY@tok@gr}{\def\PY@tc##1{\textcolor[rgb]{0.89,0.00,0.00}{##1}}}
\@namedef{PY@tok@ge}{\let\PY@it=\textit}
\@namedef{PY@tok@gs}{\let\PY@bf=\textbf}
\@namedef{PY@tok@ges}{\let\PY@bf=\textbf\let\PY@it=\textit}
\@namedef{PY@tok@gp}{\let\PY@bf=\textbf\def\PY@tc##1{\textcolor[rgb]{0.00,0.00,0.50}{##1}}}
\@namedef{PY@tok@go}{\def\PY@tc##1{\textcolor[rgb]{0.44,0.44,0.44}{##1}}}
\@namedef{PY@tok@gt}{\def\PY@tc##1{\textcolor[rgb]{0.00,0.27,0.87}{##1}}}
\@namedef{PY@tok@err}{\def\PY@bc##1{{\setlength{\fboxsep}{\string -\fboxrule}\fcolorbox[rgb]{1.00,0.00,0.00}{1,1,1}{\strut ##1}}}}
\@namedef{PY@tok@kc}{\let\PY@bf=\textbf\def\PY@tc##1{\textcolor[rgb]{0.00,0.50,0.00}{##1}}}
\@namedef{PY@tok@kd}{\let\PY@bf=\textbf\def\PY@tc##1{\textcolor[rgb]{0.00,0.50,0.00}{##1}}}
\@namedef{PY@tok@kn}{\let\PY@bf=\textbf\def\PY@tc##1{\textcolor[rgb]{0.00,0.50,0.00}{##1}}}
\@namedef{PY@tok@kr}{\let\PY@bf=\textbf\def\PY@tc##1{\textcolor[rgb]{0.00,0.50,0.00}{##1}}}
\@namedef{PY@tok@bp}{\def\PY@tc##1{\textcolor[rgb]{0.00,0.50,0.00}{##1}}}
\@namedef{PY@tok@fm}{\def\PY@tc##1{\textcolor[rgb]{0.00,0.00,1.00}{##1}}}
\@namedef{PY@tok@vc}{\def\PY@tc##1{\textcolor[rgb]{0.10,0.09,0.49}{##1}}}
\@namedef{PY@tok@vg}{\def\PY@tc##1{\textcolor[rgb]{0.10,0.09,0.49}{##1}}}
\@namedef{PY@tok@vi}{\def\PY@tc##1{\textcolor[rgb]{0.10,0.09,0.49}{##1}}}
\@namedef{PY@tok@vm}{\def\PY@tc##1{\textcolor[rgb]{0.10,0.09,0.49}{##1}}}
\@namedef{PY@tok@sa}{\def\PY@tc##1{\textcolor[rgb]{0.73,0.13,0.13}{##1}}}
\@namedef{PY@tok@sb}{\def\PY@tc##1{\textcolor[rgb]{0.73,0.13,0.13}{##1}}}
\@namedef{PY@tok@sc}{\def\PY@tc##1{\textcolor[rgb]{0.73,0.13,0.13}{##1}}}
\@namedef{PY@tok@dl}{\def\PY@tc##1{\textcolor[rgb]{0.73,0.13,0.13}{##1}}}
\@namedef{PY@tok@s2}{\def\PY@tc##1{\textcolor[rgb]{0.73,0.13,0.13}{##1}}}
\@namedef{PY@tok@sh}{\def\PY@tc##1{\textcolor[rgb]{0.73,0.13,0.13}{##1}}}
\@namedef{PY@tok@s1}{\def\PY@tc##1{\textcolor[rgb]{0.73,0.13,0.13}{##1}}}
\@namedef{PY@tok@mb}{\def\PY@tc##1{\textcolor[rgb]{0.40,0.40,0.40}{##1}}}
\@namedef{PY@tok@mf}{\def\PY@tc##1{\textcolor[rgb]{0.40,0.40,0.40}{##1}}}
\@namedef{PY@tok@mh}{\def\PY@tc##1{\textcolor[rgb]{0.40,0.40,0.40}{##1}}}
\@namedef{PY@tok@mi}{\def\PY@tc##1{\textcolor[rgb]{0.40,0.40,0.40}{##1}}}
\@namedef{PY@tok@il}{\def\PY@tc##1{\textcolor[rgb]{0.40,0.40,0.40}{##1}}}
\@namedef{PY@tok@mo}{\def\PY@tc##1{\textcolor[rgb]{0.40,0.40,0.40}{##1}}}
\@namedef{PY@tok@ch}{\let\PY@it=\textit\def\PY@tc##1{\textcolor[rgb]{0.24,0.48,0.48}{##1}}}
\@namedef{PY@tok@cm}{\let\PY@it=\textit\def\PY@tc##1{\textcolor[rgb]{0.24,0.48,0.48}{##1}}}
\@namedef{PY@tok@cpf}{\let\PY@it=\textit\def\PY@tc##1{\textcolor[rgb]{0.24,0.48,0.48}{##1}}}
\@namedef{PY@tok@c1}{\let\PY@it=\textit\def\PY@tc##1{\textcolor[rgb]{0.24,0.48,0.48}{##1}}}
\@namedef{PY@tok@cs}{\let\PY@it=\textit\def\PY@tc##1{\textcolor[rgb]{0.24,0.48,0.48}{##1}}}

\def\PYZbs{\char`\\}
\def\PYZus{\char`\_}
\def\PYZob{\char`\{}
\def\PYZcb{\char`\}}
\def\PYZca{\char`\^}
\def\PYZam{\char`\&}
\def\PYZlt{\char`\<}
\def\PYZgt{\char`\>}
\def\PYZsh{\char`\#}
\def\PYZpc{\char`\%}
\def\PYZdl{\char`\$}
\def\PYZhy{\char`\-}
\def\PYZsq{\char`\'}
\def\PYZdq{\char`\"}
\def\PYZti{\char`\~}
% for compatibility with earlier versions
\def\PYZat{@}
\def\PYZlb{[}
\def\PYZrb{]}
\makeatother


    % For linebreaks inside Verbatim environment from package fancyvrb. 
    \makeatletter
        \newbox\Wrappedcontinuationbox 
        \newbox\Wrappedvisiblespacebox 
        \newcommand*\Wrappedvisiblespace {\textcolor{red}{\textvisiblespace}} 
        \newcommand*\Wrappedcontinuationsymbol {\textcolor{red}{\llap{\tiny$\m@th\hookrightarrow$}}} 
        \newcommand*\Wrappedcontinuationindent {3ex } 
        \newcommand*\Wrappedafterbreak {\kern\Wrappedcontinuationindent\copy\Wrappedcontinuationbox} 
        % Take advantage of the already applied Pygments mark-up to insert 
        % potential linebreaks for TeX processing. 
        %        {, <, #, %, $, ' and ": go to next line. 
        %        _, }, ^, &, >, - and ~: stay at end of broken line. 
        % Use of \textquotesingle for straight quote. 
        \newcommand*\Wrappedbreaksatspecials {% 
            \def\PYGZus{\discretionary{\char`\_}{\Wrappedafterbreak}{\char`\_}}% 
            \def\PYGZob{\discretionary{}{\Wrappedafterbreak\char`\{}{\char`\{}}% 
            \def\PYGZcb{\discretionary{\char`\}}{\Wrappedafterbreak}{\char`\}}}% 
            \def\PYGZca{\discretionary{\char`\^}{\Wrappedafterbreak}{\char`\^}}% 
            \def\PYGZam{\discretionary{\char`\&}{\Wrappedafterbreak}{\char`\&}}% 
            \def\PYGZlt{\discretionary{}{\Wrappedafterbreak\char`\<}{\char`\<}}% 
            \def\PYGZgt{\discretionary{\char`\>}{\Wrappedafterbreak}{\char`\>}}% 
            \def\PYGZsh{\discretionary{}{\Wrappedafterbreak\char`\#}{\char`\#}}% 
            \def\PYGZpc{\discretionary{}{\Wrappedafterbreak\char`\%}{\char`\%}}% 
            \def\PYGZdl{\discretionary{}{\Wrappedafterbreak\char`\$}{\char`\$}}% 
            \def\PYGZhy{\discretionary{\char`\-}{\Wrappedafterbreak}{\char`\-}}% 
            \def\PYGZsq{\discretionary{}{\Wrappedafterbreak\textquotesingle}{\textquotesingle}}% 
            \def\PYGZdq{\discretionary{}{\Wrappedafterbreak\char`\"}{\char`\"}}% 
            \def\PYGZti{\discretionary{\char`\~}{\Wrappedafterbreak}{\char`\~}}% 
        } 
        % Some characters . , ; ? ! / are not pygmentized. 
        % This macro makes them "active" and they will insert potential linebreaks 
        \newcommand*\Wrappedbreaksatpunct {% 
            \lccode`\~`\.\lowercase{\def~}{\discretionary{\hbox{\char`\.}}{\Wrappedafterbreak}{\hbox{\char`\.}}}% 
            \lccode`\~`\,\lowercase{\def~}{\discretionary{\hbox{\char`\,}}{\Wrappedafterbreak}{\hbox{\char`\,}}}% 
            \lccode`\~`\;\lowercase{\def~}{\discretionary{\hbox{\char`\;}}{\Wrappedafterbreak}{\hbox{\char`\;}}}% 
            \lccode`\~`\:\lowercase{\def~}{\discretionary{\hbox{\char`\:}}{\Wrappedafterbreak}{\hbox{\char`\:}}}% 
            \lccode`\~`\?\lowercase{\def~}{\discretionary{\hbox{\char`\?}}{\Wrappedafterbreak}{\hbox{\char`\?}}}% 
            \lccode`\~`\!\lowercase{\def~}{\discretionary{\hbox{\char`\!}}{\Wrappedafterbreak}{\hbox{\char`\!}}}% 
            \lccode`\~`\/\lowercase{\def~}{\discretionary{\hbox{\char`\/}}{\Wrappedafterbreak}{\hbox{\char`\/}}}% 
            \catcode`\.\active
            \catcode`\,\active 
            \catcode`\;\active
            \catcode`\:\active
            \catcode`\?\active
            \catcode`\!\active
            \catcode`\/\active 
            \lccode`\~`\~ 	
        }
    \makeatother

    \let\OriginalVerbatim=\Verbatim
    \makeatletter
    \renewcommand{\Verbatim}[1][1]{%
        %\parskip\z@skip
        \sbox\Wrappedcontinuationbox {\Wrappedcontinuationsymbol}%
        \sbox\Wrappedvisiblespacebox {\FV@SetupFont\Wrappedvisiblespace}%
        \def\FancyVerbFormatLine ##1{\hsize\linewidth
            \vtop{\raggedright\hyphenpenalty\z@\exhyphenpenalty\z@
                \doublehyphendemerits\z@\finalhyphendemerits\z@
                \strut ##1\strut}%
        }%
        % If the linebreak is at a space, the latter will be displayed as visible
        % space at end of first line, and a continuation symbol starts next line.
        % Stretch/shrink are however usually zero for typewriter font.
        \def\FV@Space {%
            \nobreak\hskip\z@ plus\fontdimen3\font minus\fontdimen4\font
            \discretionary{\copy\Wrappedvisiblespacebox}{\Wrappedafterbreak}
            {\kern\fontdimen2\font}%
        }%
        
        % Allow breaks at special characters using \PYG... macros.
        \Wrappedbreaksatspecials
        % Breaks at punctuation characters . , ; ? ! and / need catcode=\active 	
        \OriginalVerbatim[#1,codes*=\Wrappedbreaksatpunct]%
    }
    \makeatother

    % Exact colors from NB
    \definecolor{incolor}{HTML}{303F9F}
    \definecolor{outcolor}{HTML}{D84315}
    \definecolor{cellborder}{HTML}{CFCFCF}
    \definecolor{cellbackground}{HTML}{F7F7F7}
    
    % prompt
    \makeatletter
    \newcommand{\boxspacing}{\kern\kvtcb@left@rule\kern\kvtcb@boxsep}
    \makeatother
    \newcommand{\prompt}[4]{
        {\ttfamily\llap{{\color{#2}[#3]:\hspace{3pt}#4}}\vspace{-\baselineskip}}
    }
    

    
    % Prevent overflowing lines due to hard-to-break entities
    \sloppy 
    % Setup hyperref package
    \hypersetup{
      breaklinks=true,  % so long urls are correctly broken across lines
      colorlinks=true,
      urlcolor=urlcolor,
      linkcolor=linkcolor,
      citecolor=citecolor,
      }
    % Slightly bigger margins than the latex defaults
    
    \geometry{verbose,tmargin=1in,bmargin=1in,lmargin=1in,rmargin=1in}
    
    

\begin{document}
    
    \maketitle
    
    

    
    \hypertarget{anuxe1lisis-estaduxedstico-de-series-de-tiempo-cauxf3ticas}{%
\section{Análisis Estadístico de Series de Tiempo
Caóticas}\label{anuxe1lisis-estaduxedstico-de-series-de-tiempo-cauxf3ticas}}

\hypertarget{alumno-rodrigo-gerardo-trejo-arriaga}{%
\subsection{Alumno: Rodrigo Gerardo Trejo
Arriaga}\label{alumno-rodrigo-gerardo-trejo-arriaga}}

\hypertarget{tuxedtulo-de-la-pruxe1ctica-estaduxedsticas-descriptivas-de-atractores-cauxf3ticos}{%
\subsubsection{Título de la Práctica: Estadísticas descriptivas de
atractores
caóticos}\label{tuxedtulo-de-la-pruxe1ctica-estaduxedsticas-descriptivas-de-atractores-cauxf3ticos}}

Este segmento de la práctica explora las propiedades estadísticas de los
conjuntos de datos generados por modelos caóticos. Se calcularán
métricas como la media, mediana, entropía, kurtosis, varianza y
desviación estándar para demostrar que los datos no siguen
distribuciones de probabilidad convencionales como las normales,
uniformes o gamma.

\begin{center}\rule{0.5\linewidth}{0.5pt}\end{center}

Fecha de Entrega: \textbf{24 de Junio, 2024}

    \begin{tcolorbox}[breakable, size=fbox, boxrule=1pt, pad at break*=1mm,colback=cellbackground, colframe=cellborder]
\prompt{In}{incolor}{2}{\boxspacing}
\begin{Verbatim}[commandchars=\\\{\}]
\PY{k+kn}{import} \PY{n+nn}{numpy} \PY{k}{as} \PY{n+nn}{np}
\PY{k+kn}{from} \PY{n+nn}{scipy}\PY{n+nn}{.}\PY{n+nn}{stats} \PY{k+kn}{import} \PY{n}{gmean}\PY{p}{,} \PY{n}{skew}\PY{p}{,} \PY{n}{kurtosis}\PY{p}{,} \PY{n}{mode}
\PY{k+kn}{from} \PY{n+nn}{scipy} \PY{k+kn}{import} \PY{n}{stats}
\PY{k+kn}{import} \PY{n+nn}{matplotlib}\PY{n+nn}{.}\PY{n+nn}{pyplot} \PY{k}{as} \PY{n+nn}{plt}
\PY{k+kn}{import} \PY{n+nn}{pandas} \PY{k}{as} \PY{n+nn}{pd}
\PY{k+kn}{import} \PY{n+nn}{plotly}\PY{n+nn}{.}\PY{n+nn}{graph\PYZus{}objects} \PY{k}{as} \PY{n+nn}{go}
\PY{k+kn}{import} \PY{n+nn}{plotly}\PY{n+nn}{.}\PY{n+nn}{io} \PY{k}{as} \PY{n+nn}{pio}
\PY{k+kn}{from} \PY{n+nn}{sklearn}\PY{n+nn}{.}\PY{n+nn}{neighbors} \PY{k+kn}{import} \PY{n}{KDTree}
\PY{k+kn}{from} \PY{n+nn}{joblib} \PY{k+kn}{import} \PY{n}{Parallel}\PY{p}{,} \PY{n}{delayed}
\end{Verbatim}
\end{tcolorbox}

    \begin{tcolorbox}[breakable, size=fbox, boxrule=1pt, pad at break*=1mm,colback=cellbackground, colframe=cellborder]
\prompt{In}{incolor}{9}{\boxspacing}
\begin{Verbatim}[commandchars=\\\{\}]
\PY{k}{def} \PY{n+nf}{convertir\PYZus{}camelCase}\PY{p}{(}\PY{n}{text}\PY{p}{)}\PY{p}{:}
    \PY{n}{cleaned\PYZus{}text} \PY{o}{=} \PY{l+s+s1}{\PYZsq{}}\PY{l+s+s1}{\PYZsq{}}\PY{o}{.}\PY{n}{join}\PY{p}{(}\PY{n}{char} \PY{k}{for} \PY{n}{char} \PY{o+ow}{in} \PY{n}{text} \PY{k}{if} \PY{n}{char}\PY{o}{.}\PY{n}{isalnum}\PY{p}{(}\PY{p}{)} \PY{o+ow}{or} \PY{n}{char}\PY{o}{.}\PY{n}{isspace}\PY{p}{(}\PY{p}{)}\PY{p}{)}
    \PY{n}{words} \PY{o}{=} \PY{n}{cleaned\PYZus{}text}\PY{o}{.}\PY{n}{split}\PY{p}{(}\PY{p}{)}
    \PY{k}{return} \PY{n}{words}\PY{p}{[}\PY{l+m+mi}{0}\PY{p}{]}\PY{o}{.}\PY{n}{lower}\PY{p}{(}\PY{p}{)} \PY{o}{+} \PY{l+s+s1}{\PYZsq{}}\PY{l+s+s1}{\PYZsq{}}\PY{o}{.}\PY{n}{join}\PY{p}{(}\PY{n}{word}\PY{o}{.}\PY{n}{capitalize}\PY{p}{(}\PY{p}{)} \PY{k}{for} \PY{n}{word} \PY{o+ow}{in} \PY{n}{words}\PY{p}{[}\PY{l+m+mi}{1}\PY{p}{:}\PY{p}{]}\PY{p}{)}
\end{Verbatim}
\end{tcolorbox}

    \hypertarget{clase-distribucionprobabilidad}{%
\subsection{\texorpdfstring{Clase
\texttt{DistribucionProbabilidad}}{Clase DistribucionProbabilidad}}\label{clase-distribucionprobabilidad}}

La clase \texttt{DistribucionProbabilidad} está diseñada para analizar
conjuntos de datos mediante una variedad de métricas estadísticas. Es
útil en estudios de series de tiempo y análisis de datos donde se
requiere un entendimiento profundo de las propiedades estadísticas de
una o más variables.

\hypertarget{muxe9todos-y-muxe9tricas-estaduxedsticas}{%
\subsubsection{Métodos y Métricas
Estadísticas:}\label{muxe9todos-y-muxe9tricas-estaduxedsticas}}

\begin{itemize}
\tightlist
\item
  \textbf{Media}: Calcula el promedio de los valores en el conjunto de
  datos.
\item
  \textbf{Mediana}: Determina el valor medio que divide el conjunto de
  datos en dos partes iguales.
\item
  \textbf{Moda}: Identifica el valor o valores más frecuentes en el
  conjunto de datos.
\item
  \textbf{Media Geométrica}: Calcula la media multiplicativa de los
  valores del conjunto.
\item
  \textbf{Asimetría (Skewness)}: Mide la asimetría de la distribución de
  los datos.
\item
  \textbf{Rango}: La diferencia entre el valor máximo y mínimo en el
  conjunto.
\item
  \textbf{Desviación Estándar}: Mide la cantidad de variación o
  dispersión de los datos.
\item
  \textbf{Varianza}: Calcula la varianza de los datos.
\item
  \textbf{Coeficiente de Variación}: Relaciona la desviación estándar
  con la media, útil para comparar la dispersión entre distribuciones
  con diferentes escalas.
\item
  \textbf{Percentiles y Cuartiles}: Determina valores específicos que
  dividen el conjunto de datos en intervalos iguales.
\item
  \textbf{Curtosis}: Mide la `agudeza' o `achatamiento' de la
  distribución respecto a una distribución normal.
\item
  \textbf{Entropía}: Mide la incertidumbre o la cantidad de información
  `sorpresa' en la distribución de los datos.
\end{itemize}

    Además, se realiza un análisis caótico de cada uno de los modelos
presentados con el objetivo de demostrar que estas series de tiempo
están influenciadas por atractores caóticos en lugar de comportamientos
estocásticos. Para ello, proponemos utilizar las siguientes métricas:

\hypertarget{exponentes-de-lyapunov}{%
\subsection{Exponentes de Lyapunov}\label{exponentes-de-lyapunov}}

\hypertarget{definiciuxf3n}{%
\subsubsection{Definición}\label{definiciuxf3n}}

Los exponentes de Lyapunov son una medida cuantitativa de la
sensibilidad a las condiciones iniciales en un sistema dinámico.
Describen la tasa a la cual dos trayectorias infinitesimalmente cercanas
en el espacio de fases divergen (o convergen) con el tiempo.

\hypertarget{interpretaciuxf3n}{%
\subsubsection{Interpretación}\label{interpretaciuxf3n}}

\begin{itemize}
\tightlist
\item
  \textbf{Exponente de Lyapunov positivo}: Indica que las trayectorias
  divergen exponencialmente, lo que es una característica del
  comportamiento caótico. Cuanto mayor sea el valor positivo, más rápido
  divergen las trayectorias.
\item
  \textbf{Exponente de Lyapunov negativo}: Indica que las trayectorias
  convergen exponencialmente, lo que es típico en sistemas estables.
\item
  \textbf{Exponente de Lyapunov cero}: Indica un comportamiento neutral,
  típico de sistemas en los que las trayectorias son paralelas y no se
  separan ni convergen.
\end{itemize}

\hypertarget{cuxe1lculo}{%
\subsubsection{Cálculo}\label{cuxe1lculo}}

Para calcular el exponente de Lyapunov, se sigue el siguiente
procedimiento:

\begin{enumerate}
\def\labelenumi{\arabic{enumi}.}
\item
  \textbf{División de la serie temporal}: Se divide la serie temporal en
  ventanas de tamaño \texttt{window\_size}.
\item
  \textbf{Cálculo de la derivada del sistema}: En cada punto dentro de
  una ventana, se calcula la derivada del sistema dinámico. Para el mapa
  logístico, la función es:

  \$ x\_\{n+1\} = r x\_n (1 - x\_n) \$

  y su derivada es:

  \$ f'(x) = r (1 - 2x) \$
\item
  \textbf{suma de los logaritmos}: Para cada ventana, se suma el
  logaritmo de los valores absolutos de la derivada:

  \$ \sum\_\{i=1\}\^{}\{N\} \log \left\textbar{} f'(x\_i)
  \right\textbar{} \$
\item
  \textbf{Promedio del suma}: Se promedia esta suma sobre el tamaño de
  la ventana para obtener el exponente de Lyapunov:

  \$ \lambda = \frac{1}{N} \sum\_\{i=1\}\^{}\{N\} \log \left\textbar{}
  f'(x\_i) \right\textbar{} \$
\end{enumerate}

\hypertarget{dimensiuxf3n-de-kaplan-yorke}{%
\subsection{Dimensión de
Kaplan-Yorke}\label{dimensiuxf3n-de-kaplan-yorke}}

\hypertarget{definiciuxf3n-1}{%
\subsubsection{Definición}\label{definiciuxf3n-1}}

La dimensión de Kaplan-Yorke, también conocida como dimensión de
Lyapunov, es una métrica utilizada para caracterizar la estructura
fractal de un atractor caótico en un sistema dinámico. Esta dimensión se
basa en los exponentes de Lyapunov y proporciona una estimación del
número de grados de libertad que intervienen en la dinámica caótica del
sistema.

\hypertarget{fuxf3rmula}{%
\subsubsection{Fórmula}\label{fuxf3rmula}}

La dimensión de Kaplan-Yorke se calcula utilizando los exponentes de
Lyapunov ordenados de mayor a menor (\(\lambda_i\)). La fórmula es la
siguiente:

\$ D\_\{KY\} = j + \frac{\sum_{i=1}^{j} \lambda_i}{|\lambda_{j+1}|} \$

donde: - \(\lambda_i\) son los exponentes de Lyapunov ordenados de mayor
a menor. - \(j\) es el mayor entero tal que la suma de los \(j\)
primeros exponentes de Lyapunov es positiva.

\hypertarget{interpretaciuxf3n-1}{%
\subsubsection{Interpretación}\label{interpretaciuxf3n-1}}

\begin{itemize}
\tightlist
\item
  \textbf{Dimensión alta}: Una dimensión de Kaplan-Yorke alta indica un
  sistema más caótico, con una estructura fractal más compleja y un
  mayor número de grados de libertad.
\item
  \textbf{Dimensión baja}: Una dimensión baja sugiere un comportamiento
  menos caótico y una estructura fractal más simple.
\end{itemize}

\hypertarget{cuxe1lculo-1}{%
\subsubsection{Cálculo}\label{cuxe1lculo-1}}

Para calcular la dimensión de Kaplan-Yorke, se sigue el siguiente
procedimiento:

\begin{enumerate}
\def\labelenumi{\arabic{enumi}.}
\tightlist
\item
  \textbf{Ordenar los exponentes de Lyapunov}: Los exponentes de
  Lyapunov (\(\lambda_i\)) se ordenan de mayor a menor.
\item
  \textbf{Suma acumulativa de los exponentes}: Se calcula la suma
  acumulativa de los exponentes ordenados.
\item
  \textbf{Determinar \(j\)}: Se encuentra el mayor índice \(j\) tal que
  la suma de los \(j\) primeros exponentes de Lyapunov es positiva.
\item
  \textbf{Cálculo final}: Se utiliza la fórmula para obtener la
  dimensión de Kaplan-Yorke.
\end{enumerate}

La dimensión de Kaplan-Yorke proporciona una visión clara de la
naturaleza caótica del sistema y es una herramienta valiosa en el
análisis de atractores caóticos.

\hypertarget{dimensiuxf3n-de-grassberger-procaccia}{%
\subsection{Dimensión de
Grassberger-Procaccia}\label{dimensiuxf3n-de-grassberger-procaccia}}

\hypertarget{definiciuxf3n-2}{%
\subsubsection{Definición}\label{definiciuxf3n-2}}

La dimensión de Grassberger-Procaccia es una medida utilizada en el
análisis de sistemas dinámicos para cuantificar la complejidad de un
atractor extraño. Fue propuesta por Peter Grassberger e Itamar Procaccia
en 1983. Esta dimensión describe cuántas variables independientes son
necesarias para modelar el comportamiento de un sistema dinámico.

\hypertarget{interpretaciuxf3n-2}{%
\subsubsection{Interpretación}\label{interpretaciuxf3n-2}}

\begin{itemize}
\tightlist
\item
  \textbf{Dimensión fractal}: Proporciona una idea sobre la estructura
  fractal de los atractores del sistema. Una dimensión más alta indica
  una mayor complejidad y un comportamiento más caótico del sistema.
\end{itemize}

\hypertarget{cuxe1lculo-2}{%
\subsubsection{Cálculo}\label{cuxe1lculo-2}}

Para calcular la dimensión de Grassberger-Procaccia, se sigue el
siguiente procedimiento:

\begin{enumerate}
\def\labelenumi{\arabic{enumi}.}
\item
  \textbf{Construcción de Embeddings}: A partir de una serie temporal,
  se construyen vectores de dimensión \texttt{m} con un retraso temporal
  \texttt{tau} para reconstruir el espacio de fases. Esto se realiza
  mediante la técnica de embeddings retardados.
\item
  \textbf{Cálculo de la Correlación Integral}: Se calcula la función de
  correlación integral \(C(r)\), que mide la probabilidad de que dos
  puntos en el espacio de fases reconstruido estén a una distancia menor
  que \(r\). La correlación integral se calcula como:

  \$ C(r) = \frac{1}{N(N-1)} \sum\_\{i \ne j\} \Theta(r - \textbar{}
  \mathbf{x}\_i - \mathbf{x}\_j \textbar) \$

  donde \(N\) es el número de puntos en el espacio de fases,
  \(\mathbf{x}_i\) y \(\mathbf{x}_j\) son puntos en el espacio de fases,
  \(\| \cdot \|\) es la distancia euclidiana, y \(\Theta\) es la función
  escalón de Heaviside.
\item
  \textbf{Estimación de la Dimensión Correlativa}: Se grafica \$
  \log(C(r)) \$ frente a \$ \log(r) \$ y se estima la pendiente de la
  región lineal de la gráfica. La pendiente proporciona la dimensión
  correlativa \(D_2\):

  \$ D\_2 = \lim\_\{r \to 0\} \frac{\log C(r)}{\log r} \$

  donde \(D_2\) es la dimensión correlativa, también conocida como la
  dimensión de Grassberger-Procaccia.
\end{enumerate}

\hypertarget{ventajas}{%
\subsubsection{Ventajas}\label{ventajas}}

\begin{itemize}
\tightlist
\item
  \textbf{No linealidad}: Es útil para identificar y cuantificar la no
  linealidad en sistemas caóticos.
\item
  \textbf{Complejidad del sistema}: Permite determinar la complejidad
  del sistema y la mínima cantidad de variables necesarias para
  describirlo.
\end{itemize}

\hypertarget{limitaciones}{%
\subsubsection{Limitaciones}\label{limitaciones}}

\begin{itemize}
\tightlist
\item
  \textbf{Tamaño de los datos}: Requiere una gran cantidad de datos para
  obtener una estimación precisa.
\item
  \textbf{Selección de parámetros}: La elección de los parámetros de
  embedding (dimensión \texttt{m} y retraso \texttt{tau}) puede influir
  en el resultado.
\end{itemize}

La dimensión de Grassberger-Procaccia es una herramienta poderosa para
el análisis de sistemas dinámicos y la identificación de comportamientos
caóticos en series temporales.

    \hypertarget{funciones-adicionales}{%
\subsubsection{Funciones Adicionales:}\label{funciones-adicionales}}

\begin{itemize}
\tightlist
\item
  \textbf{\texttt{calcular\_metricas\_individual(i)}:} Calcula todas las
  métricas estadísticas para la i-ésima variable y devuelve un
  diccionario con los resultados.
\item
  \textbf{\texttt{mostrar\_metricas()}:} Imprime las métricas calculadas
  para todas las variables almacenadas en la clase. Si hay múltiples
  variables, también calcula y muestra la matriz de correlación y
  gráficos de dispersión entre todas las combinaciones de variables.
\item
  \textbf{\texttt{grafico\_dispersion()}:} Genera gráficos de dispersión
  para visualizar las relaciones entre diferentes pares de variables,
  útil para identificar correlaciones visuales.
\end{itemize}

    \begin{tcolorbox}[breakable, size=fbox, boxrule=1pt, pad at break*=1mm,colback=cellbackground, colframe=cellborder]
\prompt{In}{incolor}{511}{\boxspacing}
\begin{Verbatim}[commandchars=\\\{\}]
\PY{k}{class} \PY{n+nc}{DistribucionProbabilidad}\PY{p}{:}
    \PY{k}{def} \PY{n+nf+fm}{\PYZus{}\PYZus{}init\PYZus{}\PYZus{}}\PY{p}{(}\PY{n+nb+bp}{self}\PY{p}{,} \PY{o}{*}\PY{n}{args}\PY{p}{)}\PY{p}{:}
        \PY{n+nb+bp}{self}\PY{o}{.}\PY{n}{datos} \PY{o}{=} \PY{p}{[}\PY{n}{np}\PY{o}{.}\PY{n}{array}\PY{p}{(}\PY{n}{arg}\PY{p}{)} \PY{k}{for} \PY{n}{arg} \PY{o+ow}{in} \PY{n}{args}\PY{p}{]}
        \PY{n+nb+bp}{self}\PY{o}{.}\PY{n}{n\PYZus{}vars} \PY{o}{=} \PY{n+nb}{len}\PY{p}{(}\PY{n}{args}\PY{p}{)}
    
    \PY{k}{def} \PY{n+nf}{media}\PY{p}{(}\PY{n+nb+bp}{self}\PY{p}{,} \PY{n}{i}\PY{p}{)}\PY{p}{:}
        \PY{k}{return} \PY{n}{np}\PY{o}{.}\PY{n}{mean}\PY{p}{(}\PY{n+nb+bp}{self}\PY{o}{.}\PY{n}{datos}\PY{p}{[}\PY{n}{i}\PY{p}{]}\PY{p}{)}
    
    \PY{k}{def} \PY{n+nf}{mediana}\PY{p}{(}\PY{n+nb+bp}{self}\PY{p}{,} \PY{n}{i}\PY{p}{)}\PY{p}{:}
        \PY{k}{return} \PY{n}{np}\PY{o}{.}\PY{n}{median}\PY{p}{(}\PY{n+nb+bp}{self}\PY{o}{.}\PY{n}{datos}\PY{p}{[}\PY{n}{i}\PY{p}{]}\PY{p}{)}
    
    \PY{k}{def} \PY{n+nf}{moda}\PY{p}{(}\PY{n+nb+bp}{self}\PY{p}{,} \PY{n}{i}\PY{p}{)}\PY{p}{:}
        \PY{n}{mode\PYZus{}res} \PY{o}{=} \PY{n}{stats}\PY{o}{.}\PY{n}{mode}\PY{p}{(}\PY{n+nb+bp}{self}\PY{o}{.}\PY{n}{datos}\PY{p}{[}\PY{n}{i}\PY{p}{]}\PY{p}{)}
        \PY{k}{if} \PY{n}{np}\PY{o}{.}\PY{n}{isscalar}\PY{p}{(}\PY{n}{mode\PYZus{}res}\PY{o}{.}\PY{n}{count}\PY{p}{)}\PY{p}{:}
            \PY{k}{return} \PY{n}{mode\PYZus{}res}\PY{o}{.}\PY{n}{mode}
        \PY{k}{else}\PY{p}{:}
            \PY{k}{return} \PY{n}{mode\PYZus{}res}\PY{o}{.}\PY{n}{mode} \PY{k}{if} \PY{n}{mode\PYZus{}res}\PY{o}{.}\PY{n}{count}\PY{p}{[}\PY{l+m+mi}{0}\PY{p}{]} \PY{o}{\PYZgt{}} \PY{l+m+mi}{1} \PY{k}{else} \PY{n}{mode\PYZus{}res}\PY{o}{.}\PY{n}{mode}\PY{p}{[}\PY{l+m+mi}{0}\PY{p}{]}
    
    \PY{k}{def} \PY{n+nf}{media\PYZus{}geometrica}\PY{p}{(}\PY{n+nb+bp}{self}\PY{p}{,} \PY{n}{i}\PY{p}{)}\PY{p}{:}
        \PY{k}{return} \PY{n}{gmean}\PY{p}{(}\PY{n+nb+bp}{self}\PY{o}{.}\PY{n}{datos}\PY{p}{[}\PY{n}{i}\PY{p}{]}\PY{p}{)}
    
    \PY{k}{def} \PY{n+nf}{asimetria}\PY{p}{(}\PY{n+nb+bp}{self}\PY{p}{,} \PY{n}{i}\PY{p}{)}\PY{p}{:}
        \PY{k}{return} \PY{n}{skew}\PY{p}{(}\PY{n+nb+bp}{self}\PY{o}{.}\PY{n}{datos}\PY{p}{[}\PY{n}{i}\PY{p}{]}\PY{p}{)}
    
    \PY{k}{def} \PY{n+nf}{rango}\PY{p}{(}\PY{n+nb+bp}{self}\PY{p}{,} \PY{n}{i}\PY{p}{)}\PY{p}{:}
        \PY{k}{return} \PY{n}{np}\PY{o}{.}\PY{n}{max}\PY{p}{(}\PY{n+nb+bp}{self}\PY{o}{.}\PY{n}{datos}\PY{p}{[}\PY{n}{i}\PY{p}{]}\PY{p}{)} \PY{o}{\PYZhy{}} \PY{n}{np}\PY{o}{.}\PY{n}{min}\PY{p}{(}\PY{n+nb+bp}{self}\PY{o}{.}\PY{n}{datos}\PY{p}{[}\PY{n}{i}\PY{p}{]}\PY{p}{)}
    
    \PY{k}{def} \PY{n+nf}{desviacion\PYZus{}estandar}\PY{p}{(}\PY{n+nb+bp}{self}\PY{p}{,} \PY{n}{i}\PY{p}{)}\PY{p}{:}
        \PY{k}{return} \PY{n}{np}\PY{o}{.}\PY{n}{std}\PY{p}{(}\PY{n+nb+bp}{self}\PY{o}{.}\PY{n}{datos}\PY{p}{[}\PY{n}{i}\PY{p}{]}\PY{p}{,} \PY{n}{ddof}\PY{o}{=}\PY{l+m+mi}{1}\PY{p}{)}
    
    \PY{k}{def} \PY{n+nf}{varianza}\PY{p}{(}\PY{n+nb+bp}{self}\PY{p}{,} \PY{n}{i}\PY{p}{)}\PY{p}{:}
        \PY{k}{return} \PY{n}{np}\PY{o}{.}\PY{n}{var}\PY{p}{(}\PY{n+nb+bp}{self}\PY{o}{.}\PY{n}{datos}\PY{p}{[}\PY{n}{i}\PY{p}{]}\PY{p}{,} \PY{n}{ddof}\PY{o}{=}\PY{l+m+mi}{1}\PY{p}{)}
    
    \PY{k}{def} \PY{n+nf}{coeficiente\PYZus{}variacion}\PY{p}{(}\PY{n+nb+bp}{self}\PY{p}{,} \PY{n}{i}\PY{p}{)}\PY{p}{:}
        \PY{k}{return} \PY{n+nb+bp}{self}\PY{o}{.}\PY{n}{desviacion\PYZus{}estandar}\PY{p}{(}\PY{n}{i}\PY{p}{)} \PY{o}{/} \PY{n+nb+bp}{self}\PY{o}{.}\PY{n}{media}\PY{p}{(}\PY{n}{i}\PY{p}{)}
    
    \PY{k}{def} \PY{n+nf}{percentil}\PY{p}{(}\PY{n+nb+bp}{self}\PY{p}{,} \PY{n}{p}\PY{p}{)}\PY{p}{:}
        \PY{k}{if} \PY{l+m+mi}{0} \PY{o}{\PYZlt{}}\PY{o}{=} \PY{n}{p} \PY{o}{\PYZlt{}}\PY{o}{=} \PY{l+m+mi}{100}\PY{p}{:}
            \PY{k}{return} \PY{n}{np}\PY{o}{.}\PY{n}{percentile}\PY{p}{(}\PY{n+nb+bp}{self}\PY{o}{.}\PY{n}{datos}\PY{p}{,} \PY{n}{p}\PY{p}{)}
        \PY{k}{else}\PY{p}{:}
            \PY{k}{raise} \PY{n+ne}{ValueError}\PY{p}{(}\PY{l+s+s2}{\PYZdq{}}\PY{l+s+s2}{El percentil debe estar entre 0 y 100.}\PY{l+s+s2}{\PYZdq{}}\PY{p}{)}
    
    \PY{k}{def} \PY{n+nf}{cuartil}\PY{p}{(}\PY{n+nb+bp}{self}\PY{p}{,} \PY{n}{q}\PY{p}{)}\PY{p}{:}
        \PY{k}{if} \PY{n}{q} \PY{o+ow}{in} \PY{p}{[}\PY{l+m+mi}{1}\PY{p}{,} \PY{l+m+mi}{2}\PY{p}{,} \PY{l+m+mi}{3}\PY{p}{]}\PY{p}{:}
            \PY{k}{return} \PY{n+nb+bp}{self}\PY{o}{.}\PY{n}{percentil}\PY{p}{(}\PY{n}{q} \PY{o}{*} \PY{l+m+mi}{25}\PY{p}{)}
        \PY{k}{else}\PY{p}{:}
            \PY{k}{return} \PY{n}{np}\PY{o}{.}\PY{n}{nan}
    
    \PY{k}{def} \PY{n+nf}{curtosis}\PY{p}{(}\PY{n+nb+bp}{self}\PY{p}{,} \PY{n}{i}\PY{p}{)}\PY{p}{:}
        \PY{k}{return} \PY{n}{kurtosis}\PY{p}{(}\PY{n+nb+bp}{self}\PY{o}{.}\PY{n}{datos}\PY{p}{[}\PY{n}{i}\PY{p}{]}\PY{p}{)}
    
    \PY{k}{def} \PY{n+nf}{entropia}\PY{p}{(}\PY{n+nb+bp}{self}\PY{p}{,} \PY{n}{i}\PY{p}{)}\PY{p}{:}
        \PY{n}{p}\PY{p}{,} \PY{n}{counts} \PY{o}{=} \PY{n}{np}\PY{o}{.}\PY{n}{unique}\PY{p}{(}\PY{n+nb+bp}{self}\PY{o}{.}\PY{n}{datos}\PY{p}{[}\PY{n}{i}\PY{p}{]}\PY{p}{,} \PY{n}{return\PYZus{}counts}\PY{o}{=}\PY{k+kc}{True}\PY{p}{)}
        \PY{n}{p} \PY{o}{=} \PY{n}{counts} \PY{o}{/} \PY{n+nb}{len}\PY{p}{(}\PY{n+nb+bp}{self}\PY{o}{.}\PY{n}{datos}\PY{p}{[}\PY{n}{i}\PY{p}{]}\PY{p}{)}
        \PY{k}{return} \PY{o}{\PYZhy{}}\PY{n}{np}\PY{o}{.}\PY{n}{sum}\PY{p}{(}\PY{n}{p} \PY{o}{*} \PY{n}{np}\PY{o}{.}\PY{n}{log}\PY{p}{(}\PY{n}{p}\PY{p}{)}\PY{p}{)}
    
    \PY{k}{def} \PY{n+nf}{calcular\PYZus{}metricas\PYZus{}individual}\PY{p}{(}\PY{n+nb+bp}{self}\PY{p}{,} \PY{n}{i}\PY{p}{)}\PY{p}{:}
        \PY{n}{resultados} \PY{o}{=} \PY{p}{\PYZob{}}
            \PY{l+s+s1}{\PYZsq{}}\PY{l+s+s1}{Media}\PY{l+s+s1}{\PYZsq{}}\PY{p}{:} \PY{n+nb+bp}{self}\PY{o}{.}\PY{n}{media}\PY{p}{(}\PY{n}{i}\PY{p}{)}\PY{p}{,}
            \PY{l+s+s1}{\PYZsq{}}\PY{l+s+s1}{Mediana}\PY{l+s+s1}{\PYZsq{}}\PY{p}{:} \PY{n+nb+bp}{self}\PY{o}{.}\PY{n}{mediana}\PY{p}{(}\PY{n}{i}\PY{p}{)}\PY{p}{,}
            \PY{l+s+s1}{\PYZsq{}}\PY{l+s+s1}{Moda}\PY{l+s+s1}{\PYZsq{}}\PY{p}{:} \PY{n+nb+bp}{self}\PY{o}{.}\PY{n}{moda}\PY{p}{(}\PY{n}{i}\PY{p}{)}\PY{p}{,}
            \PY{l+s+s1}{\PYZsq{}}\PY{l+s+s1}{Media Geométrica}\PY{l+s+s1}{\PYZsq{}}\PY{p}{:} \PY{n+nb+bp}{self}\PY{o}{.}\PY{n}{media\PYZus{}geometrica}\PY{p}{(}\PY{n}{i}\PY{p}{)}\PY{p}{,}
            \PY{l+s+s1}{\PYZsq{}}\PY{l+s+s1}{Rango}\PY{l+s+s1}{\PYZsq{}}\PY{p}{:} \PY{n+nb+bp}{self}\PY{o}{.}\PY{n}{rango}\PY{p}{(}\PY{n}{i}\PY{p}{)}\PY{p}{,}
            \PY{l+s+s1}{\PYZsq{}}\PY{l+s+s1}{Desviación Estándar}\PY{l+s+s1}{\PYZsq{}}\PY{p}{:} \PY{n+nb+bp}{self}\PY{o}{.}\PY{n}{desviacion\PYZus{}estandar}\PY{p}{(}\PY{n}{i}\PY{p}{)}\PY{p}{,}
            \PY{l+s+s1}{\PYZsq{}}\PY{l+s+s1}{Varianza}\PY{l+s+s1}{\PYZsq{}}\PY{p}{:} \PY{n+nb+bp}{self}\PY{o}{.}\PY{n}{varianza}\PY{p}{(}\PY{n}{i}\PY{p}{)}\PY{p}{,}
            \PY{l+s+s1}{\PYZsq{}}\PY{l+s+s1}{Asimetría}\PY{l+s+s1}{\PYZsq{}}\PY{p}{:} \PY{n+nb+bp}{self}\PY{o}{.}\PY{n}{asimetria}\PY{p}{(}\PY{n}{i}\PY{p}{)}\PY{p}{,}
            \PY{l+s+s1}{\PYZsq{}}\PY{l+s+s1}{Curtosis}\PY{l+s+s1}{\PYZsq{}}\PY{p}{:} \PY{n+nb+bp}{self}\PY{o}{.}\PY{n}{curtosis}\PY{p}{(}\PY{n}{i}\PY{p}{)}\PY{p}{,}
            \PY{l+s+s1}{\PYZsq{}}\PY{l+s+s1}{Entropia}\PY{l+s+s1}{\PYZsq{}}\PY{p}{:} \PY{n+nb+bp}{self}\PY{o}{.}\PY{n}{entropia}\PY{p}{(}\PY{n}{i}\PY{p}{)}\PY{p}{,}
            \PY{l+s+s1}{\PYZsq{}}\PY{l+s+s1}{Coeficiente de Variación}\PY{l+s+s1}{\PYZsq{}}\PY{p}{:} \PY{n+nb+bp}{self}\PY{o}{.}\PY{n}{coeficiente\PYZus{}variacion}\PY{p}{(}\PY{n}{i}\PY{p}{)}
        \PY{p}{\PYZcb{}}
        \PY{k}{return} \PY{n}{resultados}
    
    \PY{k}{def} \PY{n+nf}{mostrar\PYZus{}metricas}\PY{p}{(}\PY{n+nb+bp}{self}\PY{p}{)}\PY{p}{:}
        \PY{k}{for} \PY{n}{index}\PY{p}{,} \PY{n}{datos} \PY{o+ow}{in} \PY{n+nb}{enumerate}\PY{p}{(}\PY{n+nb+bp}{self}\PY{o}{.}\PY{n}{datos}\PY{p}{)}\PY{p}{:}
            \PY{n+nb}{print}\PY{p}{(}\PY{l+s+sa}{f}\PY{l+s+s2}{\PYZdq{}}\PY{l+s+s2}{\PYZhy{}\PYZhy{}\PYZhy{} Métricas para Variable }\PY{l+s+si}{\PYZob{}}\PY{n}{index}\PY{+w}{ }\PY{o}{+}\PY{+w}{ }\PY{l+m+mi}{1}\PY{l+s+si}{\PYZcb{}}\PY{l+s+s2}{ \PYZhy{}\PYZhy{}\PYZhy{}}\PY{l+s+s2}{\PYZdq{}}\PY{p}{)}
            \PY{n}{metricas} \PY{o}{=} \PY{n+nb+bp}{self}\PY{o}{.}\PY{n}{calcular\PYZus{}metricas\PYZus{}individual}\PY{p}{(}\PY{n}{index}\PY{p}{)}
            \PY{k}{for} \PY{n}{metrica}\PY{p}{,} \PY{n}{valor} \PY{o+ow}{in} \PY{n}{metricas}\PY{o}{.}\PY{n}{items}\PY{p}{(}\PY{p}{)}\PY{p}{:}
                \PY{n+nb}{print}\PY{p}{(}\PY{l+s+sa}{f}\PY{l+s+s2}{\PYZdq{}}\PY{l+s+si}{\PYZob{}}\PY{n}{metrica}\PY{l+s+si}{\PYZcb{}}\PY{l+s+s2}{: }\PY{l+s+si}{\PYZob{}}\PY{n}{valor}\PY{l+s+si}{\PYZcb{}}\PY{l+s+s2}{\PYZdq{}}\PY{p}{)}
            \PY{n+nb}{print}\PY{p}{(}\PY{l+s+s2}{\PYZdq{}}\PY{l+s+se}{\PYZbs{}n}\PY{l+s+s2}{\PYZdq{}}\PY{p}{,} \PY{n}{end}\PY{o}{=}\PY{l+s+s2}{\PYZdq{}}\PY{l+s+s2}{\PYZdq{}}\PY{p}{)}

        \PY{k}{if} \PY{n+nb+bp}{self}\PY{o}{.}\PY{n}{n\PYZus{}vars} \PY{o}{\PYZgt{}} \PY{l+m+mi}{1}\PY{p}{:}
            \PY{n+nb+bp}{self}\PY{o}{.}\PY{n}{mostrar\PYZus{}correlacion\PYZus{}y\PYZus{}graficos}\PY{p}{(}\PY{p}{)}
    
    \PY{k}{def} \PY{n+nf}{mostrar\PYZus{}correlacion\PYZus{}y\PYZus{}graficos}\PY{p}{(}\PY{n+nb+bp}{self}\PY{p}{)}\PY{p}{:}
        \PY{n+nb}{print}\PY{p}{(}\PY{l+s+s2}{\PYZdq{}}\PY{l+s+s2}{Matriz de Correlación:}\PY{l+s+s2}{\PYZdq{}}\PY{p}{)}
        \PY{n+nb}{print}\PY{p}{(}\PY{n}{np}\PY{o}{.}\PY{n}{corrcoef}\PY{p}{(}\PY{n+nb+bp}{self}\PY{o}{.}\PY{n}{datos}\PY{p}{)}\PY{p}{)}
        \PY{n+nb+bp}{self}\PY{o}{.}\PY{n}{grafico\PYZus{}dispersion}\PY{p}{(}\PY{p}{)}

    \PY{k}{def} \PY{n+nf}{grafico\PYZus{}dispersion}\PY{p}{(}\PY{n+nb+bp}{self}\PY{p}{)}\PY{p}{:}
        \PY{n}{plt}\PY{o}{.}\PY{n}{figure}\PY{p}{(}\PY{n}{figsize}\PY{o}{=}\PY{p}{(}\PY{l+m+mi}{10}\PY{p}{,} \PY{l+m+mi}{8}\PY{p}{)}\PY{p}{)}
        \PY{k}{for} \PY{n}{i} \PY{o+ow}{in} \PY{n+nb}{range}\PY{p}{(}\PY{n+nb+bp}{self}\PY{o}{.}\PY{n}{n\PYZus{}vars}\PY{p}{)}\PY{p}{:}
            \PY{k}{for} \PY{n}{j} \PY{o+ow}{in} \PY{n+nb}{range}\PY{p}{(}\PY{n}{i} \PY{o}{+} \PY{l+m+mi}{1}\PY{p}{,} \PY{n+nb+bp}{self}\PY{o}{.}\PY{n}{n\PYZus{}vars}\PY{p}{)}\PY{p}{:}
                \PY{n}{plt}\PY{o}{.}\PY{n}{subplot}\PY{p}{(}\PY{n+nb+bp}{self}\PY{o}{.}\PY{n}{n\PYZus{}vars} \PY{o}{\PYZhy{}} \PY{l+m+mi}{1}\PY{p}{,} \PY{n+nb+bp}{self}\PY{o}{.}\PY{n}{n\PYZus{}vars} \PY{o}{\PYZhy{}} \PY{l+m+mi}{1}\PY{p}{,} \PY{n}{i} \PY{o}{*} \PY{p}{(}\PY{n+nb+bp}{self}\PY{o}{.}\PY{n}{n\PYZus{}vars} \PY{o}{\PYZhy{}} \PY{l+m+mi}{1}\PY{p}{)} \PY{o}{+} \PY{n}{j}\PY{p}{)}
                \PY{n}{plt}\PY{o}{.}\PY{n}{scatter}\PY{p}{(}\PY{n+nb+bp}{self}\PY{o}{.}\PY{n}{datos}\PY{p}{[}\PY{n}{i}\PY{p}{]}\PY{p}{,} \PY{n+nb+bp}{self}\PY{o}{.}\PY{n}{datos}\PY{p}{[}\PY{n}{j}\PY{p}{]}\PY{p}{,} \PY{n}{alpha}\PY{o}{=}\PY{l+m+mf}{0.6}\PY{p}{,} \PY{n}{s}\PY{o}{=}\PY{l+m+mf}{0.1}\PY{p}{)}
                \PY{n}{plt}\PY{o}{.}\PY{n}{title}\PY{p}{(}\PY{l+s+sa}{f}\PY{l+s+s2}{\PYZdq{}}\PY{l+s+s2}{Dispersion }\PY{l+s+si}{\PYZob{}}\PY{n}{i}\PY{o}{+}\PY{l+m+mi}{1}\PY{l+s+si}{\PYZcb{}}\PY{l+s+s2}{ vs }\PY{l+s+si}{\PYZob{}}\PY{n}{j}\PY{o}{+}\PY{l+m+mi}{1}\PY{l+s+si}{\PYZcb{}}\PY{l+s+s2}{\PYZdq{}}\PY{p}{)}
                \PY{n}{plt}\PY{o}{.}\PY{n}{xlabel}\PY{p}{(}\PY{l+s+sa}{f}\PY{l+s+s2}{\PYZdq{}}\PY{l+s+s2}{Variable }\PY{l+s+si}{\PYZob{}}\PY{n}{i}\PY{o}{+}\PY{l+m+mi}{1}\PY{l+s+si}{\PYZcb{}}\PY{l+s+s2}{\PYZdq{}}\PY{p}{)}
                \PY{n}{plt}\PY{o}{.}\PY{n}{ylabel}\PY{p}{(}\PY{l+s+sa}{f}\PY{l+s+s2}{\PYZdq{}}\PY{l+s+s2}{Variable }\PY{l+s+si}{\PYZob{}}\PY{n}{j}\PY{o}{+}\PY{l+m+mi}{1}\PY{l+s+si}{\PYZcb{}}\PY{l+s+s2}{\PYZdq{}}\PY{p}{)}
        \PY{n}{plt}\PY{o}{.}\PY{n}{tight\PYZus{}layout}\PY{p}{(}\PY{p}{)}
        \PY{n}{plt}\PY{o}{.}\PY{n}{show}\PY{p}{(}\PY{p}{)}
\end{Verbatim}
\end{tcolorbox}

    \begin{tcolorbox}[breakable, size=fbox, boxrule=1pt, pad at break*=1mm,colback=cellbackground, colframe=cellborder]
\prompt{In}{incolor}{512}{\boxspacing}
\begin{Verbatim}[commandchars=\\\{\}]
\PY{k}{def} \PY{n+nf}{plotear\PYZus{}hist}\PY{p}{(}\PY{n}{array}\PY{p}{:} \PY{n}{np}\PY{o}{.}\PY{n}{ndarray}\PY{p}{,} \PY{n}{titulo}\PY{p}{:} \PY{n+nb}{str}\PY{p}{,} \PY{n}{label\PYZus{}x}\PY{p}{:} \PY{n+nb}{str}\PY{p}{,} \PY{n}{label\PYZus{}y}\PY{p}{:} \PY{n+nb}{str}\PY{p}{,} \PY{n}{criterio}\PY{p}{:} \PY{n+nb}{str} \PY{o}{=} \PY{l+s+s1}{\PYZsq{}}\PY{l+s+s1}{sturges}\PY{l+s+s1}{\PYZsq{}}\PY{p}{,} \PY{n}{guardar}\PY{o}{=}\PY{k+kc}{False}\PY{p}{)} \PY{o}{\PYZhy{}}\PY{o}{\PYZgt{}} \PY{k+kc}{None}\PY{p}{:}
\PY{+w}{    }\PY{l+s+sd}{\PYZdq{}\PYZdq{}\PYZdq{}}
\PY{l+s+sd}{    Genera y guarda un histograma con estilos personalizados, colores aleatorios para cada barra,}
\PY{l+s+sd}{    y el número de bins determinado por el criterio especificado.}

\PY{l+s+sd}{    Args:}
\PY{l+s+sd}{        array (np.ndarray): Array de Numpy con los datos que se quieren plasmar en el histograma.}
\PY{l+s+sd}{        titulo (str): Título del histograma.}
\PY{l+s+sd}{        label\PYZus{}x (str): Etiqueta del eje x del histograma.}
\PY{l+s+sd}{        label\PYZus{}y (str): Etiqueta del eje y del histograma.}
\PY{l+s+sd}{        ruta\PYZus{}img (str): Ruta donde se guardará la imagen del histograma.}
\PY{l+s+sd}{        criterio (str): Método para calcular el número de bins (\PYZsq{}sturges\PYZsq{}, \PYZsq{}freedman\PYZhy{}diaconis\PYZsq{}, \PYZsq{}scott\PYZsq{}, \PYZsq{}raiz\PYZus{}cuadrada\PYZsq{}, \PYZsq{}rice\PYZsq{}).}

\PY{l+s+sd}{    Returns:}
\PY{l+s+sd}{        None: La función no retorna nada.}
\PY{l+s+sd}{    \PYZdq{}\PYZdq{}\PYZdq{}}
    
    \PY{k}{match} \PY{n}{criterio}\PY{p}{:}
        \PY{k}{case} \PY{l+s+s1}{\PYZsq{}}\PY{l+s+s1}{sturges}\PY{l+s+s1}{\PYZsq{}}\PY{p}{:}
            \PY{n}{bins} \PY{o}{=} \PY{n+nb}{int}\PY{p}{(}\PY{l+m+mi}{1} \PY{o}{+} \PY{n}{np}\PY{o}{.}\PY{n}{log2}\PY{p}{(}\PY{n+nb}{len}\PY{p}{(}\PY{n}{array}\PY{p}{)}\PY{p}{)}\PY{p}{)}
        \PY{k}{case} \PY{l+s+s1}{\PYZsq{}}\PY{l+s+s1}{freedman\PYZhy{}diaconis}\PY{l+s+s1}{\PYZsq{}}\PY{p}{:}
            \PY{n}{iqr} \PY{o}{=} \PY{n}{np}\PY{o}{.}\PY{n}{subtract}\PY{p}{(}\PY{o}{*}\PY{n}{np}\PY{o}{.}\PY{n}{percentile}\PY{p}{(}\PY{n}{array}\PY{p}{,} \PY{p}{[}\PY{l+m+mi}{75}\PY{p}{,} \PY{l+m+mi}{25}\PY{p}{]}\PY{p}{)}\PY{p}{)}
            \PY{n}{bin\PYZus{}width} \PY{o}{=} \PY{l+m+mi}{2} \PY{o}{*} \PY{n}{iqr} \PY{o}{*} \PY{n+nb}{len}\PY{p}{(}\PY{n}{array}\PY{p}{)} \PY{o}{*}\PY{o}{*} \PY{p}{(}\PY{o}{\PYZhy{}}\PY{l+m+mi}{1}\PY{o}{/}\PY{l+m+mi}{3}\PY{p}{)}
            \PY{n}{bins} \PY{o}{=} \PY{n+nb}{int}\PY{p}{(}\PY{n}{np}\PY{o}{.}\PY{n}{ptp}\PY{p}{(}\PY{n}{array}\PY{p}{)} \PY{o}{/} \PY{n}{bin\PYZus{}width}\PY{p}{)}
        \PY{k}{case} \PY{l+s+s1}{\PYZsq{}}\PY{l+s+s1}{scott}\PY{l+s+s1}{\PYZsq{}}\PY{p}{:}
            \PY{n}{bin\PYZus{}width} \PY{o}{=} \PY{l+m+mf}{3.5} \PY{o}{*} \PY{n}{np}\PY{o}{.}\PY{n}{std}\PY{p}{(}\PY{n}{array}\PY{p}{)} \PY{o}{*} \PY{n+nb}{len}\PY{p}{(}\PY{n}{array}\PY{p}{)} \PY{o}{*}\PY{o}{*} \PY{p}{(}\PY{o}{\PYZhy{}}\PY{l+m+mi}{1}\PY{o}{/}\PY{l+m+mi}{3}\PY{p}{)}
            \PY{n}{bins} \PY{o}{=} \PY{n+nb}{int}\PY{p}{(}\PY{n}{np}\PY{o}{.}\PY{n}{ptp}\PY{p}{(}\PY{n}{array}\PY{p}{)} \PY{o}{/} \PY{n}{bin\PYZus{}width}\PY{p}{)}
        \PY{k}{case} \PY{l+s+s1}{\PYZsq{}}\PY{l+s+s1}{raiz\PYZus{}cuadrada}\PY{l+s+s1}{\PYZsq{}}\PY{p}{:}
            \PY{n}{bins} \PY{o}{=} \PY{n+nb}{int}\PY{p}{(}\PY{n}{np}\PY{o}{.}\PY{n}{sqrt}\PY{p}{(}\PY{n+nb}{len}\PY{p}{(}\PY{n}{array}\PY{p}{)}\PY{p}{)}\PY{p}{)}
        \PY{k}{case} \PY{l+s+s1}{\PYZsq{}}\PY{l+s+s1}{rice}\PY{l+s+s1}{\PYZsq{}}\PY{p}{:}
            \PY{n}{bins} \PY{o}{=} \PY{n+nb}{int}\PY{p}{(}\PY{l+m+mi}{2} \PY{o}{*} \PY{n+nb}{len}\PY{p}{(}\PY{n}{array}\PY{p}{)} \PY{o}{*}\PY{o}{*} \PY{p}{(}\PY{l+m+mi}{1}\PY{o}{/}\PY{l+m+mi}{3}\PY{p}{)}\PY{p}{)}
        \PY{k}{case}\PY{+w}{ }\PY{k}{\PYZus{}}\PY{p}{:}
            \PY{k}{raise} \PY{n+ne}{ValueError}\PY{p}{(}\PY{l+s+s2}{\PYZdq{}}\PY{l+s+s2}{Criterio no reconocido. Usa }\PY{l+s+s2}{\PYZsq{}}\PY{l+s+s2}{sturges}\PY{l+s+s2}{\PYZsq{}}\PY{l+s+s2}{, }\PY{l+s+s2}{\PYZsq{}}\PY{l+s+s2}{freedman\PYZhy{}diaconis}\PY{l+s+s2}{\PYZsq{}}\PY{l+s+s2}{, }\PY{l+s+s2}{\PYZsq{}}\PY{l+s+s2}{scott}\PY{l+s+s2}{\PYZsq{}}\PY{l+s+s2}{, }\PY{l+s+s2}{\PYZsq{}}\PY{l+s+s2}{raiz\PYZus{}cuadrada}\PY{l+s+s2}{\PYZsq{}}\PY{l+s+s2}{, o }\PY{l+s+s2}{\PYZsq{}}\PY{l+s+s2}{rice}\PY{l+s+s2}{\PYZsq{}}\PY{l+s+s2}{.}\PY{l+s+s2}{\PYZdq{}}\PY{p}{)}

    \PY{n}{n}\PY{p}{,} \PY{n}{bins}\PY{p}{,} \PY{n}{patches} \PY{o}{=} \PY{n}{plt}\PY{o}{.}\PY{n}{hist}\PY{p}{(}\PY{n}{array}\PY{p}{,} \PY{n}{bins}\PY{o}{=}\PY{n}{bins}\PY{p}{,} \PY{n}{alpha}\PY{o}{=}\PY{l+m+mf}{0.75}\PY{p}{,} \PY{n}{rwidth}\PY{o}{=}\PY{l+m+mf}{0.85}\PY{p}{)}

    \PY{k}{for} \PY{n}{patch} \PY{o+ow}{in} \PY{n}{patches}\PY{p}{:}
        \PY{n}{plt}\PY{o}{.}\PY{n}{setp}\PY{p}{(}\PY{n}{patch}\PY{p}{,} \PY{l+s+s1}{\PYZsq{}}\PY{l+s+s1}{facecolor}\PY{l+s+s1}{\PYZsq{}}\PY{p}{,} \PY{n}{np}\PY{o}{.}\PY{n}{random}\PY{o}{.}\PY{n}{rand}\PY{p}{(}\PY{l+m+mi}{3}\PY{p}{,}\PY{p}{)}\PY{p}{)}
        
    \PY{n}{plt}\PY{o}{.}\PY{n}{grid}\PY{p}{(}\PY{n}{axis}\PY{o}{=}\PY{l+s+s1}{\PYZsq{}}\PY{l+s+s1}{y}\PY{l+s+s1}{\PYZsq{}}\PY{p}{,} \PY{n}{linestyle}\PY{o}{=}\PY{l+s+s1}{\PYZsq{}}\PY{l+s+s1}{\PYZhy{}\PYZhy{}}\PY{l+s+s1}{\PYZsq{}}\PY{p}{,} \PY{n}{alpha}\PY{o}{=}\PY{l+m+mf}{0.6}\PY{p}{)}
    \PY{n}{plt}\PY{o}{.}\PY{n}{title}\PY{p}{(}\PY{n}{titulo}\PY{p}{,} \PY{n}{fontsize}\PY{o}{=}\PY{l+m+mi}{20}\PY{p}{,} \PY{n}{fontweight}\PY{o}{=}\PY{l+s+s1}{\PYZsq{}}\PY{l+s+s1}{bold}\PY{l+s+s1}{\PYZsq{}}\PY{p}{,} \PY{n}{color}\PY{o}{=}\PY{n}{np}\PY{o}{.}\PY{n}{random}\PY{o}{.}\PY{n}{rand}\PY{p}{(}\PY{l+m+mi}{3}\PY{p}{,}\PY{p}{)}\PY{p}{)}
    \PY{n}{plt}\PY{o}{.}\PY{n}{xlabel}\PY{p}{(}\PY{n}{label\PYZus{}x}\PY{p}{,} \PY{n}{fontsize}\PY{o}{=}\PY{l+m+mi}{15}\PY{p}{,} \PY{n}{fontstyle}\PY{o}{=}\PY{l+s+s1}{\PYZsq{}}\PY{l+s+s1}{italic}\PY{l+s+s1}{\PYZsq{}}\PY{p}{,} \PY{n}{color}\PY{o}{=}\PY{n}{np}\PY{o}{.}\PY{n}{random}\PY{o}{.}\PY{n}{rand}\PY{p}{(}\PY{l+m+mi}{3}\PY{p}{,}\PY{p}{)}\PY{p}{)}
    \PY{n}{plt}\PY{o}{.}\PY{n}{ylabel}\PY{p}{(}\PY{n}{label\PYZus{}y}\PY{p}{,} \PY{n}{fontsize}\PY{o}{=}\PY{l+m+mi}{15}\PY{p}{,} \PY{n}{fontstyle}\PY{o}{=}\PY{l+s+s1}{\PYZsq{}}\PY{l+s+s1}{italic}\PY{l+s+s1}{\PYZsq{}}\PY{p}{,} \PY{n}{color}\PY{o}{=}\PY{n}{np}\PY{o}{.}\PY{n}{random}\PY{o}{.}\PY{n}{rand}\PY{p}{(}\PY{l+m+mi}{3}\PY{p}{,}\PY{p}{)}\PY{p}{)}
    \PY{n}{plt}\PY{o}{.}\PY{n}{ylim}\PY{p}{(}\PY{l+m+mi}{0}\PY{p}{,} \PY{n+nb}{max}\PY{p}{(}\PY{n}{n}\PY{p}{)}\PY{o}{*}\PY{l+m+mf}{1.1}\PY{p}{)}

    \PY{k}{if} \PY{n}{guardar}\PY{p}{:}
        \PY{n}{ruta\PYZus{}img} \PY{o}{=} \PY{l+s+sa}{f}\PY{l+s+s2}{\PYZdq{}}\PY{l+s+si}{\PYZob{}}\PY{n}{convertir\PYZus{}camelCase}\PY{p}{(}\PY{n}{titulo}\PY{p}{)}\PY{l+s+si}{\PYZcb{}}\PY{l+s+s2}{.pdf}\PY{l+s+s2}{\PYZdq{}}
        \PY{n}{plt}\PY{o}{.}\PY{n}{savefig}\PY{p}{(}\PY{n}{ruta\PYZus{}img}\PY{p}{,} \PY{n+nb}{format}\PY{o}{=}\PY{l+s+s1}{\PYZsq{}}\PY{l+s+s1}{pdf}\PY{l+s+s1}{\PYZsq{}}\PY{p}{,} \PY{n}{bbox\PYZus{}inches}\PY{o}{=}\PY{l+s+s1}{\PYZsq{}}\PY{l+s+s1}{tight}\PY{l+s+s1}{\PYZsq{}}\PY{p}{)}
    
    \PY{n}{plt}\PY{o}{.}\PY{n}{show}\PY{p}{(}\PY{p}{)}
\end{Verbatim}
\end{tcolorbox}

    \begin{tcolorbox}[breakable, size=fbox, boxrule=1pt, pad at break*=1mm,colback=cellbackground, colframe=cellborder]
\prompt{In}{incolor}{513}{\boxspacing}
\begin{Verbatim}[commandchars=\\\{\}]
\PY{k}{def} \PY{n+nf}{kaplan\PYZus{}yorke\PYZus{}dimension}\PY{p}{(}\PY{n}{lyapunov\PYZus{}exponents}\PY{p}{)}\PY{p}{:}
    \PY{n}{sorted\PYZus{}exponents} \PY{o}{=} \PY{n+nb}{sorted}\PY{p}{(}\PY{n}{lyapunov\PYZus{}exponents}\PY{p}{,} \PY{n}{reverse}\PY{o}{=}\PY{k+kc}{True}\PY{p}{)}
    \PY{n}{sum\PYZus{}exponents} \PY{o}{=} \PY{l+m+mi}{0}
    
    \PY{k}{for} \PY{n}{i}\PY{p}{,} \PY{n}{exponent} \PY{o+ow}{in} \PY{n+nb}{enumerate}\PY{p}{(}\PY{n}{sorted\PYZus{}exponents}\PY{p}{)}\PY{p}{:}
        \PY{n}{sum\PYZus{}exponents} \PY{o}{+}\PY{o}{=} \PY{n}{exponent}
        \PY{k}{if} \PY{n}{sum\PYZus{}exponents} \PY{o}{\PYZlt{}} \PY{l+m+mi}{0}\PY{p}{:}
            \PY{n}{k} \PY{o}{=} \PY{n}{i}
            \PY{k}{break}
    \PY{k}{else}\PY{p}{:}
        \PY{k}{return} \PY{n+nb}{len}\PY{p}{(}\PY{n}{sorted\PYZus{}exponents}\PY{p}{)}
    
    \PY{k}{if} \PY{n}{k} \PY{o}{==} \PY{l+m+mi}{0}\PY{p}{:}
        \PY{k}{return} \PY{l+m+mi}{0}
    \PY{k}{else}\PY{p}{:}
        \PY{k}{return} \PY{n}{k} \PY{o}{+} \PY{n}{sum\PYZus{}exponents} \PY{o}{/} \PY{n+nb}{abs}\PY{p}{(}\PY{n}{sorted\PYZus{}exponents}\PY{p}{[}\PY{n}{k}\PY{p}{]}\PY{p}{)}
\end{Verbatim}
\end{tcolorbox}

    \begin{tcolorbox}[breakable, size=fbox, boxrule=1pt, pad at break*=1mm,colback=cellbackground, colframe=cellborder]
\prompt{In}{incolor}{514}{\boxspacing}
\begin{Verbatim}[commandchars=\\\{\}]
\PY{k}{def} \PY{n+nf}{graficar}\PY{p}{(}\PY{n}{x}\PY{p}{,} \PY{n}{t}\PY{p}{,} \PY{n}{plot\PYZus{}type}\PY{o}{=}\PY{l+s+s1}{\PYZsq{}}\PY{l+s+s1}{scatter}\PY{l+s+s1}{\PYZsq{}}\PY{p}{,} \PY{n}{width}\PY{o}{=}\PY{l+m+mi}{15}\PY{p}{,} \PY{n}{height}\PY{o}{=}\PY{l+m+mi}{10}\PY{p}{,} \PY{n}{save\PYZus{}as\PYZus{}pdf}\PY{o}{=}\PY{k+kc}{False}\PY{p}{,} \PY{n}{titulo}\PY{o}{=}\PY{l+s+s2}{\PYZdq{}}\PY{l+s+s2}{Diagrama de bifurcación cúbica de Feigenbaum}\PY{l+s+s2}{\PYZdq{}}\PY{p}{)}\PY{p}{:}
\PY{+w}{    }\PY{l+s+sd}{\PYZdq{}\PYZdq{}\PYZdq{}}
\PY{l+s+sd}{    Crea un gráfico utilizando Matplotlib con estilo personalizado y márgenes ajustados.}
\PY{l+s+sd}{    \PYZdq{}\PYZdq{}\PYZdq{}}
    \PY{n}{plt}\PY{o}{.}\PY{n}{rcParams}\PY{p}{[}\PY{l+s+s1}{\PYZsq{}}\PY{l+s+s1}{axes.facecolor}\PY{l+s+s1}{\PYZsq{}}\PY{p}{]} \PY{o}{=} \PY{l+s+s1}{\PYZsq{}}\PY{l+s+s1}{\PYZsh{}e9f0fb}\PY{l+s+s1}{\PYZsq{}}
    \PY{n}{plt}\PY{o}{.}\PY{n}{rcParams}\PY{p}{[}\PY{l+s+s1}{\PYZsq{}}\PY{l+s+s1}{grid.color}\PY{l+s+s1}{\PYZsq{}}\PY{p}{]} \PY{o}{=} \PY{l+s+s1}{\PYZsq{}}\PY{l+s+s1}{white}\PY{l+s+s1}{\PYZsq{}}
    \PY{n}{plt}\PY{o}{.}\PY{n}{rcParams}\PY{p}{[}\PY{l+s+s1}{\PYZsq{}}\PY{l+s+s1}{grid.linestyle}\PY{l+s+s1}{\PYZsq{}}\PY{p}{]} \PY{o}{=} \PY{l+s+s1}{\PYZsq{}}\PY{l+s+s1}{\PYZhy{}}\PY{l+s+s1}{\PYZsq{}}
    \PY{n}{plt}\PY{o}{.}\PY{n}{rcParams}\PY{p}{[}\PY{l+s+s1}{\PYZsq{}}\PY{l+s+s1}{grid.linewidth}\PY{l+s+s1}{\PYZsq{}}\PY{p}{]} \PY{o}{=} \PY{l+m+mf}{1.5}
    \PY{n}{plt}\PY{o}{.}\PY{n}{rcParams}\PY{p}{[}\PY{l+s+s1}{\PYZsq{}}\PY{l+s+s1}{font.size}\PY{l+s+s1}{\PYZsq{}}\PY{p}{]} \PY{o}{=} \PY{l+m+mi}{10}
    \PY{n}{plt}\PY{o}{.}\PY{n}{rcParams}\PY{p}{[}\PY{l+s+s1}{\PYZsq{}}\PY{l+s+s1}{font.family}\PY{l+s+s1}{\PYZsq{}}\PY{p}{]} \PY{o}{=} \PY{l+s+s1}{\PYZsq{}}\PY{l+s+s1}{sans\PYZhy{}serif}\PY{l+s+s1}{\PYZsq{}}
    \PY{n}{plt}\PY{o}{.}\PY{n}{rcParams}\PY{p}{[}\PY{l+s+s1}{\PYZsq{}}\PY{l+s+s1}{text.color}\PY{l+s+s1}{\PYZsq{}}\PY{p}{]} \PY{o}{=} \PY{l+s+s1}{\PYZsq{}}\PY{l+s+s1}{black}\PY{l+s+s1}{\PYZsq{}}

    \PY{n}{fig}\PY{p}{,} \PY{n}{ax} \PY{o}{=} \PY{n}{plt}\PY{o}{.}\PY{n}{subplots}\PY{p}{(}\PY{n}{figsize}\PY{o}{=}\PY{p}{(}\PY{n}{width}\PY{o}{*}\PY{l+m+mf}{1.5}\PY{p}{,} \PY{n}{height}\PY{o}{*}\PY{l+m+mf}{1.5}\PY{p}{)}\PY{p}{)}
    \PY{n}{fig}\PY{o}{.}\PY{n}{subplots\PYZus{}adjust}\PY{p}{(}\PY{n}{left}\PY{o}{=}\PY{l+m+mf}{0.15}\PY{p}{,} \PY{n}{right}\PY{o}{=}\PY{l+m+mi}{1}\PY{p}{,} \PY{n}{top}\PY{o}{=}\PY{l+m+mf}{0.85}\PY{p}{,} \PY{n}{bottom}\PY{o}{=}\PY{l+m+mf}{0.15}\PY{p}{)}

    \PY{c+c1}{\PYZsh{} Crear el gráfico}
    \PY{k}{if} \PY{n}{plot\PYZus{}type} \PY{o}{==} \PY{l+s+s1}{\PYZsq{}}\PY{l+s+s1}{scatter}\PY{l+s+s1}{\PYZsq{}}\PY{p}{:}
        \PY{n}{ax}\PY{o}{.}\PY{n}{scatter}\PY{p}{(}\PY{n}{t}\PY{p}{,} \PY{n}{x}\PY{p}{,} \PY{n}{color}\PY{o}{=}\PY{l+s+s1}{\PYZsq{}}\PY{l+s+s1}{blue}\PY{l+s+s1}{\PYZsq{}}\PY{p}{,} \PY{n}{marker}\PY{o}{=}\PY{l+s+s1}{\PYZsq{}}\PY{l+s+s1}{o}\PY{l+s+s1}{\PYZsq{}}\PY{p}{,} \PY{n}{s}\PY{o}{=}\PY{l+m+mf}{0.1}\PY{p}{)}
    \PY{k}{elif} \PY{n}{plot\PYZus{}type} \PY{o}{==} \PY{l+s+s1}{\PYZsq{}}\PY{l+s+s1}{line}\PY{l+s+s1}{\PYZsq{}}\PY{p}{:}
        \PY{n}{ax}\PY{o}{.}\PY{n}{plot}\PY{p}{(}\PY{n}{t}\PY{p}{,} \PY{n}{x}\PY{p}{,} \PY{n}{color}\PY{o}{=}\PY{l+s+s1}{\PYZsq{}}\PY{l+s+s1}{blue}\PY{l+s+s1}{\PYZsq{}}\PY{p}{,} \PY{n}{linewidth}\PY{o}{=}\PY{l+m+mi}{1}\PY{p}{)}

    \PY{n}{ax}\PY{o}{.}\PY{n}{set\PYZus{}title}\PY{p}{(}\PY{n}{titulo}\PY{p}{,} \PY{n}{fontsize}\PY{o}{=}\PY{l+m+mi}{16}\PY{p}{,} \PY{n}{loc}\PY{o}{=}\PY{l+s+s1}{\PYZsq{}}\PY{l+s+s1}{left}\PY{l+s+s1}{\PYZsq{}}\PY{p}{,} \PY{n}{pad}\PY{o}{=}\PY{l+m+mi}{20}\PY{p}{,} \PY{n}{color}\PY{o}{=}\PY{l+s+s1}{\PYZsq{}}\PY{l+s+s1}{black}\PY{l+s+s1}{\PYZsq{}}\PY{p}{)}
    \PY{n}{ax}\PY{o}{.}\PY{n}{set\PYZus{}xlabel}\PY{p}{(}\PY{l+s+s1}{\PYZsq{}}\PY{l+s+s1}{Tasa de crecimiento t}\PY{l+s+s1}{\PYZsq{}}\PY{p}{,} \PY{n}{fontsize}\PY{o}{=}\PY{l+m+mi}{13}\PY{p}{,} \PY{n}{labelpad}\PY{o}{=}\PY{l+m+mi}{15}\PY{p}{,} \PY{n}{color}\PY{o}{=}\PY{l+s+s1}{\PYZsq{}}\PY{l+s+s1}{black}\PY{l+s+s1}{\PYZsq{}}\PY{p}{)}
    \PY{n}{ax}\PY{o}{.}\PY{n}{set\PYZus{}ylabel}\PY{p}{(}\PY{l+s+s1}{\PYZsq{}}\PY{l+s+s1}{Valor de x}\PY{l+s+s1}{\PYZsq{}}\PY{p}{,} \PY{n}{fontsize}\PY{o}{=}\PY{l+m+mi}{13}\PY{p}{,} \PY{n}{labelpad}\PY{o}{=}\PY{l+m+mi}{15}\PY{p}{,} \PY{n}{color}\PY{o}{=}\PY{l+s+s1}{\PYZsq{}}\PY{l+s+s1}{black}\PY{l+s+s1}{\PYZsq{}}\PY{p}{)}
    \PY{n}{ax}\PY{o}{.}\PY{n}{tick\PYZus{}params}\PY{p}{(}\PY{n}{axis}\PY{o}{=}\PY{l+s+s1}{\PYZsq{}}\PY{l+s+s1}{both}\PY{l+s+s1}{\PYZsq{}}\PY{p}{,} \PY{n}{which}\PY{o}{=}\PY{l+s+s1}{\PYZsq{}}\PY{l+s+s1}{major}\PY{l+s+s1}{\PYZsq{}}\PY{p}{,} \PY{n}{labelsize}\PY{o}{=}\PY{l+m+mi}{10}\PY{p}{)}

    \PY{k}{if} \PY{n}{save\PYZus{}as\PYZus{}pdf}\PY{p}{:}
        \PY{n}{plt}\PY{o}{.}\PY{n}{savefig}\PY{p}{(}\PY{l+s+sa}{f}\PY{l+s+s2}{\PYZdq{}}\PY{l+s+si}{\PYZob{}}\PY{n}{titulo}\PY{o}{.}\PY{n}{replace}\PY{p}{(}\PY{l+s+s1}{\PYZsq{}}\PY{l+s+s1}{ }\PY{l+s+s1}{\PYZsq{}}\PY{p}{,}\PY{+w}{ }\PY{l+s+s1}{\PYZsq{}}\PY{l+s+s1}{\PYZus{}}\PY{l+s+s1}{\PYZsq{}}\PY{p}{)}\PY{l+s+si}{\PYZcb{}}\PY{l+s+s2}{.pdf}\PY{l+s+s2}{\PYZdq{}}\PY{p}{,} \PY{n+nb}{format}\PY{o}{=}\PY{l+s+s1}{\PYZsq{}}\PY{l+s+s1}{pdf}\PY{l+s+s1}{\PYZsq{}}\PY{p}{,} \PY{n}{dpi}\PY{o}{=}\PY{l+m+mi}{300}\PY{p}{)}

    \PY{n}{plt}\PY{o}{.}\PY{n}{show}\PY{p}{(}\PY{p}{)}
\end{Verbatim}
\end{tcolorbox}

    \begin{tcolorbox}[breakable, size=fbox, boxrule=1pt, pad at break*=1mm,colback=cellbackground, colframe=cellborder]
\prompt{In}{incolor}{7}{\boxspacing}
\begin{Verbatim}[commandchars=\\\{\}]
\PY{k}{def} \PY{n+nf}{graficar\PYZus{}3d}\PY{p}{(}\PY{n}{x}\PY{p}{,} \PY{n}{y}\PY{p}{,} \PY{n}{z}\PY{p}{,} \PY{n}{width}\PY{o}{=}\PY{l+m+mi}{10}\PY{p}{,} \PY{n}{height}\PY{o}{=}\PY{l+m+mi}{7}\PY{p}{,} \PY{n}{titulo}\PY{o}{=}\PY{l+s+s1}{\PYZsq{}}\PY{l+s+s1}{Figura2}\PY{l+s+s1}{\PYZsq{}}\PY{p}{)}\PY{p}{:}
    \PY{n}{file\PYZus{}name} \PY{o}{=} \PY{l+s+sa}{f}\PY{l+s+s2}{\PYZdq{}}\PY{l+s+si}{\PYZob{}}\PY{n}{convertir\PYZus{}camelCase}\PY{p}{(}\PY{n}{titulo}\PY{p}{)}\PY{l+s+si}{\PYZcb{}}\PY{l+s+s2}{.pdf}\PY{l+s+s2}{\PYZdq{}}
    
    \PY{n}{figura} \PY{o}{=} \PY{n}{go}\PY{o}{.}\PY{n}{Figure}\PY{p}{(}\PY{n}{data}\PY{o}{=}\PY{p}{[}\PY{n}{go}\PY{o}{.}\PY{n}{Scatter3d}\PY{p}{(}\PY{n}{x}\PY{o}{=}\PY{n}{x}\PY{p}{,} \PY{n}{y}\PY{o}{=}\PY{n}{y}\PY{p}{,} \PY{n}{z}\PY{o}{=}\PY{n}{z}\PY{p}{,} \PY{n}{mode}\PY{o}{=}\PY{l+s+s1}{\PYZsq{}}\PY{l+s+s1}{lines}\PY{l+s+s1}{\PYZsq{}}\PY{p}{)}\PY{p}{]}\PY{p}{)}
    
    \PY{n}{figura}\PY{o}{.}\PY{n}{update\PYZus{}layout}\PY{p}{(}
        \PY{n}{title}\PY{o}{=}\PY{n}{titulo}\PY{p}{,}
        \PY{n}{width}\PY{o}{=}\PY{n}{width}\PY{o}{*}\PY{l+m+mi}{100}\PY{p}{,}
        \PY{n}{height}\PY{o}{=}\PY{n}{height}\PY{o}{*}\PY{l+m+mi}{100}\PY{p}{,}
        \PY{n}{scene}\PY{o}{=}\PY{n+nb}{dict}\PY{p}{(}
            \PY{n}{xaxis}\PY{o}{=}\PY{n+nb}{dict}\PY{p}{(}\PY{n}{title}\PY{o}{=}\PY{l+s+s1}{\PYZsq{}}\PY{l+s+s1}{X\PYZhy{}axis}\PY{l+s+s1}{\PYZsq{}}\PY{p}{)}\PY{p}{,}
            \PY{n}{yaxis}\PY{o}{=}\PY{n+nb}{dict}\PY{p}{(}\PY{n}{title}\PY{o}{=}\PY{l+s+s1}{\PYZsq{}}\PY{l+s+s1}{Y\PYZhy{}axis}\PY{l+s+s1}{\PYZsq{}}\PY{p}{)}\PY{p}{,}
            \PY{n}{zaxis}\PY{o}{=}\PY{n+nb}{dict}\PY{p}{(}\PY{n}{title}\PY{o}{=}\PY{l+s+s1}{\PYZsq{}}\PY{l+s+s1}{Z\PYZhy{}axis}\PY{l+s+s1}{\PYZsq{}}\PY{p}{)}\PY{p}{,}
            \PY{n}{camera}\PY{o}{=}\PY{n+nb}{dict}\PY{p}{(}
                \PY{n}{eye}\PY{o}{=}\PY{n+nb}{dict}\PY{p}{(}\PY{n}{x}\PY{o}{=}\PY{l+m+mf}{1.5}\PY{p}{,} \PY{n}{y}\PY{o}{=}\PY{o}{\PYZhy{}}\PY{l+m+mf}{1.3}\PY{p}{,} \PY{n}{z}\PY{o}{=}\PY{l+m+mf}{0.5}\PY{p}{)}\PY{p}{,}
                \PY{n}{center}\PY{o}{=}\PY{n+nb}{dict}\PY{p}{(}\PY{n}{x}\PY{o}{=}\PY{l+m+mi}{0}\PY{p}{,} \PY{n}{y}\PY{o}{=}\PY{l+m+mi}{0}\PY{p}{,} \PY{n}{z}\PY{o}{=}\PY{l+m+mi}{0}\PY{p}{)}\PY{p}{,}
                \PY{n}{up}\PY{o}{=}\PY{n+nb}{dict}\PY{p}{(}\PY{n}{x}\PY{o}{=}\PY{l+m+mi}{0}\PY{p}{,} \PY{n}{y}\PY{o}{=}\PY{l+m+mi}{0}\PY{p}{,} \PY{n}{z}\PY{o}{=}\PY{l+m+mi}{1}\PY{p}{)}
            \PY{p}{)}
        \PY{p}{)}\PY{p}{,}
        \PY{n}{scene\PYZus{}aspectmode}\PY{o}{=}\PY{l+s+s1}{\PYZsq{}}\PY{l+s+s1}{cube}\PY{l+s+s1}{\PYZsq{}}\PY{p}{,}
        \PY{n}{margin}\PY{o}{=}\PY{n+nb}{dict}\PY{p}{(}\PY{n}{t}\PY{o}{=}\PY{l+m+mi}{50}\PY{p}{)}
    \PY{p}{)}
    
    \PY{n}{pio}\PY{o}{.}\PY{n}{write\PYZus{}image}\PY{p}{(}\PY{n}{figura}\PY{p}{,} \PY{n}{file\PYZus{}name}\PY{p}{,} \PY{n+nb}{format}\PY{o}{=}\PY{l+s+s1}{\PYZsq{}}\PY{l+s+s1}{pdf}\PY{l+s+s1}{\PYZsq{}}\PY{p}{)}
    \PY{n}{figura}\PY{o}{.}\PY{n}{show}\PY{p}{(}\PY{p}{)}
\end{Verbatim}
\end{tcolorbox}

    \begin{tcolorbox}[breakable, size=fbox, boxrule=1pt, pad at break*=1mm,colback=cellbackground, colframe=cellborder]
\prompt{In}{incolor}{5}{\boxspacing}
\begin{Verbatim}[commandchars=\\\{\}]
\PY{k}{def} \PY{n+nf}{leer\PYZus{}col\PYZus{}csv}\PY{p}{(}\PY{n}{file\PYZus{}path}\PY{p}{,} \PY{n}{column\PYZus{}name}\PY{p}{)}\PY{p}{:}
    \PY{n}{data} \PY{o}{=} \PY{n}{pd}\PY{o}{.}\PY{n}{read\PYZus{}csv}\PY{p}{(}\PY{n}{file\PYZus{}path}\PY{p}{)}
    \PY{n}{column\PYZus{}data} \PY{o}{=} \PY{n}{data}\PY{p}{[}\PY{n}{column\PYZus{}name}\PY{p}{]}
    \PY{k}{return} \PY{n}{np}\PY{o}{.}\PY{n}{array}\PY{p}{(}\PY{n}{column\PYZus{}data}\PY{p}{)}
\end{Verbatim}
\end{tcolorbox}

    \hypertarget{bifurcaciuxf3n-de-feigenbaum}{%
\section{Bifurcación de Feigenbaum}\label{bifurcaciuxf3n-de-feigenbaum}}

    \begin{tcolorbox}[breakable, size=fbox, boxrule=1pt, pad at break*=1mm,colback=cellbackground, colframe=cellborder]
\prompt{In}{incolor}{517}{\boxspacing}
\begin{Verbatim}[commandchars=\\\{\}]
\PY{n}{valores\PYZus{}x} \PY{o}{=} \PY{n}{leer\PYZus{}col\PYZus{}csv}\PY{p}{(}\PY{l+s+s2}{\PYZdq{}}\PY{l+s+s2}{datosFeigenbaum.csv}\PY{l+s+s2}{\PYZdq{}}\PY{p}{,} \PY{l+s+s2}{\PYZdq{}}\PY{l+s+s2}{Valores x}\PY{l+s+s2}{\PYZdq{}}\PY{p}{)}
\PY{n}{valores\PYZus{}k} \PY{o}{=} \PY{n+nb}{range}\PY{p}{(}\PY{l+m+mi}{1}\PY{p}{,} \PY{n+nb}{len}\PY{p}{(}\PY{n}{valores\PYZus{}x}\PY{p}{)}\PY{o}{+}\PY{l+m+mi}{1}\PY{p}{)}
\PY{n}{graficar}\PY{p}{(}\PY{n}{valores\PYZus{}x}\PY{p}{,} \PY{n}{valores\PYZus{}k}\PY{p}{,} \PY{n}{width}\PY{o}{=}\PY{l+m+mi}{10}\PY{p}{,} \PY{n}{height}\PY{o}{=}\PY{l+m+mi}{7} \PY{p}{,}\PY{n}{titulo}\PY{o}{=}\PY{l+s+s2}{\PYZdq{}}\PY{l+s+s2}{Bifurcaciones de Feigenbaum}\PY{l+s+s2}{\PYZdq{}}\PY{p}{)}
\end{Verbatim}
\end{tcolorbox}

    \begin{center}
    \adjustimage{max size={0.9\linewidth}{0.9\paperheight}}{analisisCaos2_files/analisisCaos2_13_0.png}
    \end{center}
    { \hspace*{\fill} \\}
    
    \hypertarget{probabiluxedstico}{%
\subsection{Probabilístico}\label{probabiluxedstico}}

    \begin{tcolorbox}[breakable, size=fbox, boxrule=1pt, pad at break*=1mm,colback=cellbackground, colframe=cellborder]
\prompt{In}{incolor}{518}{\boxspacing}
\begin{Verbatim}[commandchars=\\\{\}]
\PY{n}{feigen} \PY{o}{=} \PY{n}{DistribucionProbabilidad}\PY{p}{(}\PY{n}{valores\PYZus{}x}\PY{p}{)}
\PY{n}{feigen}\PY{o}{.}\PY{n}{mostrar\PYZus{}metricas}\PY{p}{(}\PY{p}{)}
\PY{n}{plotear\PYZus{}hist}\PY{p}{(}\PY{n}{valores\PYZus{}x}\PY{p}{,} \PY{l+s+s2}{\PYZdq{}}\PY{l+s+s2}{Histograma de Feigenbaum}\PY{l+s+s2}{\PYZdq{}}\PY{p}{,} \PY{l+s+s2}{\PYZdq{}}\PY{l+s+s2}{Valor}\PY{l+s+s2}{\PYZdq{}}\PY{p}{,} \PY{l+s+s2}{\PYZdq{}}\PY{l+s+s2}{Frecuencia}\PY{l+s+s2}{\PYZdq{}}\PY{p}{,} \PY{l+s+s2}{\PYZdq{}}\PY{l+s+s2}{sturges}\PY{l+s+s2}{\PYZdq{}}\PY{p}{)}
\end{Verbatim}
\end{tcolorbox}

    \begin{Verbatim}[commandchars=\\\{\}]
--- Métricas para Variable 1 ---
Media: 0.638408580608167
Mediana: 0.6630207856104768
Moda: 0.00044602653658490046
Media Geométrica: 0.5866915605238021
Rango: 0.9994424435173831
Desviación Estándar: 0.21597657197735343
Varianza: 0.04664587964308893
Asimetría: -0.47948989536842423
Curtosis: -0.6073557991822351
Entropia: 11.512925464970223
Coeficiente de Variación: 0.3383046195456955

    \end{Verbatim}

    \begin{center}
    \adjustimage{max size={0.9\linewidth}{0.9\paperheight}}{analisisCaos2_files/analisisCaos2_15_1.png}
    \end{center}
    { \hspace*{\fill} \\}
    
    \hypertarget{anuxe1lisis-de-muxe9tricas-estaduxedsticas-para-el-modelo-de-bifurcaciones-de-feigenbaum}{%
\subsection{Análisis de Métricas Estadísticas para el Modelo de
Bifurcaciones de
Feigenbaum}\label{anuxe1lisis-de-muxe9tricas-estaduxedsticas-para-el-modelo-de-bifurcaciones-de-feigenbaum}}

\begin{itemize}
\item
  \textbf{Media (0.637)}: La media está más cerca del extremo superior
  del intervalo {[}0,1{]}, lo que indica una tendencia de los valores a
  ser relativamente altos. Esto puede sugerir que hay regiones dentro
  del rango de parámetros donde el comportamiento tiende a estabilizarse
  en valores más altos.
\item
  \textbf{Mediana (0.6393)}: Al estar muy cerca de la media, refuerza la
  idea de que la distribución de los valores puede ser simétrica
  alrededor de este punto medio, aunque esto no es concluyente por sí
  solo.
\item
  \textbf{Moda (0.000217509)}: La moda es significativamente baja, lo
  que indica que el valor más frecuente en los datos es cercano a 0.
  Esto puede ser indicativo de que hay una concentración de iteraciones
  que convergen a valores bajos, posiblemente representando estabilidad
  temporal en esas regiones.
\item
  \textbf{Media Geométrica (0.6021)}: La media geométrica, siendo menor
  que la media aritmética, sugiere una distribución asimétrica con una
  cola hacia valores menores.
\item
  \textbf{Rango (0.9997)}: Un rango muy cercano al máximo teórico
  {[}0,1{]} indica una amplia dispersión de los datos a lo largo del
  intervalo completo, característico de un sistema que experimenta
  dinámicas desde estables a caóticas.
\item
  \textbf{Desviación Estándar (0.1769)} y \textbf{Varianza (0.0313)}:
  Ambos indican una variabilidad considerable en los datos, lo cual es
  típico en sistemas caóticos donde pequeñas diferencias en condiciones
  iniciales o parámetros pueden llevar a grandes diferencias en el
  comportamiento.
\item
  \textbf{Asimetría (-0.5606)}: Una asimetría negativa sugiere una cola
  más pesada hacia valores más bajos. Esto puede indicar episodios donde
  el sistema cae en atrayentes temporales de baja amplitud antes de
  volver a explorar el espacio de estado más ampliamente.
\item
  \textbf{Curtosis (0.5446)}: La curtosis positiva indica una
  distribución más puntiaguda que una normal, sugiriendo un
  comportamiento de clustering de valores con colas gruesas, típico en
  dinámicas caóticas.
\item
  \textbf{Entropía (11.9184)}: La alta entropía refleja una alta
  incertidumbre y diversidad en los valores de la variable, reafirmando
  el comportamiento caótico y la sensibilidad a condiciones iniciales.
\item
  \textbf{Coeficiente de Variación (0.2771)}: Este valor, siendo
  relativamente bajo, sugiere que la desviación estándar es pequeña en
  comparación con la media, indicando una dispersión proporcionalmente
  moderada en relación con el nivel de la media.
\end{itemize}

\hypertarget{conclusiuxf3n}{%
\subsubsection{Conclusión}\label{conclusiuxf3n}}

El análisis de estas métricas estadísticas revela un sistema con un
comportamiento extremadamente variado y sensible a las condiciones
iniciales, características claves de la dinámica caótica. Los patrones
observados en las métricas como la moda, asimetría y curtosis son
particularmente útiles para discernir la naturaleza no lineal y no
periódica del sistema modelado.

    \hypertarget{cauxf3tico}{%
\subsection{Caótico}\label{cauxf3tico}}

    \hypertarget{exponentes-de-lyapounov}{%
\subsubsection{Exponentes de Lyapounov}\label{exponentes-de-lyapounov}}

    \begin{tcolorbox}[breakable, size=fbox, boxrule=1pt, pad at break*=1mm,colback=cellbackground, colframe=cellborder]
\prompt{In}{incolor}{519}{\boxspacing}
\begin{Verbatim}[commandchars=\\\{\}]
\PY{k}{def} \PY{n+nf}{leer\PYZus{}csv\PYZus{}a\PYZus{}numpy}\PY{p}{(}\PY{n}{file\PYZus{}path}\PY{p}{)}\PY{p}{:}
    \PY{n}{df} \PY{o}{=} \PY{n}{pd}\PY{o}{.}\PY{n}{read\PYZus{}csv}\PY{p}{(}\PY{n}{file\PYZus{}path}\PY{p}{)}
    \PY{n}{valores\PYZus{}x} \PY{o}{=} \PY{n}{df}\PY{p}{[}\PY{l+s+s1}{\PYZsq{}}\PY{l+s+s1}{Valores x}\PY{l+s+s1}{\PYZsq{}}\PY{p}{]}\PY{o}{.}\PY{n}{to\PYZus{}numpy}\PY{p}{(}\PY{p}{)}
    \PY{n}{valores\PYZus{}r} \PY{o}{=} \PY{n}{df}\PY{p}{[}\PY{l+s+s1}{\PYZsq{}}\PY{l+s+s1}{Valores r}\PY{l+s+s1}{\PYZsq{}}\PY{p}{]}\PY{o}{.}\PY{n}{to\PYZus{}numpy}\PY{p}{(}\PY{p}{)}
    \PY{k}{return} \PY{n}{valores\PYZus{}x}\PY{p}{,} \PY{n}{valores\PYZus{}r}


\PY{k}{def} \PY{n+nf}{lyapunov\PYZus{}spectrum\PYZus{}n}\PY{p}{(}\PY{n}{datos}\PY{p}{,} \PY{n}{valores\PYZus{}r}\PY{p}{,} \PY{n}{window\PYZus{}size}\PY{p}{)}\PY{p}{:}
\PY{+w}{    }\PY{l+s+sd}{\PYZdq{}\PYZdq{}\PYZdq{}}
\PY{l+s+sd}{    Calcula los exponentes de Lyapunov para el sistema logístico de Feigenbaum.}

\PY{l+s+sd}{    :param datos: Serie temporal del sistema de Feigenbaum.}
\PY{l+s+sd}{    :param valores\PYZus{}r: Array de valores de r (uno para cada punto de la serie temporal).}
\PY{l+s+sd}{    :param window\PYZus{}size: Tamaño de la ventana para el cálculo del exponente de Lyapunov.}
\PY{l+s+sd}{    :return: Array con los exponentes de Lyapunov.}
\PY{l+s+sd}{    \PYZdq{}\PYZdq{}\PYZdq{}}
    \PY{n}{n} \PY{o}{=} \PY{n+nb}{len}\PY{p}{(}\PY{n}{datos}\PY{p}{)}
    \PY{n}{n\PYZus{}windows} \PY{o}{=} \PY{n}{n} \PY{o}{/}\PY{o}{/} \PY{n}{window\PYZus{}size}
    \PY{n}{exponentes\PYZus{}lyapunov} \PY{o}{=} \PY{p}{[}\PY{p}{]}

    \PY{k}{for} \PY{n}{j} \PY{o+ow}{in} \PY{n+nb}{range}\PY{p}{(}\PY{n}{n\PYZus{}windows}\PY{p}{)}\PY{p}{:}
        \PY{n}{suma\PYZus{}logaritmos\PYZus{}derivadas} \PY{o}{=} \PY{l+m+mi}{0}
        \PY{k}{for} \PY{n}{i} \PY{o+ow}{in} \PY{n+nb}{range}\PY{p}{(}\PY{n}{window\PYZus{}size}\PY{p}{)}\PY{p}{:}
            \PY{n}{x\PYZus{}k} \PY{o}{=} \PY{n}{datos}\PY{p}{[}\PY{n}{j} \PY{o}{*} \PY{n}{window\PYZus{}size} \PY{o}{+} \PY{n}{i}\PY{p}{]}
            \PY{n}{r\PYZus{}val} \PY{o}{=} \PY{n}{valores\PYZus{}r}\PY{p}{[}\PY{n}{j} \PY{o}{*} \PY{n}{window\PYZus{}size} \PY{o}{+} \PY{n}{i}\PY{p}{]}
            \PY{n}{derivada} \PY{o}{=} \PY{n}{r\PYZus{}val} \PY{o}{*} \PY{p}{(}\PY{l+m+mi}{1} \PY{o}{\PYZhy{}} \PY{l+m+mi}{2} \PY{o}{*} \PY{n}{x\PYZus{}k}\PY{p}{)}
            \PY{n}{suma\PYZus{}logaritmos\PYZus{}derivadas} \PY{o}{+}\PY{o}{=} \PY{n}{np}\PY{o}{.}\PY{n}{log}\PY{p}{(}\PY{n+nb}{abs}\PY{p}{(}\PY{n}{derivada}\PY{p}{)}\PY{p}{)}
        
        \PY{n}{exponente\PYZus{}lyapunov} \PY{o}{=} \PY{p}{(}\PY{l+m+mi}{1} \PY{o}{/} \PY{n}{window\PYZus{}size}\PY{p}{)} \PY{o}{*} \PY{n}{suma\PYZus{}logaritmos\PYZus{}derivadas}
        \PY{n}{exponentes\PYZus{}lyapunov}\PY{o}{.}\PY{n}{append}\PY{p}{(}\PY{n}{exponente\PYZus{}lyapunov}\PY{p}{)}

    \PY{k}{return} \PY{n}{np}\PY{o}{.}\PY{n}{array}\PY{p}{(}\PY{n}{exponentes\PYZus{}lyapunov}\PY{p}{)}
\end{Verbatim}
\end{tcolorbox}

    \begin{tcolorbox}[breakable, size=fbox, boxrule=1pt, pad at break*=1mm,colback=cellbackground, colframe=cellborder]
\prompt{In}{incolor}{520}{\boxspacing}
\begin{Verbatim}[commandchars=\\\{\}]
\PY{n}{window\PYZus{}size} \PY{o}{=} \PY{l+m+mi}{50}
\PY{n}{valores\PYZus{}x}\PY{p}{,} \PY{n}{valores\PYZus{}r} \PY{o}{=} \PY{n}{leer\PYZus{}csv\PYZus{}a\PYZus{}numpy}\PY{p}{(}\PY{l+s+s2}{\PYZdq{}}\PY{l+s+s2}{datosFeigenbaum.csv}\PY{l+s+s2}{\PYZdq{}}\PY{p}{)}
\PY{n}{pasos} \PY{o}{=} \PY{n+nb}{range}\PY{p}{(}\PY{l+m+mi}{0}\PY{p}{,} \PY{n+nb}{len}\PY{p}{(}\PY{n}{valores\PYZus{}x}\PY{p}{)}\PY{p}{)}
\PY{n}{lyapunov\PYZus{}exponents} \PY{o}{=} \PY{n}{lyapunov\PYZus{}spectrum\PYZus{}n}\PY{p}{(}\PY{n}{valores\PYZus{}x}\PY{p}{,} \PY{n}{valores\PYZus{}r}\PY{p}{,} \PY{n}{window\PYZus{}size}\PY{p}{)}

\PY{n}{fig}\PY{p}{,} \PY{n}{ax1} \PY{o}{=} \PY{n}{plt}\PY{o}{.}\PY{n}{subplots}\PY{p}{(}\PY{n}{figsize}\PY{o}{=}\PY{p}{(}\PY{l+m+mi}{12}\PY{p}{,} \PY{l+m+mi}{6}\PY{p}{)}\PY{p}{)}
\PY{n}{color} \PY{o}{=} \PY{l+s+s1}{\PYZsq{}}\PY{l+s+s1}{tab:blue}\PY{l+s+s1}{\PYZsq{}}
\PY{n}{ax1}\PY{o}{.}\PY{n}{set\PYZus{}xlabel}\PY{p}{(}\PY{l+s+s1}{\PYZsq{}}\PY{l+s+s1}{Tiempo}\PY{l+s+s1}{\PYZsq{}}\PY{p}{)}
\PY{n}{ax1}\PY{o}{.}\PY{n}{set\PYZus{}ylabel}\PY{p}{(}\PY{l+s+s1}{\PYZsq{}}\PY{l+s+s1}{Serie Temporal}\PY{l+s+s1}{\PYZsq{}}\PY{p}{,} \PY{n}{color}\PY{o}{=}\PY{n}{color}\PY{p}{)}
\PY{n}{ax1}\PY{o}{.}\PY{n}{scatter}\PY{p}{(}\PY{n}{pasos}\PY{p}{,} \PY{n}{valores\PYZus{}x}\PY{p}{,} \PY{n}{color}\PY{o}{=}\PY{n}{color}\PY{p}{,} \PY{n}{label}\PY{o}{=}\PY{l+s+s1}{\PYZsq{}}\PY{l+s+s1}{Serie Temporal}\PY{l+s+s1}{\PYZsq{}}\PY{p}{,} \PY{n}{marker}\PY{o}{=}\PY{l+s+s1}{\PYZsq{}}\PY{l+s+s1}{o}\PY{l+s+s1}{\PYZsq{}}\PY{p}{,} \PY{n}{s}\PY{o}{=}\PY{l+m+mf}{0.1}\PY{p}{)}
\PY{n}{ax1}\PY{o}{.}\PY{n}{tick\PYZus{}params}\PY{p}{(}\PY{n}{axis}\PY{o}{=}\PY{l+s+s1}{\PYZsq{}}\PY{l+s+s1}{y}\PY{l+s+s1}{\PYZsq{}}\PY{p}{,} \PY{n}{labelcolor}\PY{o}{=}\PY{n}{color}\PY{p}{)}

\PY{n}{ax2} \PY{o}{=} \PY{n}{ax1}\PY{o}{.}\PY{n}{twinx}\PY{p}{(}\PY{p}{)}
\PY{n}{color} \PY{o}{=} \PY{l+s+s1}{\PYZsq{}}\PY{l+s+s1}{tab:red}\PY{l+s+s1}{\PYZsq{}}
\PY{n}{ax2}\PY{o}{.}\PY{n}{set\PYZus{}ylabel}\PY{p}{(}\PY{l+s+s1}{\PYZsq{}}\PY{l+s+s1}{Exponente de Lyapunov}\PY{l+s+s1}{\PYZsq{}}\PY{p}{,} \PY{n}{color}\PY{o}{=}\PY{n}{color}\PY{p}{)}
\PY{n}{ax2}\PY{o}{.}\PY{n}{plot}\PY{p}{(}\PY{n}{np}\PY{o}{.}\PY{n}{arange}\PY{p}{(}\PY{n+nb}{len}\PY{p}{(}\PY{n}{lyapunov\PYZus{}exponents}\PY{p}{)}\PY{p}{)} \PY{o}{*} \PY{n}{window\PYZus{}size} \PY{o}{+} \PY{n}{window\PYZus{}size} \PY{o}{/}\PY{o}{/} \PY{l+m+mi}{2}\PY{p}{,} \PY{n}{lyapunov\PYZus{}exponents}\PY{p}{,} \PY{n}{color}\PY{o}{=}\PY{n}{color}\PY{p}{,} \PY{n}{label}\PY{o}{=}\PY{l+s+s1}{\PYZsq{}}\PY{l+s+s1}{Exponente de Lyapunov}\PY{l+s+s1}{\PYZsq{}}\PY{p}{)}
\PY{n}{ax2}\PY{o}{.}\PY{n}{tick\PYZus{}params}\PY{p}{(}\PY{n}{axis}\PY{o}{=}\PY{l+s+s1}{\PYZsq{}}\PY{l+s+s1}{y}\PY{l+s+s1}{\PYZsq{}}\PY{p}{,} \PY{n}{labelcolor}\PY{o}{=}\PY{n}{color}\PY{p}{)}

\PY{n}{fig}\PY{o}{.}\PY{n}{suptitle}\PY{p}{(}\PY{l+s+s1}{\PYZsq{}}\PY{l+s+s1}{Serie Temporal y Espectro de Exponentes de Lyapunov}\PY{l+s+s1}{\PYZsq{}}\PY{p}{)}
\PY{n}{fig}\PY{o}{.}\PY{n}{tight\PYZus{}layout}\PY{p}{(}\PY{p}{)}
\PY{n}{plt}\PY{o}{.}\PY{n}{show}\PY{p}{(}\PY{p}{)}
\end{Verbatim}
\end{tcolorbox}

    \begin{center}
    \adjustimage{max size={0.9\linewidth}{0.9\paperheight}}{analisisCaos2_files/analisisCaos2_20_0.png}
    \end{center}
    { \hspace*{\fill} \\}
    
    \hypertarget{interpretaciuxf3n}{%
\paragraph{Interpretación:}\label{interpretaciuxf3n}}

\begin{itemize}
\tightlist
\item
  \textbf{Serie Temporal (Azul):} Esta gráfica muestra la evolución
  temporal del sistema de bifurcación de Feigenbaum. Observamos la
  característica bifurcación del sistema, donde a medida que avanzamos
  en el tiempo, el sistema exhibe comportamientos cada vez más complejos
  y caóticos.
\item
  \textbf{Exponentes de Lyapunov (Rojo):} Los exponentes de Lyapunov
  superpuestos a la serie temporal indican la sensibilidad a las
  condiciones iniciales del sistema. Un exponente de Lyapunov positivo
  indica comportamiento caótico. En este gráfico, podemos observar que
  los exponentes de Lyapunov se vuelven positivos en varias regiones, lo
  cual confirma la presencia de caos en el sistema.
\end{itemize}

    \hypertarget{dimensiuxf3n-de-kaplan-yorke}{%
\subsection{Dimensión de
Kaplan-Yorke}\label{dimensiuxf3n-de-kaplan-yorke}}

    \begin{tcolorbox}[breakable, size=fbox, boxrule=1pt, pad at break*=1mm,colback=cellbackground, colframe=cellborder]
\prompt{In}{incolor}{521}{\boxspacing}
\begin{Verbatim}[commandchars=\\\{\}]
\PY{k}{def} \PY{n+nf}{calcular\PYZus{}dimension\PYZus{}kaplan\PYZus{}yorke}\PY{p}{(}\PY{n}{exponentes\PYZus{}lyapunov}\PY{p}{)}\PY{p}{:}
\PY{+w}{    }\PY{l+s+sd}{\PYZdq{}\PYZdq{}\PYZdq{}}
\PY{l+s+sd}{    Calcula la dimensión de Kaplan\PYZhy{}Yorke a partir de los exponentes de Lyapunov.}

\PY{l+s+sd}{    :param exponentes\PYZus{}lyapunov: Array con los exponentes de Lyapunov.}
\PY{l+s+sd}{    :return: Dimensión de Kaplan\PYZhy{}Yorke.}
\PY{l+s+sd}{    \PYZdq{}\PYZdq{}\PYZdq{}}
    \PY{n}{exponentes\PYZus{}lyapunov} \PY{o}{=} \PY{n}{np}\PY{o}{.}\PY{n}{sort}\PY{p}{(}\PY{n}{exponentes\PYZus{}lyapunov}\PY{p}{)}\PY{p}{[}\PY{p}{:}\PY{p}{:}\PY{o}{\PYZhy{}}\PY{l+m+mi}{1}\PY{p}{]}
    \PY{n}{suma} \PY{o}{=} \PY{l+m+mf}{0.0}
    \PY{n}{j} \PY{o}{=} \PY{l+m+mi}{0}
    \PY{k}{for} \PY{n}{j} \PY{o+ow}{in} \PY{n+nb}{range}\PY{p}{(}\PY{n+nb}{len}\PY{p}{(}\PY{n}{exponentes\PYZus{}lyapunov}\PY{p}{)}\PY{p}{)}\PY{p}{:}
        \PY{n}{suma} \PY{o}{+}\PY{o}{=} \PY{n}{exponentes\PYZus{}lyapunov}\PY{p}{[}\PY{n}{j}\PY{p}{]}
        \PY{k}{if} \PY{n}{suma} \PY{o}{\PYZlt{}} \PY{l+m+mi}{0}\PY{p}{:}
            \PY{n}{j} \PY{o}{\PYZhy{}}\PY{o}{=} \PY{l+m+mi}{1}
            \PY{k}{break}
        
    \PY{k}{if} \PY{n}{j} \PY{o}{\PYZlt{}} \PY{l+m+mi}{0}\PY{p}{:}
        \PY{k}{return} \PY{l+m+mi}{0}
    \PY{k}{elif} \PY{n}{j} \PY{o}{==} \PY{n+nb}{len}\PY{p}{(}\PY{n}{exponentes\PYZus{}lyapunov}\PY{p}{)} \PY{o}{\PYZhy{}} \PY{l+m+mi}{1}\PY{p}{:}
        \PY{n}{suma\PYZus{}positiva} \PY{o}{=} \PY{n}{np}\PY{o}{.}\PY{n}{sum}\PY{p}{(}\PY{n}{exponentes\PYZus{}lyapunov}\PY{p}{[}\PY{p}{:}\PY{n}{j}\PY{o}{+}\PY{l+m+mi}{1}\PY{p}{]}\PY{p}{)}
        \PY{n}{dimension\PYZus{}kaplan\PYZus{}yorke} \PY{o}{=} \PY{n}{j} \PY{o}{+} \PY{n}{suma\PYZus{}positiva} \PY{o}{/} \PY{n+nb}{abs}\PY{p}{(}\PY{n}{exponentes\PYZus{}lyapunov}\PY{p}{[}\PY{n}{j}\PY{p}{]}\PY{p}{)}
    \PY{k}{else}\PY{p}{:}
        \PY{n}{suma\PYZus{}positiva} \PY{o}{=} \PY{n}{np}\PY{o}{.}\PY{n}{sum}\PY{p}{(}\PY{n}{exponentes\PYZus{}lyapunov}\PY{p}{[}\PY{p}{:}\PY{n}{j}\PY{o}{+}\PY{l+m+mi}{1}\PY{p}{]}\PY{p}{)}
        \PY{n}{dimension\PYZus{}kaplan\PYZus{}yorke} \PY{o}{=} \PY{n}{j} \PY{o}{+} \PY{n}{suma\PYZus{}positiva} \PY{o}{/} \PY{n+nb}{abs}\PY{p}{(}\PY{n}{exponentes\PYZus{}lyapunov}\PY{p}{[}\PY{n}{j}\PY{o}{+}\PY{l+m+mi}{1}\PY{p}{]}\PY{p}{)}
    
    \PY{k}{return} \PY{n}{dimension\PYZus{}kaplan\PYZus{}yorke}
\end{Verbatim}
\end{tcolorbox}

    \begin{tcolorbox}[breakable, size=fbox, boxrule=1pt, pad at break*=1mm,colback=cellbackground, colframe=cellborder]
\prompt{In}{incolor}{522}{\boxspacing}
\begin{Verbatim}[commandchars=\\\{\}]
\PY{n}{kaplan\PYZus{}yorke\PYZus{}dimension} \PY{o}{=} \PY{n}{calcular\PYZus{}dimension\PYZus{}kaplan\PYZus{}yorke}\PY{p}{(}\PY{n}{lyapunov\PYZus{}exponents}\PY{p}{)}
\PY{n+nb}{print}\PY{p}{(}\PY{l+s+s2}{\PYZdq{}}\PY{l+s+s2}{Dimensión de Kaplan\PYZhy{}Yorke:}\PY{l+s+s2}{\PYZdq{}}\PY{p}{,} \PY{n}{kaplan\PYZus{}yorke\PYZus{}dimension}\PY{p}{)}
\end{Verbatim}
\end{tcolorbox}

    \begin{Verbatim}[commandchars=\\\{\}]
Dimensión de Kaplan-Yorke: 1859.7799991032848
    \end{Verbatim}

    \hypertarget{interpretaciuxf3n}{%
\paragraph{Interpretación:}\label{interpretaciuxf3n}}

\begin{itemize}
\tightlist
\item
  La dimensión de Kaplan-Yorke calculada es \textbf{1859.7799991032848}.
  Este valor es sorprendentemente alto y sugiere que el sistema tiene un
  comportamiento muy complejo y un atractor de alta dimensión.
\end{itemize}

    \hypertarget{dimensiuxf3n-grassberger-procaccia}{%
\subsubsection{Dimensión
Grassberger-Procaccia}\label{dimensiuxf3n-grassberger-procaccia}}

    \begin{tcolorbox}[breakable, size=fbox, boxrule=1pt, pad at break*=1mm,colback=cellbackground, colframe=cellborder]
\prompt{In}{incolor}{523}{\boxspacing}
\begin{Verbatim}[commandchars=\\\{\}]
\PY{k}{def} \PY{n+nf}{crear\PYZus{}embedds}\PY{p}{(}\PY{n}{datos}\PY{p}{,} \PY{n}{m}\PY{p}{,} \PY{n}{tau}\PY{p}{)}\PY{p}{:}
    \PY{n}{N} \PY{o}{=} \PY{n+nb}{len}\PY{p}{(}\PY{n}{datos}\PY{p}{)}
    \PY{n}{embedds} \PY{o}{=} \PY{n}{np}\PY{o}{.}\PY{n}{array}\PY{p}{(}\PY{p}{[}\PY{n}{datos}\PY{p}{[}\PY{n}{i}\PY{p}{:}\PY{n}{N}\PY{o}{\PYZhy{}}\PY{n}{tau}\PY{o}{*}\PY{p}{(}\PY{n}{m}\PY{o}{\PYZhy{}}\PY{l+m+mi}{1}\PY{p}{)}\PY{o}{+}\PY{n}{i}\PY{p}{:}\PY{n}{tau}\PY{p}{]} \PY{k}{for} \PY{n}{i} \PY{o+ow}{in} \PY{n+nb}{range}\PY{p}{(}\PY{n}{m}\PY{p}{)}\PY{p}{]}\PY{p}{)}
    \PY{k}{return} \PY{n}{embedds}\PY{o}{.}\PY{n}{T}

\PY{k}{def} \PY{n+nf}{calcular\PYZus{}correlacion\PYZus{}integral}\PY{p}{(}\PY{n}{kdtree}\PY{p}{,} \PY{n}{embedds}\PY{p}{,} \PY{n}{r}\PY{p}{)}\PY{p}{:}
    \PY{n}{N} \PY{o}{=} \PY{n+nb}{len}\PY{p}{(}\PY{n}{embedds}\PY{p}{)}
    \PY{n}{C\PYZus{}r} \PY{o}{=} \PY{n}{np}\PY{o}{.}\PY{n}{mean}\PY{p}{(}\PY{p}{[}\PY{n+nb}{len}\PY{p}{(}\PY{n}{kdtree}\PY{o}{.}\PY{n}{query\PYZus{}radius}\PY{p}{(}\PY{n}{point}\PY{o}{.}\PY{n}{reshape}\PY{p}{(}\PY{l+m+mi}{1}\PY{p}{,} \PY{o}{\PYZhy{}}\PY{l+m+mi}{1}\PY{p}{)}\PY{p}{,} \PY{n}{r}\PY{o}{=}\PY{n}{r}\PY{p}{)}\PY{p}{[}\PY{l+m+mi}{0}\PY{p}{]}\PY{p}{)} \PY{k}{for} \PY{n}{point} \PY{o+ow}{in} \PY{n}{embedds}\PY{p}{]}\PY{p}{)} \PY{o}{/} \PY{n}{N}
    \PY{k}{return} \PY{n}{C\PYZus{}r}

\PY{k}{def} \PY{n+nf}{calcular\PYZus{}dimension\PYZus{}gp}\PY{p}{(}\PY{n}{datos}\PY{p}{,} \PY{n}{m}\PY{p}{,} \PY{n}{tau}\PY{p}{,} \PY{n}{r\PYZus{}vals}\PY{p}{,} \PY{n}{n\PYZus{}jobs}\PY{o}{=}\PY{o}{\PYZhy{}}\PY{l+m+mi}{1}\PY{p}{)}\PY{p}{:}
    \PY{n}{embedds} \PY{o}{=} \PY{n}{crear\PYZus{}embedds}\PY{p}{(}\PY{n}{datos}\PY{p}{,} \PY{n}{m}\PY{p}{,} \PY{n}{tau}\PY{p}{)}
    \PY{n}{kdtree} \PY{o}{=} \PY{n}{KDTree}\PY{p}{(}\PY{n}{embedds}\PY{p}{)}
    \PY{n}{C\PYZus{}r\PYZus{}vals} \PY{o}{=} \PY{n}{Parallel}\PY{p}{(}\PY{n}{n\PYZus{}jobs}\PY{o}{=}\PY{n}{n\PYZus{}jobs}\PY{p}{)}\PY{p}{(}\PY{n}{delayed}\PY{p}{(}\PY{n}{calcular\PYZus{}correlacion\PYZus{}integral}\PY{p}{)}\PY{p}{(}\PY{n}{kdtree}\PY{p}{,} \PY{n}{embedds}\PY{p}{,} \PY{n}{r}\PY{p}{)} \PY{k}{for} \PY{n}{r} \PY{o+ow}{in} \PY{n}{r\PYZus{}vals}\PY{p}{)}
    \PY{n}{log\PYZus{}r} \PY{o}{=} \PY{n}{np}\PY{o}{.}\PY{n}{log}\PY{p}{(}\PY{n}{r\PYZus{}vals}\PY{p}{)}
    \PY{n}{log\PYZus{}C\PYZus{}r} \PY{o}{=} \PY{n}{np}\PY{o}{.}\PY{n}{log}\PY{p}{(}\PY{n}{C\PYZus{}r\PYZus{}vals}\PY{p}{)}
    \PY{n}{slope}\PY{p}{,} \PY{n}{intercept} \PY{o}{=} \PY{n}{np}\PY{o}{.}\PY{n}{polyfit}\PY{p}{(}\PY{n}{log\PYZus{}r}\PY{p}{,} \PY{n}{log\PYZus{}C\PYZus{}r}\PY{p}{,} \PY{l+m+mi}{1}\PY{p}{)}
    \PY{k}{return} \PY{n}{slope}
\end{Verbatim}
\end{tcolorbox}

    \begin{tcolorbox}[breakable, size=fbox, boxrule=1pt, pad at break*=1mm,colback=cellbackground, colframe=cellborder]
\prompt{In}{incolor}{524}{\boxspacing}
\begin{Verbatim}[commandchars=\\\{\}]
\PY{n}{m} \PY{o}{=} \PY{l+m+mi}{10}
\PY{n}{tau} \PY{o}{=} \PY{l+m+mi}{1}
\PY{n}{r\PYZus{}vals} \PY{o}{=} \PY{n}{np}\PY{o}{.}\PY{n}{logspace}\PY{p}{(}\PY{o}{\PYZhy{}}\PY{l+m+mi}{3}\PY{p}{,} \PY{l+m+mi}{0}\PY{p}{,} \PY{l+m+mi}{50}\PY{p}{)}
\PY{n}{dimension\PYZus{}gp} \PY{o}{=} \PY{n}{calcular\PYZus{}dimension\PYZus{}gp}\PY{p}{(}\PY{n}{valores\PYZus{}x}\PY{p}{,} \PY{n}{m}\PY{p}{,} \PY{n}{tau}\PY{p}{,} \PY{n}{r\PYZus{}vals}\PY{p}{)}
\PY{n+nb}{print}\PY{p}{(}\PY{l+s+s2}{\PYZdq{}}\PY{l+s+s2}{La dimensión de Grassberger\PYZhy{}Procaccia es:}\PY{l+s+s2}{\PYZdq{}}\PY{p}{,} \PY{n}{dimension\PYZus{}gp}\PY{p}{)}
\end{Verbatim}
\end{tcolorbox}

    \begin{Verbatim}[commandchars=\\\{\}]
La dimensión de Grassberger-Procaccia es: 0.9925943073228688
    \end{Verbatim}

    \hypertarget{interpretaciuxf3n}{%
\paragraph{Interpretación:}\label{interpretaciuxf3n}}

\begin{itemize}
\tightlist
\item
  La dimensión de Grassberger-Procaccia calculada es
  \textbf{0.9925943073228688}. Este valor sugiere que el atractor del
  sistema se comporta casi como una curva unidimensional. En sistemas
  dinámicos, valores cercanos a 1 pueden indicar que el sistema está en
  una dimensión baja, aunque aún presenta características fractales.
\end{itemize}

    \hypertarget{bifurcaciuxf3n-exponencial-de-feigenbaum}{%
\section{Bifurcación Exponencial de
Feigenbaum}\label{bifurcaciuxf3n-exponencial-de-feigenbaum}}

    \begin{tcolorbox}[breakable, size=fbox, boxrule=1pt, pad at break*=1mm,colback=cellbackground, colframe=cellborder]
\prompt{In}{incolor}{525}{\boxspacing}
\begin{Verbatim}[commandchars=\\\{\}]
\PY{n}{valores\PYZus{}x} \PY{o}{=} \PY{n}{leer\PYZus{}col\PYZus{}csv}\PY{p}{(}\PY{l+s+s2}{\PYZdq{}}\PY{l+s+s2}{datosFeigenbaumExponencial.csv}\PY{l+s+s2}{\PYZdq{}}\PY{p}{,} \PY{l+s+s2}{\PYZdq{}}\PY{l+s+s2}{Valores x}\PY{l+s+s2}{\PYZdq{}}\PY{p}{)}
\PY{n}{valores\PYZus{}k} \PY{o}{=} \PY{n+nb}{range}\PY{p}{(}\PY{l+m+mi}{1}\PY{p}{,} \PY{n+nb}{len}\PY{p}{(}\PY{n}{valores\PYZus{}x}\PY{p}{)}\PY{o}{+}\PY{l+m+mi}{1}\PY{p}{)}
\PY{n}{graficar}\PY{p}{(}\PY{n}{valores\PYZus{}x}\PY{p}{,} \PY{n}{valores\PYZus{}k}\PY{p}{,} \PY{n}{width}\PY{o}{=}\PY{l+m+mi}{10}\PY{p}{,} \PY{n}{height}\PY{o}{=}\PY{l+m+mi}{7} \PY{p}{,}\PY{n}{titulo}\PY{o}{=}\PY{l+s+s2}{\PYZdq{}}\PY{l+s+s2}{Bifurcacion Exponencial de Feigenbaum}\PY{l+s+s2}{\PYZdq{}}\PY{p}{)}
\end{Verbatim}
\end{tcolorbox}

    \begin{center}
    \adjustimage{max size={0.9\linewidth}{0.9\paperheight}}{analisisCaos2_files/analisisCaos2_31_0.png}
    \end{center}
    { \hspace*{\fill} \\}
    
    \hypertarget{probabiluxedstico}{%
\subsection{Probabilístico}\label{probabiluxedstico}}

    \begin{tcolorbox}[breakable, size=fbox, boxrule=1pt, pad at break*=1mm,colback=cellbackground, colframe=cellborder]
\prompt{In}{incolor}{526}{\boxspacing}
\begin{Verbatim}[commandchars=\\\{\}]
\PY{n}{feigen\PYZus{}e} \PY{o}{=} \PY{n}{DistribucionProbabilidad}\PY{p}{(}\PY{n}{valores\PYZus{}x}\PY{p}{)}
\PY{n}{feigen\PYZus{}e}\PY{o}{.}\PY{n}{mostrar\PYZus{}metricas}\PY{p}{(}\PY{p}{)}
\PY{n}{plotear\PYZus{}hist}\PY{p}{(}\PY{n}{valores\PYZus{}x}\PY{p}{,} \PY{l+s+s2}{\PYZdq{}}\PY{l+s+s2}{Histograma de Feigenbaum Exponencial}\PY{l+s+s2}{\PYZdq{}}\PY{p}{,} \PY{l+s+s2}{\PYZdq{}}\PY{l+s+s2}{Valor}\PY{l+s+s2}{\PYZdq{}}\PY{p}{,} \PY{l+s+s2}{\PYZdq{}}\PY{l+s+s2}{Frecuencia}\PY{l+s+s2}{\PYZdq{}}\PY{p}{,} \PY{l+s+s2}{\PYZdq{}}\PY{l+s+s2}{sturges}\PY{l+s+s2}{\PYZdq{}}\PY{p}{)}
\end{Verbatim}
\end{tcolorbox}

    \begin{Verbatim}[commandchars=\\\{\}]
--- Métricas para Variable 1 ---
Media: 1.0000106107876798
Mediana: 0.6220634691695446
Moda: 0.0004820849853227784
Media Geométrica: 0.40555674899751426
Rango: 3.5279839013453684
Desviación Estándar: 0.938773102207703
Varianza: 0.8812949374286744
Asimetría: 0.6445668032337394
Curtosis: -0.8737655379697409
Entropia: 11.512925464970223
Coeficiente de Variación: 0.9387631411913302

    \end{Verbatim}

    \begin{center}
    \adjustimage{max size={0.9\linewidth}{0.9\paperheight}}{analisisCaos2_files/analisisCaos2_33_1.png}
    \end{center}
    { \hspace*{\fill} \\}
    
    \hypertarget{anuxe1lisis-de-muxe9tricas-estaduxedsticas-para-el-modelo-de-bifurcaciones-exponencial-de-feigenbaum}{%
\subsection{Análisis de Métricas Estadísticas para el Modelo de
Bifurcaciones Exponencial de
Feigenbaum}\label{anuxe1lisis-de-muxe9tricas-estaduxedsticas-para-el-modelo-de-bifurcaciones-exponencial-de-feigenbaum}}

\begin{itemize}
\item
  \textbf{Media (1.000007)}: La media es exactamente 1, lo cual es
  interesante y podría indicar un comportamiento estabilizador alrededor
  de este valor en el sistema. Esto sugiere que, a pesar de la
  naturaleza exponencial del modelo, los valores tienden a converger o
  fluctuar alrededor de este punto.
\item
  \textbf{Mediana (0.4721)}: La mediana considerablemente menor que la
  media implica una distribución asimétrica de los datos. Esto podría
  indicar la presencia de una cola larga hacia valores más altos, que no
  son comunes pero contribuyen significativamente al promedio.
\item
  \textbf{Desviación Estándar (1.0533)} y \textbf{Varianza (1.1094)}:
  Ambas métricas son relativamente altas, destacando una gran dispersión
  de los datos. Esta alta variabilidad es característica de los sistemas
  caóticos donde pequeños cambios en los parámetros iniciales pueden
  producir grandes variaciones en los resultados.
\item
  \textbf{Asimetría (0.9807)}: La asimetría positiva significa que hay
  una cola más pesada hacia valores más altos. Esto reafirma la
  presencia de valores extremos que pueden estar influenciando la media.
\item
  \textbf{Curtosis (0.0634)}: Una curtosis cercana a cero sugiere que la
  distribución no es ni muy picuda ni muy plana, lo que es inusual para
  sistemas dinámicos caóticos y merece una investigación más profunda.
\item
  \textbf{Entropía (11.9184)}: Similar al modelo logístico, la alta
  entropía refleja una considerable incertidumbre y diversidad en los
  valores, lo que es típico en comportamientos caóticos.
\item
  \textbf{Coeficiente de Variación (1.0533)}: Este valor indica que la
  desviación estándar es comparable a la media, lo que sugiere que hay
  una amplia variación en los datos en relación con su nivel promedio.
\end{itemize}

\hypertarget{conclusiuxf3n}{%
\subsubsection{Conclusión}\label{conclusiuxf3n}}

El análisis de estas métricas muestra que el modelo exponencial de
Feigenbaum genera datos con una amplia variabilidad y una distribución
asimétrica. La presencia de asimetría y una alta entropía son
indicativos de un sistema que exhibe un comportamiento dinámico complejo
y caótico. Estas características son fundamentales para comprender la
dinámica subyacente del modelo y para explorar cómo pequeñas variaciones
en las condiciones iniciales pueden influir dramáticamente en el
comportamiento del sistema.

    \hypertarget{cauxf3tico}{%
\subsection{Caótico}\label{cauxf3tico}}

    \hypertarget{exponentes-de-laypounov}{%
\subsubsection{Exponentes de Laypounov}\label{exponentes-de-laypounov}}

    \begin{tcolorbox}[breakable, size=fbox, boxrule=1pt, pad at break*=1mm,colback=cellbackground, colframe=cellborder]
\prompt{In}{incolor}{527}{\boxspacing}
\begin{Verbatim}[commandchars=\\\{\}]
\PY{k}{def} \PY{n+nf}{lyapunov\PYZus{}spectrum\PYZus{}e}\PY{p}{(}\PY{n}{series}\PY{p}{,} \PY{n}{rates}\PY{p}{,} \PY{n}{window\PYZus{}size}\PY{p}{)}\PY{p}{:}
    \PY{n}{N} \PY{o}{=} \PY{n+nb}{len}\PY{p}{(}\PY{n}{series}\PY{p}{)}
    \PY{n}{n\PYZus{}windows} \PY{o}{=} \PY{n}{N} \PY{o}{/}\PY{o}{/} \PY{n}{window\PYZus{}size}
    \PY{n}{lyapunov\PYZus{}exponents} \PY{o}{=} \PY{p}{[}\PY{p}{]}

    \PY{k}{for} \PY{n}{i} \PY{o+ow}{in} \PY{n+nb}{range}\PY{p}{(}\PY{n}{n\PYZus{}windows}\PY{p}{)}\PY{p}{:}
        \PY{n}{window} \PY{o}{=} \PY{n}{series}\PY{p}{[}\PY{n}{i} \PY{o}{*} \PY{n}{window\PYZus{}size}\PY{p}{:}\PY{p}{(}\PY{n}{i} \PY{o}{+} \PY{l+m+mi}{1}\PY{p}{)} \PY{o}{*} \PY{n}{window\PYZus{}size}\PY{p}{]}
        \PY{n}{rate\PYZus{}window} \PY{o}{=} \PY{n}{rates}\PY{p}{[}\PY{n}{i} \PY{o}{*} \PY{n}{window\PYZus{}size}\PY{p}{:}\PY{p}{(}\PY{n}{i} \PY{o}{+} \PY{l+m+mi}{1}\PY{p}{)} \PY{o}{*} \PY{n}{window\PYZus{}size}\PY{p}{]}
        \PY{n}{sum\PYZus{}log\PYZus{}der} \PY{o}{=} \PY{l+m+mf}{0.0}
        \PY{k}{for} \PY{n}{j} \PY{o+ow}{in} \PY{n+nb}{range}\PY{p}{(}\PY{l+m+mi}{1}\PY{p}{,} \PY{n}{window\PYZus{}size}\PY{p}{)}\PY{p}{:}
            \PY{n}{x} \PY{o}{=} \PY{n}{window}\PY{p}{[}\PY{n}{j}\PY{p}{]}
            \PY{n}{r} \PY{o}{=} \PY{n}{rate\PYZus{}window}\PY{p}{[}\PY{n}{j}\PY{p}{]}
            \PY{n}{derivative} \PY{o}{=} \PY{n}{np}\PY{o}{.}\PY{n}{exp}\PY{p}{(}\PY{n}{r} \PY{o}{*} \PY{p}{(}\PY{l+m+mi}{1} \PY{o}{\PYZhy{}} \PY{n}{x}\PY{p}{)}\PY{p}{)} \PY{o}{*} \PY{p}{(}\PY{l+m+mi}{1} \PY{o}{\PYZhy{}} \PY{n}{r} \PY{o}{*} \PY{n}{x}\PY{p}{)}
            \PY{n}{sum\PYZus{}log\PYZus{}der} \PY{o}{+}\PY{o}{=} \PY{n}{np}\PY{o}{.}\PY{n}{log}\PY{p}{(}\PY{n+nb}{abs}\PY{p}{(}\PY{n}{derivative}\PY{p}{)}\PY{p}{)}
        
        \PY{n}{le} \PY{o}{=} \PY{n}{sum\PYZus{}log\PYZus{}der} \PY{o}{/} \PY{n}{window\PYZus{}size}
        \PY{n}{lyapunov\PYZus{}exponents}\PY{o}{.}\PY{n}{append}\PY{p}{(}\PY{n}{le}\PY{p}{)}
    
    \PY{k}{return} \PY{n}{np}\PY{o}{.}\PY{n}{array}\PY{p}{(}\PY{n}{lyapunov\PYZus{}exponents}\PY{p}{)}
\end{Verbatim}
\end{tcolorbox}

    \begin{tcolorbox}[breakable, size=fbox, boxrule=1pt, pad at break*=1mm,colback=cellbackground, colframe=cellborder]
\prompt{In}{incolor}{528}{\boxspacing}
\begin{Verbatim}[commandchars=\\\{\}]
\PY{n}{window\PYZus{}size} \PY{o}{=} \PY{l+m+mi}{100}
\PY{n}{valores\PYZus{}x}\PY{p}{,} \PY{n}{valores\PYZus{}r} \PY{o}{=} \PY{n}{leer\PYZus{}csv\PYZus{}a\PYZus{}numpy}\PY{p}{(}\PY{l+s+s2}{\PYZdq{}}\PY{l+s+s2}{datosFeigenbaumExponencial.csv}\PY{l+s+s2}{\PYZdq{}}\PY{p}{)}
\PY{n}{pasos} \PY{o}{=} \PY{n+nb}{range}\PY{p}{(}\PY{l+m+mi}{0}\PY{p}{,} \PY{n+nb}{len}\PY{p}{(}\PY{n}{valores\PYZus{}x}\PY{p}{)}\PY{p}{)}
\PY{n}{lyapunov\PYZus{}exponents} \PY{o}{=} \PY{n}{lyapunov\PYZus{}spectrum\PYZus{}e}\PY{p}{(}\PY{n}{valores\PYZus{}x}\PY{p}{,} \PY{n}{valores\PYZus{}r}\PY{p}{,} \PY{n}{window\PYZus{}size}\PY{p}{)}
\PY{n}{fig}\PY{p}{,} \PY{n}{ax1} \PY{o}{=} \PY{n}{plt}\PY{o}{.}\PY{n}{subplots}\PY{p}{(}\PY{n}{figsize}\PY{o}{=}\PY{p}{(}\PY{l+m+mi}{12}\PY{p}{,} \PY{l+m+mi}{6}\PY{p}{)}\PY{p}{)}

\PY{n}{color} \PY{o}{=} \PY{l+s+s1}{\PYZsq{}}\PY{l+s+s1}{tab:blue}\PY{l+s+s1}{\PYZsq{}}
\PY{n}{ax1}\PY{o}{.}\PY{n}{set\PYZus{}xlabel}\PY{p}{(}\PY{l+s+s1}{\PYZsq{}}\PY{l+s+s1}{Tiempo}\PY{l+s+s1}{\PYZsq{}}\PY{p}{)}
\PY{n}{ax1}\PY{o}{.}\PY{n}{set\PYZus{}ylabel}\PY{p}{(}\PY{l+s+s1}{\PYZsq{}}\PY{l+s+s1}{Serie Temporal}\PY{l+s+s1}{\PYZsq{}}\PY{p}{,} \PY{n}{color}\PY{o}{=}\PY{n}{color}\PY{p}{)}
\PY{n}{ax1}\PY{o}{.}\PY{n}{scatter}\PY{p}{(}\PY{n}{pasos}\PY{p}{,} \PY{n}{valores\PYZus{}x}\PY{p}{,} \PY{n}{color}\PY{o}{=}\PY{n}{color}\PY{p}{,} \PY{n}{label}\PY{o}{=}\PY{l+s+s1}{\PYZsq{}}\PY{l+s+s1}{Serie Temporal}\PY{l+s+s1}{\PYZsq{}}\PY{p}{,} \PY{n}{marker}\PY{o}{=}\PY{l+s+s1}{\PYZsq{}}\PY{l+s+s1}{o}\PY{l+s+s1}{\PYZsq{}}\PY{p}{,} \PY{n}{s}\PY{o}{=}\PY{l+m+mf}{0.1}\PY{p}{)}
\PY{n}{ax1}\PY{o}{.}\PY{n}{tick\PYZus{}params}\PY{p}{(}\PY{n}{axis}\PY{o}{=}\PY{l+s+s1}{\PYZsq{}}\PY{l+s+s1}{y}\PY{l+s+s1}{\PYZsq{}}\PY{p}{,} \PY{n}{labelcolor}\PY{o}{=}\PY{n}{color}\PY{p}{)}
\PY{n}{ax2} \PY{o}{=} \PY{n}{ax1}\PY{o}{.}\PY{n}{twinx}\PY{p}{(}\PY{p}{)}
\PY{n}{color} \PY{o}{=} \PY{l+s+s1}{\PYZsq{}}\PY{l+s+s1}{tab:red}\PY{l+s+s1}{\PYZsq{}}
\PY{n}{ax2}\PY{o}{.}\PY{n}{set\PYZus{}ylabel}\PY{p}{(}\PY{l+s+s1}{\PYZsq{}}\PY{l+s+s1}{Exponente de Lyapunov}\PY{l+s+s1}{\PYZsq{}}\PY{p}{,} \PY{n}{color}\PY{o}{=}\PY{n}{color}\PY{p}{)}
\PY{n}{ax2}\PY{o}{.}\PY{n}{plot}\PY{p}{(}\PY{n}{np}\PY{o}{.}\PY{n}{arange}\PY{p}{(}\PY{n+nb}{len}\PY{p}{(}\PY{n}{lyapunov\PYZus{}exponents}\PY{p}{)}\PY{p}{)} \PY{o}{*} \PY{n}{window\PYZus{}size} \PY{o}{+} \PY{n}{window\PYZus{}size} \PY{o}{/}\PY{o}{/} \PY{l+m+mi}{2}\PY{p}{,} \PY{n}{lyapunov\PYZus{}exponents}\PY{p}{,} \PY{n}{color}\PY{o}{=}\PY{n}{color}\PY{p}{,} \PY{n}{label}\PY{o}{=}\PY{l+s+s1}{\PYZsq{}}\PY{l+s+s1}{Exponente de Lyapunov}\PY{l+s+s1}{\PYZsq{}}\PY{p}{)}
\PY{n}{ax2}\PY{o}{.}\PY{n}{tick\PYZus{}params}\PY{p}{(}\PY{n}{axis}\PY{o}{=}\PY{l+s+s1}{\PYZsq{}}\PY{l+s+s1}{y}\PY{l+s+s1}{\PYZsq{}}\PY{p}{,} \PY{n}{labelcolor}\PY{o}{=}\PY{n}{color}\PY{p}{)}
\PY{n}{fig}\PY{o}{.}\PY{n}{suptitle}\PY{p}{(}\PY{l+s+s1}{\PYZsq{}}\PY{l+s+s1}{Serie Temporal y Espectro de Exponentes de Lyapunov}\PY{l+s+s1}{\PYZsq{}}\PY{p}{)}
\PY{n}{fig}\PY{o}{.}\PY{n}{tight\PYZus{}layout}\PY{p}{(}\PY{p}{)}
\PY{n}{plt}\PY{o}{.}\PY{n}{show}\PY{p}{(}\PY{p}{)}
\end{Verbatim}
\end{tcolorbox}

    \begin{center}
    \adjustimage{max size={0.9\linewidth}{0.9\paperheight}}{analisisCaos2_files/analisisCaos2_38_0.png}
    \end{center}
    { \hspace*{\fill} \\}
    
    \hypertarget{interpretaciuxf3n}{%
\paragraph{Interpretación:}\label{interpretaciuxf3n}}

\begin{itemize}
\tightlist
\item
  \textbf{Serie Temporal (Azul):} Esta gráfica muestra la evolución
  temporal del sistema exponencial de bifurcación. Se puede observar una
  estructura compleja y densa, indicando la presencia de múltiples
  bifurcaciones y comportamiento caótico en el sistema.
\item
  \textbf{Exponentes de Lyapunov (Rojo):} Los exponentes de Lyapunov
  superpuestos a la serie temporal indican la sensibilidad a las
  condiciones iniciales del sistema. Un exponente de Lyapunov positivo
  indica comportamiento caótico. En este gráfico, se observa que los
  exponentes de Lyapunov se mantienen mayormente positivos en varias
  regiones, lo que confirma la presencia de caos en el sistema.
\end{itemize}

    \hypertarget{dimensiuxf3n-de-kaplan-yorke}{%
\subsubsection{Dimensión de
Kaplan-Yorke}\label{dimensiuxf3n-de-kaplan-yorke}}

    \begin{tcolorbox}[breakable, size=fbox, boxrule=1pt, pad at break*=1mm,colback=cellbackground, colframe=cellborder]
\prompt{In}{incolor}{529}{\boxspacing}
\begin{Verbatim}[commandchars=\\\{\}]
\PY{n}{kaplan\PYZus{}yorke\PYZus{}dimension} \PY{o}{=} \PY{n}{calcular\PYZus{}dimension\PYZus{}kaplan\PYZus{}yorke}\PY{p}{(}\PY{n}{lyapunov\PYZus{}exponents}\PY{p}{)}
\PY{n+nb}{print}\PY{p}{(}\PY{l+s+s2}{\PYZdq{}}\PY{l+s+s2}{Dimensión de Kaplan\PYZhy{}Yorke:}\PY{l+s+s2}{\PYZdq{}}\PY{p}{,} \PY{n}{kaplan\PYZus{}yorke\PYZus{}dimension}\PY{p}{)}
\end{Verbatim}
\end{tcolorbox}

    \begin{Verbatim}[commandchars=\\\{\}]
Dimensión de Kaplan-Yorke: 1118.1010399811667
    \end{Verbatim}

    \hypertarget{interpretaciuxf3n}{%
\paragraph{Interpretación:}\label{interpretaciuxf3n}}

\begin{itemize}
\tightlist
\item
  La dimensión de Kaplan-Yorke calculada es \textbf{1118.1010399811667}.
  Este valor es alto y sugiere que el sistema tiene un comportamiento
  muy complejo y un atractor de alta dimensión.
\end{itemize}

    \hypertarget{dimensiuxf3n-grassberger-procaccia}{%
\subsubsection{Dimensión
Grassberger-Procaccia}\label{dimensiuxf3n-grassberger-procaccia}}

    \begin{tcolorbox}[breakable, size=fbox, boxrule=1pt, pad at break*=1mm,colback=cellbackground, colframe=cellborder]
\prompt{In}{incolor}{530}{\boxspacing}
\begin{Verbatim}[commandchars=\\\{\}]
\PY{n}{m} \PY{o}{=} \PY{l+m+mi}{10}
\PY{n}{tau} \PY{o}{=} \PY{l+m+mi}{1}
\PY{n}{r\PYZus{}vals} \PY{o}{=} \PY{n}{np}\PY{o}{.}\PY{n}{logspace}\PY{p}{(}\PY{o}{\PYZhy{}}\PY{l+m+mi}{3}\PY{p}{,} \PY{l+m+mi}{0}\PY{p}{,} \PY{l+m+mi}{50}\PY{p}{)}
\PY{n}{dimension\PYZus{}gp} \PY{o}{=} \PY{n}{calcular\PYZus{}dimension\PYZus{}gp}\PY{p}{(}\PY{n}{valores\PYZus{}x}\PY{p}{,} \PY{n}{m}\PY{p}{,} \PY{n}{tau}\PY{p}{,} \PY{n}{r\PYZus{}vals}\PY{p}{)}
\PY{n+nb}{print}\PY{p}{(}\PY{l+s+s2}{\PYZdq{}}\PY{l+s+s2}{La dimensión de Grassberger\PYZhy{}Procaccia es:}\PY{l+s+s2}{\PYZdq{}}\PY{p}{,} \PY{n}{dimension\PYZus{}gp}\PY{p}{)}
\end{Verbatim}
\end{tcolorbox}

    \begin{Verbatim}[commandchars=\\\{\}]
La dimensión de Grassberger-Procaccia es: 0.9846650435688487
    \end{Verbatim}

    \hypertarget{interpretaciuxf3n}{%
\paragraph{Interpretación:}\label{interpretaciuxf3n}}

\begin{itemize}
\tightlist
\item
  La dimensión de Grassberger-Procaccia calculada es
  \textbf{0.9846650435688487}. Este valor sugiere que el atractor del
  sistema se comporta casi como una curva unidimensional. En sistemas
  dinámicos, valores cercanos a 1 pueden indicar que el sistema está en
  una dimensión baja, aunque aún presenta características fractales.
\end{itemize}

    \hypertarget{bifurcaciuxf3n-cuxfabica-de-feigenbaum}{%
\section{Bifurcación Cúbica de
Feigenbaum}\label{bifurcaciuxf3n-cuxfabica-de-feigenbaum}}

    \begin{tcolorbox}[breakable, size=fbox, boxrule=1pt, pad at break*=1mm,colback=cellbackground, colframe=cellborder]
\prompt{In}{incolor}{531}{\boxspacing}
\begin{Verbatim}[commandchars=\\\{\}]
\PY{n}{valores\PYZus{}x} \PY{o}{=} \PY{n}{leer\PYZus{}col\PYZus{}csv}\PY{p}{(}\PY{l+s+s2}{\PYZdq{}}\PY{l+s+s2}{datosFeigenbaumCubica.csv}\PY{l+s+s2}{\PYZdq{}}\PY{p}{,} \PY{l+s+s2}{\PYZdq{}}\PY{l+s+s2}{Valores x}\PY{l+s+s2}{\PYZdq{}}\PY{p}{)}
\PY{n}{valores\PYZus{}k} \PY{o}{=} \PY{n+nb}{range}\PY{p}{(}\PY{l+m+mi}{1}\PY{p}{,} \PY{n+nb}{len}\PY{p}{(}\PY{n}{valores\PYZus{}x}\PY{p}{)}\PY{o}{+}\PY{l+m+mi}{1}\PY{p}{)}
\PY{n}{graficar}\PY{p}{(}\PY{n}{valores\PYZus{}x}\PY{p}{,} \PY{n}{valores\PYZus{}k}\PY{p}{,} \PY{n}{width}\PY{o}{=}\PY{l+m+mi}{10}\PY{p}{,} \PY{n}{height}\PY{o}{=}\PY{l+m+mi}{7} \PY{p}{,}\PY{n}{titulo}\PY{o}{=}\PY{l+s+s2}{\PYZdq{}}\PY{l+s+s2}{Bifurcacion Cubica de Feigenbaum}\PY{l+s+s2}{\PYZdq{}}\PY{p}{)}
\end{Verbatim}
\end{tcolorbox}

    \begin{center}
    \adjustimage{max size={0.9\linewidth}{0.9\paperheight}}{analisisCaos2_files/analisisCaos2_47_0.png}
    \end{center}
    { \hspace*{\fill} \\}
    
    \hypertarget{probabiluxedstico}{%
\subsection{Probabilístico}\label{probabiluxedstico}}

    \begin{tcolorbox}[breakable, size=fbox, boxrule=1pt, pad at break*=1mm,colback=cellbackground, colframe=cellborder]
\prompt{In}{incolor}{532}{\boxspacing}
\begin{Verbatim}[commandchars=\\\{\}]
\PY{n}{feigen\PYZus{}c} \PY{o}{=} \PY{n}{DistribucionProbabilidad}\PY{p}{(}\PY{n}{valores\PYZus{}x}\PY{p}{)}
\PY{n}{feigen\PYZus{}c}\PY{o}{.}\PY{n}{mostrar\PYZus{}metricas}\PY{p}{(}\PY{p}{)}
\PY{n}{plotear\PYZus{}hist}\PY{p}{(}\PY{n}{valores\PYZus{}x}\PY{p}{,} \PY{l+s+s2}{\PYZdq{}}\PY{l+s+s2}{Histograma de Feigenbaum Cubica}\PY{l+s+s2}{\PYZdq{}}\PY{p}{,} \PY{l+s+s2}{\PYZdq{}}\PY{l+s+s2}{Valor}\PY{l+s+s2}{\PYZdq{}}\PY{p}{,} \PY{l+s+s2}{\PYZdq{}}\PY{l+s+s2}{Frecuencia}\PY{l+s+s2}{\PYZdq{}}\PY{p}{,} \PY{l+s+s2}{\PYZdq{}}\PY{l+s+s2}{sturges}\PY{l+s+s2}{\PYZdq{}}\PY{p}{)}
\end{Verbatim}
\end{tcolorbox}

    \begin{Verbatim}[commandchars=\\\{\}]
--- Métricas para Variable 1 ---
Media: 0.035128893534851074
Mediana: -0.006948518172824103
Moda: -0.992241615612928
Media Geométrica: nan
Rango: 1.9838105846338316
Desviación Estándar: 0.5932030500549282
Varianza: 0.3518898585944697
Asimetría: -0.009584765711514487
Curtosis: -1.5738697743200434
Entropia: 11.512925464970223
Coeficiente de Variación: 16.886471231051345

    \end{Verbatim}

    \begin{Verbatim}[commandchars=\\\{\}]
/home/rodrigo/.local/lib/python3.10/site-packages/scipy/stats/\_stats\_py.py:197:
RuntimeWarning:

invalid value encountered in log

    \end{Verbatim}

    \begin{center}
    \adjustimage{max size={0.9\linewidth}{0.9\paperheight}}{analisisCaos2_files/analisisCaos2_49_2.png}
    \end{center}
    { \hspace*{\fill} \\}
    
    \hypertarget{anuxe1lisis-de-muxe9tricas-estaduxedsticas-para-el-modelo-de-bifurcaciones-cuxfabico-de-feigenbaum}{%
\subsection{Análisis de Métricas Estadísticas para el Modelo de
Bifurcaciones Cúbico de
Feigenbaum}\label{anuxe1lisis-de-muxe9tricas-estaduxedsticas-para-el-modelo-de-bifurcaciones-cuxfabico-de-feigenbaum}}

\begin{itemize}
\item
  \textbf{Media (0.0244)}: Esta media cercana a cero sugiere que, en
  promedio, los valores de la serie tienden a ser bajos, aunque esto no
  descarta la presencia de valores extremos en ambos sentidos.
\item
  \textbf{Desviación Estándar (0.5557)} y \textbf{Varianza (0.3089)}:
  Estas métricas indican una dispersión significativa de los datos
  alrededor de la media, lo que es característico de los sistemas
  caóticos donde la variabilidad es alta.
\item
  \textbf{Rango (1.9245)}: Un rango amplio como este demuestra que los
  valores se extienden casi por todo el intervalo teórico posible en el
  modelo (-1 a 1), lo cual es indicativo de una dinámica extrema que
  puede estar explorando múltiples estados.
\item
  \textbf{Curtosis (-1.6627)}: Una curtosis negativa indica una
  distribución más aplanada que la normal, lo cual podría sugerir una
  mayor igualdad en la frecuencia de aparición de los valores a lo largo
  del rango, en lugar de una acumulación alrededor de un valor medio.
\item
  \textbf{Entropía (11.9184)}: Una entropía muy alta es típica de los
  sistemas dinámicos caóticos, donde la diversidad de estados es máxima,
  indicando que el sistema puede estar en muchos estados diferentes con
  igual probabilidad.
\item
  \textbf{Coeficiente de Variación (22.7498)}: Este valor extremadamente
  alto muestra que la desviación estándar es mucho mayor que la media,
  reforzando el punto de alta variabilidad y la naturaleza impredecible
  del sistema bajo estudio.
\end{itemize}

\hypertarget{conclusiuxf3n}{%
\subsubsection{Conclusión}\label{conclusiuxf3n}}

Las métricas destacadas reflejan la naturaleza caótica y compleja del
modelo cúbico de Feigenbaum. La alta variabilidad, el amplio rango y la
alta entropía son indicativos de un sistema que muestra un
comportamiento dinámico muy rico y posiblemente caótico, explorando un
amplio espectro de estados posibles. Esto es característico de los
modelos que incluyen términos no lineales elevados, como es el caso del
modelo cúbico.

    \hypertarget{cauxf3tico}{%
\subsection{Caótico}\label{cauxf3tico}}

    \hypertarget{exponentes-de-laypounov}{%
\subsubsection{Exponentes de Laypounov}\label{exponentes-de-laypounov}}

    \begin{tcolorbox}[breakable, size=fbox, boxrule=1pt, pad at break*=1mm,colback=cellbackground, colframe=cellborder]
\prompt{In}{incolor}{533}{\boxspacing}
\begin{Verbatim}[commandchars=\\\{\}]
\PY{k}{def} \PY{n+nf}{lyapunov\PYZus{}spectrum\PYZus{}c}\PY{p}{(}\PY{n}{series}\PY{p}{,} \PY{n}{rates}\PY{p}{,} \PY{n}{window\PYZus{}size}\PY{p}{)}\PY{p}{:}
    \PY{n}{N} \PY{o}{=} \PY{n+nb}{len}\PY{p}{(}\PY{n}{series}\PY{p}{)}
    \PY{n}{n\PYZus{}windows} \PY{o}{=} \PY{n}{N} \PY{o}{/}\PY{o}{/} \PY{n}{window\PYZus{}size}
    \PY{n}{lyapunov\PYZus{}exponents} \PY{o}{=} \PY{p}{[}\PY{p}{]}

    \PY{k}{for} \PY{n}{i} \PY{o+ow}{in} \PY{n+nb}{range}\PY{p}{(}\PY{n}{n\PYZus{}windows}\PY{p}{)}\PY{p}{:}
        \PY{n}{window} \PY{o}{=} \PY{n}{series}\PY{p}{[}\PY{n}{i} \PY{o}{*} \PY{n}{window\PYZus{}size}\PY{p}{:}\PY{p}{(}\PY{n}{i} \PY{o}{+} \PY{l+m+mi}{1}\PY{p}{)} \PY{o}{*} \PY{n}{window\PYZus{}size}\PY{p}{]}
        \PY{n}{rate\PYZus{}window} \PY{o}{=} \PY{n}{rates}\PY{p}{[}\PY{n}{i} \PY{o}{*} \PY{n}{window\PYZus{}size}\PY{p}{:}\PY{p}{(}\PY{n}{i} \PY{o}{+} \PY{l+m+mi}{1}\PY{p}{)} \PY{o}{*} \PY{n}{window\PYZus{}size}\PY{p}{]}
        \PY{n}{sum\PYZus{}log\PYZus{}der} \PY{o}{=} \PY{l+m+mf}{0.0}
        \PY{k}{for} \PY{n}{j} \PY{o+ow}{in} \PY{n+nb}{range}\PY{p}{(}\PY{l+m+mi}{1}\PY{p}{,} \PY{n}{window\PYZus{}size}\PY{p}{)}\PY{p}{:}
            \PY{n}{x} \PY{o}{=} \PY{n}{window}\PY{p}{[}\PY{n}{j}\PY{p}{]}
            \PY{n}{r} \PY{o}{=} \PY{n}{rate\PYZus{}window}\PY{p}{[}\PY{n}{j}\PY{p}{]}
            \PY{n}{derivative} \PY{o}{=} \PY{l+m+mi}{1} \PY{o}{+} \PY{n}{r} \PY{o}{*} \PY{p}{(}\PY{l+m+mi}{3} \PY{o}{*} \PY{n}{x}\PY{o}{*}\PY{o}{*}\PY{l+m+mi}{2} \PY{o}{\PYZhy{}} \PY{l+m+mi}{1}\PY{p}{)}
            \PY{n}{sum\PYZus{}log\PYZus{}der} \PY{o}{+}\PY{o}{=} \PY{n}{np}\PY{o}{.}\PY{n}{log}\PY{p}{(}\PY{n+nb}{abs}\PY{p}{(}\PY{n}{derivative}\PY{p}{)}\PY{p}{)}
        
        \PY{n}{le} \PY{o}{=} \PY{n}{sum\PYZus{}log\PYZus{}der} \PY{o}{/} \PY{n}{window\PYZus{}size}
        \PY{n}{lyapunov\PYZus{}exponents}\PY{o}{.}\PY{n}{append}\PY{p}{(}\PY{n}{le}\PY{p}{)}
    
    \PY{k}{return} \PY{n}{np}\PY{o}{.}\PY{n}{array}\PY{p}{(}\PY{n}{lyapunov\PYZus{}exponents}\PY{p}{)}
\end{Verbatim}
\end{tcolorbox}

    \begin{tcolorbox}[breakable, size=fbox, boxrule=1pt, pad at break*=1mm,colback=cellbackground, colframe=cellborder]
\prompt{In}{incolor}{534}{\boxspacing}
\begin{Verbatim}[commandchars=\\\{\}]
\PY{n}{valores\PYZus{}x}\PY{p}{,} \PY{n}{valores\PYZus{}r} \PY{o}{=} \PY{n}{leer\PYZus{}csv\PYZus{}a\PYZus{}numpy}\PY{p}{(}\PY{l+s+s2}{\PYZdq{}}\PY{l+s+s2}{datosFeigenbaumCubica.csv}\PY{l+s+s2}{\PYZdq{}}\PY{p}{)}
\PY{n}{pasos} \PY{o}{=} \PY{n+nb}{range}\PY{p}{(}\PY{l+m+mi}{0}\PY{p}{,} \PY{n+nb}{len}\PY{p}{(}\PY{n}{valores\PYZus{}x}\PY{p}{)}\PY{p}{)}
\PY{n}{lyapunov\PYZus{}exponents} \PY{o}{=} \PY{n}{lyapunov\PYZus{}spectrum\PYZus{}c}\PY{p}{(}\PY{n}{valores\PYZus{}x}\PY{p}{,} \PY{n}{valores\PYZus{}r}\PY{p}{,} \PY{n}{window\PYZus{}size}\PY{p}{)}
\PY{n}{fig}\PY{p}{,} \PY{n}{ax1} \PY{o}{=} \PY{n}{plt}\PY{o}{.}\PY{n}{subplots}\PY{p}{(}\PY{n}{figsize}\PY{o}{=}\PY{p}{(}\PY{l+m+mi}{12}\PY{p}{,} \PY{l+m+mi}{6}\PY{p}{)}\PY{p}{)}

\PY{n}{color} \PY{o}{=} \PY{l+s+s1}{\PYZsq{}}\PY{l+s+s1}{tab:blue}\PY{l+s+s1}{\PYZsq{}}
\PY{n}{ax1}\PY{o}{.}\PY{n}{set\PYZus{}xlabel}\PY{p}{(}\PY{l+s+s1}{\PYZsq{}}\PY{l+s+s1}{Tiempo}\PY{l+s+s1}{\PYZsq{}}\PY{p}{)}
\PY{n}{ax1}\PY{o}{.}\PY{n}{set\PYZus{}ylabel}\PY{p}{(}\PY{l+s+s1}{\PYZsq{}}\PY{l+s+s1}{Serie Temporal}\PY{l+s+s1}{\PYZsq{}}\PY{p}{,} \PY{n}{color}\PY{o}{=}\PY{n}{color}\PY{p}{)}
\PY{n}{ax1}\PY{o}{.}\PY{n}{scatter}\PY{p}{(}\PY{n}{pasos}\PY{p}{,} \PY{n}{valores\PYZus{}x}\PY{p}{,} \PY{n}{color}\PY{o}{=}\PY{n}{color}\PY{p}{,} \PY{n}{label}\PY{o}{=}\PY{l+s+s1}{\PYZsq{}}\PY{l+s+s1}{Serie Temporal}\PY{l+s+s1}{\PYZsq{}}\PY{p}{,} \PY{n}{marker}\PY{o}{=}\PY{l+s+s1}{\PYZsq{}}\PY{l+s+s1}{o}\PY{l+s+s1}{\PYZsq{}}\PY{p}{,} \PY{n}{s}\PY{o}{=}\PY{l+m+mf}{0.1}\PY{p}{)}
\PY{n}{ax1}\PY{o}{.}\PY{n}{tick\PYZus{}params}\PY{p}{(}\PY{n}{axis}\PY{o}{=}\PY{l+s+s1}{\PYZsq{}}\PY{l+s+s1}{y}\PY{l+s+s1}{\PYZsq{}}\PY{p}{,} \PY{n}{labelcolor}\PY{o}{=}\PY{n}{color}\PY{p}{)}
\PY{n}{ax2} \PY{o}{=} \PY{n}{ax1}\PY{o}{.}\PY{n}{twinx}\PY{p}{(}\PY{p}{)}
\PY{n}{color} \PY{o}{=} \PY{l+s+s1}{\PYZsq{}}\PY{l+s+s1}{tab:red}\PY{l+s+s1}{\PYZsq{}}
\PY{n}{ax2}\PY{o}{.}\PY{n}{set\PYZus{}ylabel}\PY{p}{(}\PY{l+s+s1}{\PYZsq{}}\PY{l+s+s1}{Exponente de Lyapunov}\PY{l+s+s1}{\PYZsq{}}\PY{p}{,} \PY{n}{color}\PY{o}{=}\PY{n}{color}\PY{p}{)}
\PY{n}{ax2}\PY{o}{.}\PY{n}{plot}\PY{p}{(}\PY{n}{np}\PY{o}{.}\PY{n}{arange}\PY{p}{(}\PY{n+nb}{len}\PY{p}{(}\PY{n}{lyapunov\PYZus{}exponents}\PY{p}{)}\PY{p}{)} \PY{o}{*} \PY{n}{window\PYZus{}size} \PY{o}{+} \PY{n}{window\PYZus{}size} \PY{o}{/}\PY{o}{/} \PY{l+m+mi}{2}\PY{p}{,} \PY{n}{lyapunov\PYZus{}exponents}\PY{p}{,} \PY{n}{color}\PY{o}{=}\PY{n}{color}\PY{p}{,} \PY{n}{label}\PY{o}{=}\PY{l+s+s1}{\PYZsq{}}\PY{l+s+s1}{Exponente de Lyapunov}\PY{l+s+s1}{\PYZsq{}}\PY{p}{)}
\PY{n}{ax2}\PY{o}{.}\PY{n}{tick\PYZus{}params}\PY{p}{(}\PY{n}{axis}\PY{o}{=}\PY{l+s+s1}{\PYZsq{}}\PY{l+s+s1}{y}\PY{l+s+s1}{\PYZsq{}}\PY{p}{,} \PY{n}{labelcolor}\PY{o}{=}\PY{n}{color}\PY{p}{)}

\PY{n}{fig}\PY{o}{.}\PY{n}{suptitle}\PY{p}{(}\PY{l+s+s1}{\PYZsq{}}\PY{l+s+s1}{Serie Temporal y Espectro de Exponentes de Lyapunov}\PY{l+s+s1}{\PYZsq{}}\PY{p}{)}
\PY{n}{fig}\PY{o}{.}\PY{n}{tight\PYZus{}layout}\PY{p}{(}\PY{p}{)}
\PY{n}{plt}\PY{o}{.}\PY{n}{show}\PY{p}{(}\PY{p}{)}
\end{Verbatim}
\end{tcolorbox}

    \begin{center}
    \adjustimage{max size={0.9\linewidth}{0.9\paperheight}}{analisisCaos2_files/analisisCaos2_54_0.png}
    \end{center}
    { \hspace*{\fill} \\}
    
    \hypertarget{interpretaciuxf3n}{%
\paragraph{Interpretación:}\label{interpretaciuxf3n}}

\begin{itemize}
\tightlist
\item
  \textbf{Serie Temporal (Azul):} Esta gráfica muestra la evolución
  temporal del sistema cúbico de bifurcación. La estructura muestra una
  serie de bifurcaciones que indican una transición a comportamientos
  caóticos en el sistema.
\item
  \textbf{Exponentes de Lyapunov (Rojo):} Los exponentes de Lyapunov
  superpuestos a la serie temporal indican la sensibilidad a las
  condiciones iniciales del sistema. Un exponente de Lyapunov positivo
  indica comportamiento caótico. En este gráfico, se observa que los
  exponentes de Lyapunov se vuelven positivos en varias regiones, lo
  cual confirma la presencia de caos en el sistema.
\end{itemize}

    \hypertarget{dimensiuxf3n-de-kaplan-yorke}{%
\subsubsection{Dimensión de
Kaplan-Yorke}\label{dimensiuxf3n-de-kaplan-yorke}}

    \begin{tcolorbox}[breakable, size=fbox, boxrule=1pt, pad at break*=1mm,colback=cellbackground, colframe=cellborder]
\prompt{In}{incolor}{535}{\boxspacing}
\begin{Verbatim}[commandchars=\\\{\}]
\PY{n}{kaplan\PYZus{}yorke\PYZus{}dimension} \PY{o}{=} \PY{n}{calcular\PYZus{}dimension\PYZus{}kaplan\PYZus{}yorke}\PY{p}{(}\PY{n}{lyapunov\PYZus{}exponents}\PY{p}{)}
\PY{n+nb}{print}\PY{p}{(}\PY{l+s+s2}{\PYZdq{}}\PY{l+s+s2}{Dimensión de Kaplan\PYZhy{}Yorke:}\PY{l+s+s2}{\PYZdq{}}\PY{p}{,} \PY{n}{kaplan\PYZus{}yorke\PYZus{}dimension}\PY{p}{)}
\end{Verbatim}
\end{tcolorbox}

    \begin{Verbatim}[commandchars=\\\{\}]
Dimensión de Kaplan-Yorke: 1077.3710854151905
    \end{Verbatim}

    \hypertarget{interpretaciuxf3n}{%
\paragraph{Interpretación:}\label{interpretaciuxf3n}}

\begin{itemize}
\tightlist
\item
  La dimensión de Kaplan-Yorke calculada es \textbf{1077.3710854151905}.
  Este valor es alto y sugiere que el sistema tiene un comportamiento
  muy complejo y un atractor de alta dimensión.
\end{itemize}

    \hypertarget{dimensiuxf3n-grassberger-procaccia}{%
\subsubsection{Dimensión
Grassberger-Procaccia}\label{dimensiuxf3n-grassberger-procaccia}}

    \begin{tcolorbox}[breakable, size=fbox, boxrule=1pt, pad at break*=1mm,colback=cellbackground, colframe=cellborder]
\prompt{In}{incolor}{536}{\boxspacing}
\begin{Verbatim}[commandchars=\\\{\}]
\PY{n}{m} \PY{o}{=} \PY{l+m+mi}{10}
\PY{n}{tau} \PY{o}{=} \PY{l+m+mi}{1}
\PY{n}{r\PYZus{}vals} \PY{o}{=} \PY{n}{np}\PY{o}{.}\PY{n}{logspace}\PY{p}{(}\PY{o}{\PYZhy{}}\PY{l+m+mi}{3}\PY{p}{,} \PY{l+m+mi}{0}\PY{p}{,} \PY{l+m+mi}{50}\PY{p}{)}
\PY{n}{dimension\PYZus{}gp} \PY{o}{=} \PY{n}{calcular\PYZus{}dimension\PYZus{}gp}\PY{p}{(}\PY{n}{valores\PYZus{}x}\PY{p}{,} \PY{n}{m}\PY{p}{,} \PY{n}{tau}\PY{p}{,} \PY{n}{r\PYZus{}vals}\PY{p}{)}
\PY{n+nb}{print}\PY{p}{(}\PY{l+s+s2}{\PYZdq{}}\PY{l+s+s2}{La dimensión de Grassberger\PYZhy{}Procaccia es:}\PY{l+s+s2}{\PYZdq{}}\PY{p}{,} \PY{n}{dimension\PYZus{}gp}\PY{p}{)}
\end{Verbatim}
\end{tcolorbox}

    \begin{Verbatim}[commandchars=\\\{\}]
La dimensión de Grassberger-Procaccia es: 0.9262214344050015
    \end{Verbatim}

    \hypertarget{interpretaciuxf3n}{%
\paragraph{Interpretación:}\label{interpretaciuxf3n}}

\begin{itemize}
\tightlist
\item
  La dimensión de Grassberger-Procaccia calculada es
  \textbf{0.9262214344050015}. Este valor sugiere que el atractor del
  sistema se comporta casi como una curva unidimensional. En sistemas
  dinámicos, valores cercanos a 1 pueden indicar que el sistema está en
  una dimensión baja, aunque aún presenta características fractales.
\end{itemize}

    \hypertarget{bifurcaciuxf3n-triangular-de-feigenbaum}{%
\section{Bifurcación Triangular de
Feigenbaum}\label{bifurcaciuxf3n-triangular-de-feigenbaum}}

    \begin{tcolorbox}[breakable, size=fbox, boxrule=1pt, pad at break*=1mm,colback=cellbackground, colframe=cellborder]
\prompt{In}{incolor}{537}{\boxspacing}
\begin{Verbatim}[commandchars=\\\{\}]
\PY{n}{valores\PYZus{}x} \PY{o}{=} \PY{n}{leer\PYZus{}col\PYZus{}csv}\PY{p}{(}\PY{l+s+s2}{\PYZdq{}}\PY{l+s+s2}{datosFeigenbaumTriangular.csv}\PY{l+s+s2}{\PYZdq{}}\PY{p}{,} \PY{l+s+s2}{\PYZdq{}}\PY{l+s+s2}{Valores x}\PY{l+s+s2}{\PYZdq{}}\PY{p}{)}
\PY{n}{valores\PYZus{}k} \PY{o}{=} \PY{n+nb}{range}\PY{p}{(}\PY{l+m+mi}{1}\PY{p}{,} \PY{n+nb}{len}\PY{p}{(}\PY{n}{valores\PYZus{}x}\PY{p}{)}\PY{o}{+}\PY{l+m+mi}{1}\PY{p}{)}
\PY{n}{graficar}\PY{p}{(}\PY{n}{valores\PYZus{}x}\PY{p}{,} \PY{n}{valores\PYZus{}k}\PY{p}{,} \PY{n}{width}\PY{o}{=}\PY{l+m+mi}{10}\PY{p}{,} \PY{n}{height}\PY{o}{=}\PY{l+m+mi}{7} \PY{p}{,}\PY{n}{titulo}\PY{o}{=}\PY{l+s+s2}{\PYZdq{}}\PY{l+s+s2}{Bifurcacion Triangular de Feigenbaum}\PY{l+s+s2}{\PYZdq{}}\PY{p}{)}
\end{Verbatim}
\end{tcolorbox}

    \begin{center}
    \adjustimage{max size={0.9\linewidth}{0.9\paperheight}}{analisisCaos2_files/analisisCaos2_63_0.png}
    \end{center}
    { \hspace*{\fill} \\}
    
    \hypertarget{probabiluxedstico}{%
\subsection{Probabilístico}\label{probabiluxedstico}}

    \begin{tcolorbox}[breakable, size=fbox, boxrule=1pt, pad at break*=1mm,colback=cellbackground, colframe=cellborder]
\prompt{In}{incolor}{538}{\boxspacing}
\begin{Verbatim}[commandchars=\\\{\}]
\PY{n}{feigen\PYZus{}t} \PY{o}{=} \PY{n}{DistribucionProbabilidad}\PY{p}{(}\PY{n}{valores\PYZus{}x}\PY{p}{)}
\PY{n}{feigen\PYZus{}t}\PY{o}{.}\PY{n}{mostrar\PYZus{}metricas}\PY{p}{(}\PY{p}{)}
\PY{n}{plotear\PYZus{}hist}\PY{p}{(}\PY{n}{valores\PYZus{}x}\PY{p}{,} \PY{l+s+s2}{\PYZdq{}}\PY{l+s+s2}{Histograma de Feigenbaum Triangular}\PY{l+s+s2}{\PYZdq{}}\PY{p}{,} \PY{l+s+s2}{\PYZdq{}}\PY{l+s+s2}{Valor}\PY{l+s+s2}{\PYZdq{}}\PY{p}{,} \PY{l+s+s2}{\PYZdq{}}\PY{l+s+s2}{Frecuencia}\PY{l+s+s2}{\PYZdq{}}\PY{p}{,} \PY{l+s+s2}{\PYZdq{}}\PY{l+s+s2}{sturges}\PY{l+s+s2}{\PYZdq{}}\PY{p}{)}
\end{Verbatim}
\end{tcolorbox}

    \begin{Verbatim}[commandchars=\\\{\}]
--- Métricas para Variable 1 ---
Media: 0.5634713853957487
Mediana: 0.5823316406627415
Moda: 0.42023661123507994
Media Geométrica: 0.5578876915654843
Rango: 0.27959009254991707
Desviación Estándar: 0.0786209973339597
Varianza: 0.0061812612217865
Asimetría: -0.0877370026433518
Curtosis: -1.5352630947880699
Entropia: 11.512925464970223
Coeficiente de Variación: 0.1395297070475744

    \end{Verbatim}

    \begin{center}
    \adjustimage{max size={0.9\linewidth}{0.9\paperheight}}{analisisCaos2_files/analisisCaos2_65_1.png}
    \end{center}
    { \hspace*{\fill} \\}
    
    \hypertarget{anuxe1lisis-de-muxe9tricas-estaduxedsticas-para-el-modelo-de-bifurcaciones-triangular-de-feigenbaum}{%
\subsection{Análisis de Métricas Estadísticas para el Modelo de
Bifurcaciones Triangular de
Feigenbaum}\label{anuxe1lisis-de-muxe9tricas-estaduxedsticas-para-el-modelo-de-bifurcaciones-triangular-de-feigenbaum}}

\begin{itemize}
\item
  \textbf{Media (0.5364)} y \textbf{Mediana (0.5173)}: Estos valores
  cercanos entre sí indican que los datos no están sesgados de manera
  significativa hacia ningún extremo del rango, lo cual es un indicio de
  una distribución relativamente simétrica alrededor de este valor
  central.
\item
  \textbf{Desviación Estándar (0.0445)} y \textbf{Varianza (0.0020)}:
  Estos valores bajos muestran que los datos están bastante concentrados
  alrededor de la media, indicando una baja dispersión en los resultados
  del modelo.
\item
  \textbf{Rango (0.3249)}: Un rango moderado sugiere que mientras los
  valores exploran una variedad de estados, no se extienden por todo el
  espectro posible, lo que podría indicar limitaciones en la dinámica
  explorada por este modelo.
\item
  \textbf{Asimetría (0.2056)}: Una asimetría ligeramente positiva indica
  una cola más pesada hacia el lado derecho de la mediana. Esto puede
  ser indicativo de episodios donde el sistema explora valores más altos
  con menos frecuencia.
\item
  \textbf{Curtosis (-1.5760)}: Una curtosis negativa muestra una
  distribución más plana que una distribución normal. Esto sugiere que
  los valores están más uniformemente distribuidos a lo largo del rango,
  sin un pico pronunciado.
\item
  \textbf{Entropía (11.9184)}: La alta entropía se mantiene como un
  indicador de la diversidad y complejidad en los estados del sistema,
  reafirmando la presencia de un comportamiento caótico y la
  variabilidad en los datos generados.
\item
  \textbf{Coeficiente de Variación (0.0828)}: Este valor bajo indica que
  la desviación estándar es pequeña en relación con la media, lo que
  sugiere que la variabilidad de los datos, aunque presente, no domina
  la escala de los datos.
\end{itemize}

\hypertarget{conclusiuxf3n}{%
\subsubsection{Conclusión}\label{conclusiuxf3n}}

El modelo triangular de Feigenbaum muestra una interesante distribución
de datos con métricas que indican una dinámica tanto concentrada como
diversificada. La alta entropía junto con una curtosis negativa y
asimetría positiva sugiere que, aunque el sistema es predominantemente
estable en torno a ciertos valores, también es capaz de explorar estados
menos comunes, lo cual es característico de los comportamientos
dinámicos no lineales y caóticos observados en este tipo de modelos.

    \hypertarget{cauxf3tico}{%
\subsection{Caótico}\label{cauxf3tico}}

    \hypertarget{exponentes-de-laypounov}{%
\subsubsection{Exponentes de Laypounov}\label{exponentes-de-laypounov}}

    \begin{tcolorbox}[breakable, size=fbox, boxrule=1pt, pad at break*=1mm,colback=cellbackground, colframe=cellborder]
\prompt{In}{incolor}{539}{\boxspacing}
\begin{Verbatim}[commandchars=\\\{\}]
\PY{k}{def} \PY{n+nf}{lyapunov\PYZus{}spectrum\PYZus{}t}\PY{p}{(}\PY{n}{series}\PY{p}{,} \PY{n}{rates}\PY{p}{,} \PY{n}{window\PYZus{}size}\PY{p}{)}\PY{p}{:}
    \PY{n}{N} \PY{o}{=} \PY{n+nb}{len}\PY{p}{(}\PY{n}{series}\PY{p}{)}
    \PY{n}{n\PYZus{}windows} \PY{o}{=} \PY{n}{N} \PY{o}{/}\PY{o}{/} \PY{n}{window\PYZus{}size}
    \PY{n}{lyapunov\PYZus{}exponents} \PY{o}{=} \PY{p}{[}\PY{p}{]}

    \PY{k}{for} \PY{n}{i} \PY{o+ow}{in} \PY{n+nb}{range}\PY{p}{(}\PY{n}{n\PYZus{}windows}\PY{p}{)}\PY{p}{:}
        \PY{n}{window} \PY{o}{=} \PY{n}{series}\PY{p}{[}\PY{n}{i} \PY{o}{*} \PY{n}{window\PYZus{}size}\PY{p}{:}\PY{p}{(}\PY{n}{i} \PY{o}{+} \PY{l+m+mi}{1}\PY{p}{)} \PY{o}{*} \PY{n}{window\PYZus{}size}\PY{p}{]}
        \PY{n}{rate\PYZus{}window} \PY{o}{=} \PY{n}{rates}\PY{p}{[}\PY{n}{i} \PY{o}{*} \PY{n}{window\PYZus{}size}\PY{p}{:}\PY{p}{(}\PY{n}{i} \PY{o}{+} \PY{l+m+mi}{1}\PY{p}{)} \PY{o}{*} \PY{n}{window\PYZus{}size}\PY{p}{]}
        \PY{n}{sum\PYZus{}log\PYZus{}der} \PY{o}{=} \PY{l+m+mf}{0.0}
        \PY{k}{for} \PY{n}{j} \PY{o+ow}{in} \PY{n+nb}{range}\PY{p}{(}\PY{l+m+mi}{1}\PY{p}{,} \PY{n}{window\PYZus{}size}\PY{p}{)}\PY{p}{:}
            \PY{n}{x} \PY{o}{=} \PY{n}{window}\PY{p}{[}\PY{n}{j}\PY{p}{]}
            \PY{n}{r} \PY{o}{=} \PY{n}{rate\PYZus{}window}\PY{p}{[}\PY{n}{j}\PY{p}{]}
            \PY{k}{if} \PY{n}{x} \PY{o}{\PYZlt{}}\PY{o}{=} \PY{l+m+mf}{0.5}\PY{p}{:}
                \PY{n}{derivative} \PY{o}{=} \PY{n}{r}
            \PY{k}{else}\PY{p}{:}
                \PY{n}{derivative} \PY{o}{=} \PY{o}{\PYZhy{}}\PY{n}{r}
            \PY{n}{sum\PYZus{}log\PYZus{}der} \PY{o}{+}\PY{o}{=} \PY{n}{np}\PY{o}{.}\PY{n}{log}\PY{p}{(}\PY{n+nb}{abs}\PY{p}{(}\PY{n}{derivative}\PY{p}{)}\PY{p}{)}
        
        \PY{n}{le} \PY{o}{=} \PY{n}{sum\PYZus{}log\PYZus{}der} \PY{o}{/} \PY{n}{window\PYZus{}size}
        \PY{n}{lyapunov\PYZus{}exponents}\PY{o}{.}\PY{n}{append}\PY{p}{(}\PY{n}{le}\PY{p}{)}
    
    \PY{k}{return} \PY{n}{np}\PY{o}{.}\PY{n}{array}\PY{p}{(}\PY{n}{lyapunov\PYZus{}exponents}\PY{p}{)}
\end{Verbatim}
\end{tcolorbox}

    \begin{tcolorbox}[breakable, size=fbox, boxrule=1pt, pad at break*=1mm,colback=cellbackground, colframe=cellborder]
\prompt{In}{incolor}{540}{\boxspacing}
\begin{Verbatim}[commandchars=\\\{\}]
\PY{n}{valores\PYZus{}x}\PY{p}{,} \PY{n}{valores\PYZus{}r} \PY{o}{=} \PY{n}{leer\PYZus{}csv\PYZus{}a\PYZus{}numpy}\PY{p}{(}\PY{l+s+s2}{\PYZdq{}}\PY{l+s+s2}{datosFeigenbaumTriangular.csv}\PY{l+s+s2}{\PYZdq{}}\PY{p}{)}
\PY{n}{pasos} \PY{o}{=} \PY{n+nb}{range}\PY{p}{(}\PY{l+m+mi}{0}\PY{p}{,} \PY{n+nb}{len}\PY{p}{(}\PY{n}{valores\PYZus{}x}\PY{p}{)}\PY{p}{)}
\PY{n}{lyapunov\PYZus{}exponents} \PY{o}{=} \PY{n}{lyapunov\PYZus{}spectrum\PYZus{}t}\PY{p}{(}\PY{n}{valores\PYZus{}x}\PY{p}{,} \PY{n}{valores\PYZus{}r}\PY{p}{,} \PY{n}{window\PYZus{}size}\PY{p}{)}
\PY{n}{fig}\PY{p}{,} \PY{n}{ax1} \PY{o}{=} \PY{n}{plt}\PY{o}{.}\PY{n}{subplots}\PY{p}{(}\PY{n}{figsize}\PY{o}{=}\PY{p}{(}\PY{l+m+mi}{12}\PY{p}{,} \PY{l+m+mi}{6}\PY{p}{)}\PY{p}{)}

\PY{n}{color} \PY{o}{=} \PY{l+s+s1}{\PYZsq{}}\PY{l+s+s1}{tab:blue}\PY{l+s+s1}{\PYZsq{}}
\PY{n}{ax1}\PY{o}{.}\PY{n}{set\PYZus{}xlabel}\PY{p}{(}\PY{l+s+s1}{\PYZsq{}}\PY{l+s+s1}{Tiempo}\PY{l+s+s1}{\PYZsq{}}\PY{p}{)}
\PY{n}{ax1}\PY{o}{.}\PY{n}{set\PYZus{}ylabel}\PY{p}{(}\PY{l+s+s1}{\PYZsq{}}\PY{l+s+s1}{Serie Temporal}\PY{l+s+s1}{\PYZsq{}}\PY{p}{,} \PY{n}{color}\PY{o}{=}\PY{n}{color}\PY{p}{)}
\PY{n}{ax1}\PY{o}{.}\PY{n}{scatter}\PY{p}{(}\PY{n}{pasos}\PY{p}{,} \PY{n}{valores\PYZus{}x}\PY{p}{,} \PY{n}{color}\PY{o}{=}\PY{n}{color}\PY{p}{,} \PY{n}{label}\PY{o}{=}\PY{l+s+s1}{\PYZsq{}}\PY{l+s+s1}{Serie Temporal}\PY{l+s+s1}{\PYZsq{}}\PY{p}{,} \PY{n}{marker}\PY{o}{=}\PY{l+s+s1}{\PYZsq{}}\PY{l+s+s1}{o}\PY{l+s+s1}{\PYZsq{}}\PY{p}{,} \PY{n}{s}\PY{o}{=}\PY{l+m+mf}{0.1}\PY{p}{)}
\PY{n}{ax1}\PY{o}{.}\PY{n}{tick\PYZus{}params}\PY{p}{(}\PY{n}{axis}\PY{o}{=}\PY{l+s+s1}{\PYZsq{}}\PY{l+s+s1}{y}\PY{l+s+s1}{\PYZsq{}}\PY{p}{,} \PY{n}{labelcolor}\PY{o}{=}\PY{n}{color}\PY{p}{)}

\PY{n}{ax2} \PY{o}{=} \PY{n}{ax1}\PY{o}{.}\PY{n}{twinx}\PY{p}{(}\PY{p}{)}
\PY{n}{color} \PY{o}{=} \PY{l+s+s1}{\PYZsq{}}\PY{l+s+s1}{tab:red}\PY{l+s+s1}{\PYZsq{}}
\PY{n}{ax2}\PY{o}{.}\PY{n}{set\PYZus{}ylabel}\PY{p}{(}\PY{l+s+s1}{\PYZsq{}}\PY{l+s+s1}{Exponente de Lyapunov}\PY{l+s+s1}{\PYZsq{}}\PY{p}{,} \PY{n}{color}\PY{o}{=}\PY{n}{color}\PY{p}{)}
\PY{n}{ax2}\PY{o}{.}\PY{n}{plot}\PY{p}{(}\PY{n}{np}\PY{o}{.}\PY{n}{arange}\PY{p}{(}\PY{n+nb}{len}\PY{p}{(}\PY{n}{lyapunov\PYZus{}exponents}\PY{p}{)}\PY{p}{)} \PY{o}{*} \PY{n}{window\PYZus{}size} \PY{o}{+} \PY{n}{window\PYZus{}size} \PY{o}{/}\PY{o}{/} \PY{l+m+mi}{2}\PY{p}{,} \PY{n}{lyapunov\PYZus{}exponents}\PY{p}{,} \PY{n}{color}\PY{o}{=}\PY{n}{color}\PY{p}{,} \PY{n}{label}\PY{o}{=}\PY{l+s+s1}{\PYZsq{}}\PY{l+s+s1}{Exponente de Lyapunov}\PY{l+s+s1}{\PYZsq{}}\PY{p}{)}
\PY{n}{ax2}\PY{o}{.}\PY{n}{tick\PYZus{}params}\PY{p}{(}\PY{n}{axis}\PY{o}{=}\PY{l+s+s1}{\PYZsq{}}\PY{l+s+s1}{y}\PY{l+s+s1}{\PYZsq{}}\PY{p}{,} \PY{n}{labelcolor}\PY{o}{=}\PY{n}{color}\PY{p}{)}

\PY{n}{fig}\PY{o}{.}\PY{n}{suptitle}\PY{p}{(}\PY{l+s+s1}{\PYZsq{}}\PY{l+s+s1}{Serie Temporal y Espectro de Exponentes de Lyapunov}\PY{l+s+s1}{\PYZsq{}}\PY{p}{)}
\PY{n}{fig}\PY{o}{.}\PY{n}{tight\PYZus{}layout}\PY{p}{(}\PY{p}{)}
\PY{n}{plt}\PY{o}{.}\PY{n}{show}\PY{p}{(}\PY{p}{)}
\end{Verbatim}
\end{tcolorbox}

    \begin{center}
    \adjustimage{max size={0.9\linewidth}{0.9\paperheight}}{analisisCaos2_files/analisisCaos2_70_0.png}
    \end{center}
    { \hspace*{\fill} \\}
    
    \hypertarget{interpretaciuxf3n}{%
\paragraph{Interpretación:}\label{interpretaciuxf3n}}

\begin{itemize}
\tightlist
\item
  \textbf{Serie Temporal (Azul):} Esta gráfica muestra la evolución
  temporal del sistema triangular de bifurcación. La estructura no
  parece exhibir bifurcaciones tradicionales, pero hay una separación
  clara en dos bandas paralelas, lo que sugiere un comportamiento
  dinámico interesante.
\item
  \textbf{Exponentes de Lyapunov (Rojo):} Los exponentes de Lyapunov
  superpuestos a la serie temporal indican la sensibilidad a las
  condiciones iniciales del sistema. La línea roja inclinada muestra una
  tendencia positiva constante, lo que puede indicar que el sistema es
  caótico pero con una estructura diferente a la de los sistemas
  caóticos típicos.
\end{itemize}

    \hypertarget{dimensiuxf3n-de-kaplan-yorke}{%
\subsubsection{Dimensión de
Kaplan-Yorke}\label{dimensiuxf3n-de-kaplan-yorke}}

    \begin{tcolorbox}[breakable, size=fbox, boxrule=1pt, pad at break*=1mm,colback=cellbackground, colframe=cellborder]
\prompt{In}{incolor}{541}{\boxspacing}
\begin{Verbatim}[commandchars=\\\{\}]
\PY{n}{kaplan\PYZus{}yorke\PYZus{}dimension} \PY{o}{=} \PY{n}{calcular\PYZus{}dimension\PYZus{}kaplan\PYZus{}yorke}\PY{p}{(}\PY{n}{lyapunov\PYZus{}exponents}\PY{p}{)}
\PY{n+nb}{print}\PY{p}{(}\PY{l+s+s2}{\PYZdq{}}\PY{l+s+s2}{Dimensión de Kaplan\PYZhy{}Yorke:}\PY{l+s+s2}{\PYZdq{}}\PY{p}{,} \PY{n}{kaplan\PYZus{}yorke\PYZus{}dimension}\PY{p}{)}
\end{Verbatim}
\end{tcolorbox}

    \begin{Verbatim}[commandchars=\\\{\}]
Dimensión de Kaplan-Yorke: 2141.8077760850883
    \end{Verbatim}

    \hypertarget{interpretaciuxf3n}{%
\paragraph{Interpretación:}\label{interpretaciuxf3n}}

\begin{itemize}
\tightlist
\item
  La dimensión de Kaplan-Yorke calculada es \textbf{2141.8077760850883}.
  Este valor es extremadamente alto y sugiere que el sistema tiene un
  comportamiento muy complejo y un atractor de alta dimensión.
\end{itemize}

    \hypertarget{dimensiuxf3n-grassberger-procaccia}{%
\subsubsection{Dimensión
Grassberger-Procaccia}\label{dimensiuxf3n-grassberger-procaccia}}

    \begin{tcolorbox}[breakable, size=fbox, boxrule=1pt, pad at break*=1mm,colback=cellbackground, colframe=cellborder]
\prompt{In}{incolor}{542}{\boxspacing}
\begin{Verbatim}[commandchars=\\\{\}]
\PY{n}{m} \PY{o}{=} \PY{l+m+mi}{10}
\PY{n}{tau} \PY{o}{=} \PY{l+m+mi}{1}
\PY{n}{r\PYZus{}vals} \PY{o}{=} \PY{n}{np}\PY{o}{.}\PY{n}{logspace}\PY{p}{(}\PY{o}{\PYZhy{}}\PY{l+m+mi}{3}\PY{p}{,} \PY{l+m+mi}{0}\PY{p}{,} \PY{l+m+mi}{50}\PY{p}{)}
\PY{n}{dimension\PYZus{}gp} \PY{o}{=} \PY{n}{calcular\PYZus{}dimension\PYZus{}gp}\PY{p}{(}\PY{n}{valores\PYZus{}x}\PY{p}{,} \PY{n}{m}\PY{p}{,} \PY{n}{tau}\PY{p}{,} \PY{n}{r\PYZus{}vals}\PY{p}{)}
\PY{n+nb}{print}\PY{p}{(}\PY{l+s+s2}{\PYZdq{}}\PY{l+s+s2}{La dimensión de Grassberger\PYZhy{}Procaccia es:}\PY{l+s+s2}{\PYZdq{}}\PY{p}{,} \PY{n}{dimension\PYZus{}gp}\PY{p}{)}
\end{Verbatim}
\end{tcolorbox}

    \begin{Verbatim}[commandchars=\\\{\}]
La dimensión de Grassberger-Procaccia es: 1.761396392229035
    \end{Verbatim}

    \hypertarget{interpretaciuxf3n}{%
\paragraph{Interpretación:}\label{interpretaciuxf3n}}

\begin{itemize}
\tightlist
\item
  La dimensión de Grassberger-Procaccia calculada es
  \textbf{1.761396392229035}. Este valor sugiere que el atractor del
  sistema tiene una dimensión fractal intermedia, lo cual puede indicar
  un comportamiento caótico complejo pero en una dimensión más baja
  comparada con la dimensión de Kaplan-Yorke obtenida.
\end{itemize}

    \hypertarget{mapa-de-henon}{%
\section{Mapa de Henon}\label{mapa-de-henon}}

    \begin{tcolorbox}[breakable, size=fbox, boxrule=1pt, pad at break*=1mm,colback=cellbackground, colframe=cellborder]
\prompt{In}{incolor}{543}{\boxspacing}
\begin{Verbatim}[commandchars=\\\{\}]
\PY{n}{valores\PYZus{}x} \PY{o}{=} \PY{n}{leer\PYZus{}col\PYZus{}csv}\PY{p}{(}\PY{l+s+s2}{\PYZdq{}}\PY{l+s+s2}{datosHenon.csv}\PY{l+s+s2}{\PYZdq{}}\PY{p}{,} \PY{l+s+s2}{\PYZdq{}}\PY{l+s+s2}{Valores x}\PY{l+s+s2}{\PYZdq{}}\PY{p}{)}
\PY{n}{valores\PYZus{}y} \PY{o}{=} \PY{n}{leer\PYZus{}col\PYZus{}csv}\PY{p}{(}\PY{l+s+s2}{\PYZdq{}}\PY{l+s+s2}{datosHenon.csv}\PY{l+s+s2}{\PYZdq{}}\PY{p}{,} \PY{l+s+s2}{\PYZdq{}}\PY{l+s+s2}{Valores y}\PY{l+s+s2}{\PYZdq{}}\PY{p}{)}
\PY{n}{valores\PYZus{}k} \PY{o}{=} \PY{n+nb}{range}\PY{p}{(}\PY{l+m+mi}{1}\PY{p}{,} \PY{n+nb}{len}\PY{p}{(}\PY{n}{valores\PYZus{}x}\PY{p}{)}\PY{o}{+}\PY{l+m+mi}{1}\PY{p}{)}
\PY{n}{graficar}\PY{p}{(}\PY{n}{valores\PYZus{}x}\PY{p}{,} \PY{n}{valores\PYZus{}y}\PY{p}{,} \PY{n}{width}\PY{o}{=}\PY{l+m+mi}{10}\PY{p}{,} \PY{n}{height}\PY{o}{=}\PY{l+m+mi}{7} \PY{p}{,}\PY{n}{titulo}\PY{o}{=}\PY{l+s+s2}{\PYZdq{}}\PY{l+s+s2}{Bifurcacion Triangular de Feigenbaum}\PY{l+s+s2}{\PYZdq{}}\PY{p}{)}
\end{Verbatim}
\end{tcolorbox}

    \begin{center}
    \adjustimage{max size={0.9\linewidth}{0.9\paperheight}}{analisisCaos2_files/analisisCaos2_79_0.png}
    \end{center}
    { \hspace*{\fill} \\}
    
    \hypertarget{probabiluxedstico}{%
\subsection{Probabilístico}\label{probabiluxedstico}}

    \begin{tcolorbox}[breakable, size=fbox, boxrule=1pt, pad at break*=1mm,colback=cellbackground, colframe=cellborder]
\prompt{In}{incolor}{544}{\boxspacing}
\begin{Verbatim}[commandchars=\\\{\}]
\PY{n}{henon} \PY{o}{=} \PY{n}{DistribucionProbabilidad}\PY{p}{(}\PY{n}{valores\PYZus{}x}\PY{p}{,} \PY{n}{valores\PYZus{}y}\PY{p}{)}
\PY{n}{henon}\PY{o}{.}\PY{n}{mostrar\PYZus{}metricas}\PY{p}{(}\PY{p}{)}
\PY{n}{plotear\PYZus{}hist}\PY{p}{(}\PY{n}{valores\PYZus{}x}\PY{p}{,} \PY{l+s+s2}{\PYZdq{}}\PY{l+s+s2}{Histograma de Henon X}\PY{l+s+s2}{\PYZdq{}}\PY{p}{,} \PY{l+s+s2}{\PYZdq{}}\PY{l+s+s2}{Valor}\PY{l+s+s2}{\PYZdq{}}\PY{p}{,} \PY{l+s+s2}{\PYZdq{}}\PY{l+s+s2}{Frecuencia}\PY{l+s+s2}{\PYZdq{}}\PY{p}{,} \PY{l+s+s2}{\PYZdq{}}\PY{l+s+s2}{sturges}\PY{l+s+s2}{\PYZdq{}}\PY{p}{)}
\PY{n}{plotear\PYZus{}hist}\PY{p}{(}\PY{n}{valores\PYZus{}y}\PY{p}{,} \PY{l+s+s2}{\PYZdq{}}\PY{l+s+s2}{Histograma de Henon Y}\PY{l+s+s2}{\PYZdq{}}\PY{p}{,} \PY{l+s+s2}{\PYZdq{}}\PY{l+s+s2}{Valor}\PY{l+s+s2}{\PYZdq{}}\PY{p}{,} \PY{l+s+s2}{\PYZdq{}}\PY{l+s+s2}{Frecuencia}\PY{l+s+s2}{\PYZdq{}}\PY{p}{,} \PY{l+s+s2}{\PYZdq{}}\PY{l+s+s2}{sturges}\PY{l+s+s2}{\PYZdq{}}\PY{p}{)}
\PY{n}{graficar}\PY{p}{(}\PY{n}{valores\PYZus{}x}\PY{p}{,} \PY{n}{valores\PYZus{}k}\PY{p}{,} \PY{n}{width}\PY{o}{=}\PY{l+m+mi}{10}\PY{p}{,} \PY{n}{height}\PY{o}{=}\PY{l+m+mi}{7} \PY{p}{,}\PY{n}{titulo}\PY{o}{=}\PY{l+s+s2}{\PYZdq{}}\PY{l+s+s2}{Variable X vs k}\PY{l+s+s2}{\PYZdq{}}\PY{p}{)}
\PY{n}{graficar}\PY{p}{(}\PY{n}{valores\PYZus{}y}\PY{p}{,} \PY{n}{valores\PYZus{}k}\PY{p}{,} \PY{n}{width}\PY{o}{=}\PY{l+m+mi}{10}\PY{p}{,} \PY{n}{height}\PY{o}{=}\PY{l+m+mi}{7} \PY{p}{,}\PY{n}{titulo}\PY{o}{=}\PY{l+s+s2}{\PYZdq{}}\PY{l+s+s2}{Variable y vs k}\PY{l+s+s2}{\PYZdq{}}\PY{p}{)}
\end{Verbatim}
\end{tcolorbox}

    \begin{Verbatim}[commandchars=\\\{\}]
--- Métricas para Variable 1 ---
Media: 0.25826445541982584
Mediana: 0.41086307203174455
Moda: -1.2846637827292586
Media Geométrica: nan
Rango: 2.557636523935283
Desviación Estándar: 0.7200413783449912
Varianza: 0.5184595865289547
Asimetría: -0.4982432032837881
Curtosis: -0.8744955533890941
Entropia: 11.512925464970223
Coeficiente de Variación: 2.7880002967288573

--- Métricas para Variable 2 ---
Media: 0.07747980171430803
Mediana: 0.12325892160952336
Moda: -0.3853991348187776
Media Geométrica: nan
Rango: 0.7672909571805848
Desviación Estándar: 0.21601284253966427
Varianza: 0.046661548142065794
Asimetría: -0.498241172200149
Curtosis: -0.8745012960453131
Entropia: 11.512925464970223
Coeficiente de Variación: 2.7879890985804323

Matriz de Correlación:
[[ 1.         -0.31554664]
 [-0.31554664  1.        ]]
    \end{Verbatim}

    \begin{Verbatim}[commandchars=\\\{\}]
/home/rodrigo/.local/lib/python3.10/site-packages/scipy/stats/\_stats\_py.py:197:
RuntimeWarning:

invalid value encountered in log

    \end{Verbatim}

    \begin{center}
    \adjustimage{max size={0.9\linewidth}{0.9\paperheight}}{analisisCaos2_files/analisisCaos2_81_2.png}
    \end{center}
    { \hspace*{\fill} \\}
    
    \begin{center}
    \adjustimage{max size={0.9\linewidth}{0.9\paperheight}}{analisisCaos2_files/analisisCaos2_81_3.png}
    \end{center}
    { \hspace*{\fill} \\}
    
    \begin{center}
    \adjustimage{max size={0.9\linewidth}{0.9\paperheight}}{analisisCaos2_files/analisisCaos2_81_4.png}
    \end{center}
    { \hspace*{\fill} \\}
    
    \begin{center}
    \adjustimage{max size={0.9\linewidth}{0.9\paperheight}}{analisisCaos2_files/analisisCaos2_81_5.png}
    \end{center}
    { \hspace*{\fill} \\}
    
    \begin{center}
    \adjustimage{max size={0.9\linewidth}{0.9\paperheight}}{analisisCaos2_files/analisisCaos2_81_6.png}
    \end{center}
    { \hspace*{\fill} \\}
    
    \hypertarget{anuxe1lisis-de-muxe9tricas-estaduxedsticas-para-el-modelo-del-mapa-de-henon}{%
\subsection{Análisis de Métricas Estadísticas para el Modelo del Mapa de
Henon}\label{anuxe1lisis-de-muxe9tricas-estaduxedsticas-para-el-modelo-del-mapa-de-henon}}

\hypertarget{variable-x}{%
\paragraph{Variable X}\label{variable-x}}

\begin{itemize}
\item
  \textbf{Media (0.2569)} y \textbf{Mediana (0.4109)}: Estos valores
  indican que, aunque la media está más cerca de cero, la mediana más
  alta sugiere una distribución con una cola hacia valores mayores. Esto
  es típico en sistemas caóticos donde la distribución no es simétrica.
\item
  \textbf{Rango (2.5576)}: Un rango amplio muestra que la variable X
  explora un espectro extenso de valores, lo cual es esperado en
  dinámicas caóticas como las del mapa de Henon.
\item
  \textbf{Desviación Estándar (0.7209)} y \textbf{Varianza (0.5197)}:
  Estas métricas altas reflejan la considerable dispersión de los datos,
  indicativa de la alta variabilidad inherente a este sistema.
\item
  \textbf{Entropía (11.5129)}: Una entropía muy alta sugiere una
  distribución compleja y diversa de los valores, característica de los
  comportamientos caóticos.
\end{itemize}

\hypertarget{variable-y}{%
\paragraph{Variable Y}\label{variable-y}}

\begin{itemize}
\item
  \textbf{Media (0.0771)} y \textbf{Mediana (0.1233)}: Similar a la
  variable X, los valores muestran una distribución con ligera
  asimetría, aunque la diferencia entre la media y la mediana es menor.
\item
  \textbf{Desviación Estándar (0.2163)}: Menor en comparación con la
  variable X, indicando menos variabilidad en esta variable.
\end{itemize}

\hypertarget{anuxe1lisis-de-correlaciuxf3n-y-gruxe1fica-de-dispersiuxf3n}{%
\subsubsection{Análisis de Correlación y Gráfica de
Dispersión}\label{anuxe1lisis-de-correlaciuxf3n-y-gruxe1fica-de-dispersiuxf3n}}

\begin{itemize}
\tightlist
\item
  \textbf{Matriz de Correlación}: Un coeficiente de -0.315 indica una
  correlación negativa moderada entre las variables. Esto sugiere que a
  medida que una variable aumenta, la otra tiende a disminuir, lo cual
  es coherente con el comportamiento esperado del mapa de Henon donde
  las variables están interconectadas en un sistema dinámico complejo.
\end{itemize}

\hypertarget{conclusiuxf3n}{%
\subsubsection{Conclusión}\label{conclusiuxf3n}}

Las métricas estadísticas junto con la matriz de correlación y la
gráfica de dispersión revelan un sistema con alta variabilidad y
dinámicas complejas interdependientes. La alta entropía y el amplio
rango de ambas variables subrayan la rica dinámica caótica del mapa de
Henon. Estos resultados son cruciales para entender cómo pequeñas
variaciones en las condiciones iniciales pueden resultar en cambios
significativos en el comportamiento del sistema, un rasgo definitorio
del caos.

    \hypertarget{cauxf3tico}{%
\subsection{Caótico}\label{cauxf3tico}}

    \hypertarget{exponentes-de-laypounov}{%
\subsubsection{Exponentes de Laypounov}\label{exponentes-de-laypounov}}

    \begin{tcolorbox}[breakable, size=fbox, boxrule=1pt, pad at break*=1mm,colback=cellbackground, colframe=cellborder]
\prompt{In}{incolor}{545}{\boxspacing}
\begin{Verbatim}[commandchars=\\\{\}]
\PY{k}{def} \PY{n+nf}{leer\PYZus{}datos}\PY{p}{(}\PY{n}{csv\PYZus{}path}\PY{p}{)}\PY{p}{:}
    \PY{n}{df} \PY{o}{=} \PY{n}{pd}\PY{o}{.}\PY{n}{read\PYZus{}csv}\PY{p}{(}\PY{n}{csv\PYZus{}path}\PY{p}{)}
    \PY{k}{return} \PY{n}{df}\PY{p}{[}\PY{l+s+s1}{\PYZsq{}}\PY{l+s+s1}{Valores x}\PY{l+s+s1}{\PYZsq{}}\PY{p}{]}\PY{o}{.}\PY{n}{values}\PY{p}{,} \PY{n}{df}\PY{p}{[}\PY{l+s+s1}{\PYZsq{}}\PY{l+s+s1}{Valores y}\PY{l+s+s1}{\PYZsq{}}\PY{p}{]}\PY{o}{.}\PY{n}{values}

\PY{k}{def} \PY{n+nf}{jacobian\PYZus{}henon}\PY{p}{(}\PY{n}{x}\PY{p}{,} \PY{n}{y}\PY{p}{,} \PY{n}{a}\PY{o}{=}\PY{l+m+mf}{1.4}\PY{p}{,} \PY{n}{b}\PY{o}{=}\PY{l+m+mf}{0.3}\PY{p}{)}\PY{p}{:}
    \PY{k}{return} \PY{n}{np}\PY{o}{.}\PY{n}{array}\PY{p}{(}\PY{p}{[}\PY{p}{[}\PY{o}{\PYZhy{}}\PY{l+m+mi}{2} \PY{o}{*} \PY{n}{a} \PY{o}{*} \PY{n}{x}\PY{p}{,} \PY{l+m+mi}{1}\PY{p}{]}\PY{p}{,}
                     \PY{p}{[}\PY{n}{b}\PY{p}{,} \PY{l+m+mi}{0}\PY{p}{]}\PY{p}{]}\PY{p}{)}

\PY{k}{def} \PY{n+nf}{calculate\PYZus{}lyapunov\PYZus{}exponents}\PY{p}{(}\PY{n}{x}\PY{p}{,} \PY{n}{y}\PY{p}{,} \PY{n}{a}\PY{o}{=}\PY{l+m+mf}{1.4}\PY{p}{,} \PY{n}{b}\PY{o}{=}\PY{l+m+mf}{0.3}\PY{p}{,} \PY{n}{window\PYZus{}size}\PY{o}{=}\PY{l+m+mi}{100}\PY{p}{)}\PY{p}{:}
    \PY{n}{n} \PY{o}{=} \PY{n+nb}{len}\PY{p}{(}\PY{n}{x}\PY{p}{)}
    \PY{n}{perturbations} \PY{o}{=} \PY{n}{np}\PY{o}{.}\PY{n}{eye}\PY{p}{(}\PY{l+m+mi}{2}\PY{p}{)}
    \PY{n}{lyapunov\PYZus{}exponents} \PY{o}{=} \PY{n}{np}\PY{o}{.}\PY{n}{zeros}\PY{p}{(}\PY{p}{(}\PY{n}{n} \PY{o}{\PYZhy{}} \PY{n}{window\PYZus{}size}\PY{p}{,} \PY{l+m+mi}{2}\PY{p}{)}\PY{p}{)}
    
    \PY{k}{for} \PY{n}{i} \PY{o+ow}{in} \PY{n+nb}{range}\PY{p}{(}\PY{n}{window\PYZus{}size}\PY{p}{,} \PY{n}{n}\PY{p}{)}\PY{p}{:}
        \PY{n}{J} \PY{o}{=} \PY{n}{jacobian\PYZus{}henon}\PY{p}{(}\PY{n}{x}\PY{p}{[}\PY{n}{i}\PY{o}{\PYZhy{}}\PY{l+m+mi}{1}\PY{p}{]}\PY{p}{,} \PY{n}{y}\PY{p}{[}\PY{n}{i}\PY{o}{\PYZhy{}}\PY{l+m+mi}{1}\PY{p}{]}\PY{p}{,} \PY{n}{a}\PY{p}{,} \PY{n}{b}\PY{p}{)}
        \PY{n}{perturbations} \PY{o}{=} \PY{n}{J} \PY{o}{@} \PY{n}{perturbations}
        
        \PY{k}{if} \PY{n}{i} \PY{o}{\PYZpc{}} \PY{n}{window\PYZus{}size} \PY{o}{==} \PY{l+m+mi}{0}\PY{p}{:}
            \PY{n}{Q}\PY{p}{,} \PY{n}{R} \PY{o}{=} \PY{n}{np}\PY{o}{.}\PY{n}{linalg}\PY{o}{.}\PY{n}{qr}\PY{p}{(}\PY{n}{perturbations}\PY{p}{)}
            \PY{n}{perturbations} \PY{o}{=} \PY{n}{Q}
            \PY{n}{lyapunov\PYZus{}exponents}\PY{p}{[}\PY{n}{i} \PY{o}{\PYZhy{}} \PY{n}{window\PYZus{}size}\PY{p}{]} \PY{o}{=} \PY{n}{np}\PY{o}{.}\PY{n}{log}\PY{p}{(}\PY{n}{np}\PY{o}{.}\PY{n}{abs}\PY{p}{(}\PY{n}{np}\PY{o}{.}\PY{n}{diag}\PY{p}{(}\PY{n}{R}\PY{p}{)}\PY{p}{)}\PY{p}{)} \PY{o}{/} \PY{n}{window\PYZus{}size}
    
    \PY{k}{return} \PY{n}{lyapunov\PYZus{}exponents}
\end{Verbatim}
\end{tcolorbox}

    \begin{tcolorbox}[breakable, size=fbox, boxrule=1pt, pad at break*=1mm,colback=cellbackground, colframe=cellborder]
\prompt{In}{incolor}{546}{\boxspacing}
\begin{Verbatim}[commandchars=\\\{\}]
\PY{n}{x}\PY{p}{,} \PY{n}{y} \PY{o}{=} \PY{n}{leer\PYZus{}datos}\PY{p}{(}\PY{l+s+s2}{\PYZdq{}}\PY{l+s+s2}{datosHenon.csv}\PY{l+s+s2}{\PYZdq{}}\PY{p}{)}
\PY{n}{exponents} \PY{o}{=} \PY{n}{calculate\PYZus{}lyapunov\PYZus{}exponents}\PY{p}{(}\PY{n}{x}\PY{p}{,} \PY{n}{y}\PY{p}{)}
\PY{n}{plt}\PY{o}{.}\PY{n}{figure}\PY{p}{(}\PY{n}{figsize}\PY{o}{=}\PY{p}{(}\PY{l+m+mi}{12}\PY{p}{,} \PY{l+m+mi}{6}\PY{p}{)}\PY{p}{)}
\PY{n}{plt}\PY{o}{.}\PY{n}{plot}\PY{p}{(}\PY{n}{exponents}\PY{p}{[}\PY{p}{:}\PY{p}{,} \PY{l+m+mi}{0}\PY{p}{]}\PY{p}{,} \PY{n}{label}\PY{o}{=}\PY{l+s+s1}{\PYZsq{}}\PY{l+s+s1}{Exponente de Lyapunov \PYZdl{}}\PY{l+s+s1}{\PYZbs{}}\PY{l+s+s1}{lambda\PYZus{}1\PYZdl{}}\PY{l+s+s1}{\PYZsq{}}\PY{p}{)}
\PY{n}{plt}\PY{o}{.}\PY{n}{plot}\PY{p}{(}\PY{n}{exponents}\PY{p}{[}\PY{p}{:}\PY{p}{,} \PY{l+m+mi}{1}\PY{p}{]}\PY{p}{,} \PY{n}{label}\PY{o}{=}\PY{l+s+s1}{\PYZsq{}}\PY{l+s+s1}{Exponente de Lyapunov \PYZdl{}}\PY{l+s+s1}{\PYZbs{}}\PY{l+s+s1}{lambda\PYZus{}2\PYZdl{}}\PY{l+s+s1}{\PYZsq{}}\PY{p}{)}
\PY{n}{plt}\PY{o}{.}\PY{n}{xlabel}\PY{p}{(}\PY{l+s+s1}{\PYZsq{}}\PY{l+s+s1}{Tiempo (en ventanas de }\PY{l+s+s1}{\PYZsq{}} \PY{o}{+} \PY{n+nb}{str}\PY{p}{(}\PY{n}{window\PYZus{}size}\PY{p}{)} \PY{o}{+} \PY{l+s+s1}{\PYZsq{}}\PY{l+s+s1}{)}\PY{l+s+s1}{\PYZsq{}}\PY{p}{)}
\PY{n}{plt}\PY{o}{.}\PY{n}{ylabel}\PY{p}{(}\PY{l+s+s1}{\PYZsq{}}\PY{l+s+s1}{Valor del Exponente de Lyapunov}\PY{l+s+s1}{\PYZsq{}}\PY{p}{)}
\PY{n}{plt}\PY{o}{.}\PY{n}{title}\PY{p}{(}\PY{l+s+s1}{\PYZsq{}}\PY{l+s+s1}{Espectro de Exponentes de Lyapunov del Atractor de Henón}\PY{l+s+s1}{\PYZsq{}}\PY{p}{)}
\PY{n}{plt}\PY{o}{.}\PY{n}{legend}\PY{p}{(}\PY{p}{)}
\PY{n}{plt}\PY{o}{.}\PY{n}{grid}\PY{p}{(}\PY{k+kc}{True}\PY{p}{)}
\PY{n}{plt}\PY{o}{.}\PY{n}{show}\PY{p}{(}\PY{p}{)}
\end{Verbatim}
\end{tcolorbox}

    \begin{Verbatim}[commandchars=\\\{\}]
/tmp/ipykernel\_17541/3196099903.py:21: RuntimeWarning:

divide by zero encountered in log

    \end{Verbatim}

    \begin{center}
    \adjustimage{max size={0.9\linewidth}{0.9\paperheight}}{analisisCaos2_files/analisisCaos2_86_1.png}
    \end{center}
    { \hspace*{\fill} \\}
    
    \hypertarget{interpretaciuxf3n}{%
\paragraph{Interpretación:}\label{interpretaciuxf3n}}

\begin{itemize}
\tightlist
\item
  **Exponente de Lyapunov \$ \lambda\_1 \$ (Azul):** Este exponente
  muestra una serie de valores positivos alrededor de 0.4 a 0.5, lo que
  indica que el sistema es caótico.
\item
  **Exponente de Lyapunov \$ \lambda\_2 \$ (Naranja):** Este exponente
  se mantiene cerca de 0, indicando estabilidad en otra dirección. En
  los sistemas bidimensionales como el mapa de Henón, es común tener un
  exponente positivo y otro negativo.
\end{itemize}

    \hypertarget{dimensiuxf3n-de-kaplan-yorke}{%
\subsubsection{Dimensión de
Kaplan-Yorke}\label{dimensiuxf3n-de-kaplan-yorke}}

    \begin{tcolorbox}[breakable, size=fbox, boxrule=1pt, pad at break*=1mm,colback=cellbackground, colframe=cellborder]
\prompt{In}{incolor}{547}{\boxspacing}
\begin{Verbatim}[commandchars=\\\{\}]
\PY{k}{def} \PY{n+nf}{calcular\PYZus{}dimension\PYZus{}kaplan\PYZus{}yorke}\PY{p}{(}\PY{n}{exponentes\PYZus{}lyapunov}\PY{p}{)}\PY{p}{:}
    \PY{n}{exponentes\PYZus{}lyapunov} \PY{o}{=} \PY{n}{np}\PY{o}{.}\PY{n}{sort}\PY{p}{(}\PY{n}{exponentes\PYZus{}lyapunov}\PY{p}{)}\PY{p}{[}\PY{p}{:}\PY{p}{:}\PY{o}{\PYZhy{}}\PY{l+m+mi}{1}\PY{p}{]}
    \PY{n}{suma} \PY{o}{=} \PY{l+m+mf}{0.0}
    \PY{n}{j} \PY{o}{=} \PY{l+m+mi}{0}
    \PY{k}{for} \PY{n}{j} \PY{o+ow}{in} \PY{n+nb}{range}\PY{p}{(}\PY{n+nb}{len}\PY{p}{(}\PY{n}{exponentes\PYZus{}lyapunov}\PY{p}{)}\PY{p}{)}\PY{p}{:}
        \PY{n}{suma} \PY{o}{+}\PY{o}{=} \PY{n}{exponentes\PYZus{}lyapunov}\PY{p}{[}\PY{n}{j}\PY{p}{]}
        \PY{k}{if} \PY{n}{suma} \PY{o}{\PYZlt{}} \PY{l+m+mi}{0}\PY{p}{:}
            \PY{n}{j} \PY{o}{\PYZhy{}}\PY{o}{=} \PY{l+m+mi}{1}
            \PY{k}{break}

    \PY{k}{if} \PY{n}{j} \PY{o}{\PYZlt{}} \PY{l+m+mi}{0}\PY{p}{:}
        \PY{k}{return} \PY{l+m+mi}{0}
    \PY{k}{elif} \PY{n}{j} \PY{o}{==} \PY{n+nb}{len}\PY{p}{(}\PY{n}{exponentes\PYZus{}lyapunov}\PY{p}{)} \PY{o}{\PYZhy{}} \PY{l+m+mi}{1}\PY{p}{:}
        \PY{n}{suma\PYZus{}positiva} \PY{o}{=} \PY{n}{np}\PY{o}{.}\PY{n}{sum}\PY{p}{(}\PY{n}{exponentes\PYZus{}lyapunov}\PY{p}{[}\PY{p}{:}\PY{n}{j}\PY{o}{+}\PY{l+m+mi}{1}\PY{p}{]}\PY{p}{)}
        \PY{n}{dimension\PYZus{}kaplan\PYZus{}yorke} \PY{o}{=} \PY{n}{j} \PY{o}{+} \PY{n}{suma\PYZus{}positiva} \PY{o}{/} \PY{n+nb}{abs}\PY{p}{(}\PY{n}{exponentes\PYZus{}lyapunov}\PY{p}{[}\PY{n}{j}\PY{p}{]}\PY{p}{)}
    \PY{k}{else}\PY{p}{:}
        \PY{n}{suma\PYZus{}positiva} \PY{o}{=} \PY{n}{np}\PY{o}{.}\PY{n}{sum}\PY{p}{(}\PY{n}{exponentes\PYZus{}lyapunov}\PY{p}{[}\PY{p}{:}\PY{n}{j}\PY{o}{+}\PY{l+m+mi}{1}\PY{p}{]}\PY{p}{)}
        \PY{n}{dimension\PYZus{}kaplan\PYZus{}yorke} \PY{o}{=} \PY{n}{j} \PY{o}{+} \PY{n}{suma\PYZus{}positiva} \PY{o}{/} \PY{n+nb}{abs}\PY{p}{(}\PY{n}{exponentes\PYZus{}lyapunov}\PY{p}{[}\PY{n}{j}\PY{o}{+}\PY{l+m+mi}{1}\PY{p}{]}\PY{p}{)}
    
    \PY{k}{return} \PY{n}{dimension\PYZus{}kaplan\PYZus{}yorke}
\end{Verbatim}
\end{tcolorbox}

    \begin{tcolorbox}[breakable, size=fbox, boxrule=1pt, pad at break*=1mm,colback=cellbackground, colframe=cellborder]
\prompt{In}{incolor}{548}{\boxspacing}
\begin{Verbatim}[commandchars=\\\{\}]
\PY{n}{averages} \PY{o}{=} \PY{n}{np}\PY{o}{.}\PY{n}{mean}\PY{p}{(}\PY{n}{exponents}\PY{p}{,} \PY{n}{axis}\PY{o}{=}\PY{l+m+mi}{1}\PY{p}{)}
\PY{n}{averages} \PY{o}{=} \PY{n}{averages}\PY{o}{.}\PY{n}{reshape}\PY{p}{(}\PY{o}{\PYZhy{}}\PY{l+m+mi}{1}\PY{p}{,} \PY{l+m+mi}{1}\PY{p}{)}
\PY{n}{kaplan\PYZus{}yorke\PYZus{}dimension} \PY{o}{=} \PY{n}{calcular\PYZus{}dimension\PYZus{}kaplan\PYZus{}yorke}\PY{p}{(}\PY{n}{averages}\PY{p}{)}
\PY{n+nb}{print}\PY{p}{(}\PY{l+s+s2}{\PYZdq{}}\PY{l+s+s2}{Dimensión de Kaplan\PYZhy{}Yorke:}\PY{l+s+s2}{\PYZdq{}}\PY{p}{,} \PY{n}{kaplan\PYZus{}yorke\PYZus{}dimension}\PY{p}{[}\PY{l+m+mi}{0}\PY{p}{]}\PY{p}{)}
\end{Verbatim}
\end{tcolorbox}

    \begin{Verbatim}[commandchars=\\\{\}]
Dimensión de Kaplan-Yorke: 398.0
    \end{Verbatim}

    \hypertarget{interpretaciuxf3n}{%
\paragraph{Interpretación:}\label{interpretaciuxf3n}}

\begin{itemize}
\tightlist
\item
  La dimensión de Kaplan-Yorke calculada es \textbf{398.0}. Este valor
  es extremadamente alto y sugiere que el sistema tiene un
  comportamiento muy complejo y un atractor de alta dimensión.
\end{itemize}

    \hypertarget{dimensiuxf3n-de-grassberger-procaccia}{%
\subsubsection{Dimensión de
Grassberger-Procaccia}\label{dimensiuxf3n-de-grassberger-procaccia}}

    \begin{tcolorbox}[breakable, size=fbox, boxrule=1pt, pad at break*=1mm,colback=cellbackground, colframe=cellborder]
\prompt{In}{incolor}{549}{\boxspacing}
\begin{Verbatim}[commandchars=\\\{\}]
\PY{k}{def} \PY{n+nf}{crear\PYZus{}embedds\PYZus{}multivariado}\PY{p}{(}\PY{n}{datos}\PY{p}{,} \PY{n}{m}\PY{p}{,} \PY{n}{tau}\PY{p}{)}\PY{p}{:}
    \PY{n}{N} \PY{o}{=} \PY{n+nb}{len}\PY{p}{(}\PY{n}{datos}\PY{p}{[}\PY{l+m+mi}{0}\PY{p}{]}\PY{p}{)}
    \PY{n}{embedds} \PY{o}{=} \PY{p}{[}\PY{p}{]}
    \PY{k}{for} \PY{n}{i} \PY{o+ow}{in} \PY{n+nb}{range}\PY{p}{(}\PY{n}{m}\PY{p}{)}\PY{p}{:}
        \PY{n}{embedds}\PY{o}{.}\PY{n}{append}\PY{p}{(}\PY{p}{[}\PY{n}{serie}\PY{p}{[}\PY{n}{i}\PY{p}{:}\PY{n}{N}\PY{o}{\PYZhy{}}\PY{n}{tau}\PY{o}{*}\PY{p}{(}\PY{n}{m}\PY{o}{\PYZhy{}}\PY{l+m+mi}{1}\PY{p}{)}\PY{o}{+}\PY{n}{i}\PY{p}{:}\PY{n}{tau}\PY{p}{]} \PY{k}{for} \PY{n}{serie} \PY{o+ow}{in} \PY{n}{datos}\PY{p}{]}\PY{p}{)}
    \PY{n}{embedds} \PY{o}{=} \PY{n}{np}\PY{o}{.}\PY{n}{hstack}\PY{p}{(}\PY{n}{embedds}\PY{p}{)}
    \PY{k}{return} \PY{n}{embedds}

\PY{k}{def} \PY{n+nf}{calcular\PYZus{}correlacion\PYZus{}integral}\PY{p}{(}\PY{n}{kdtree}\PY{p}{,} \PY{n}{embedds}\PY{p}{,} \PY{n}{r}\PY{p}{)}\PY{p}{:}
    \PY{n}{N} \PY{o}{=} \PY{n+nb}{len}\PY{p}{(}\PY{n}{embedds}\PY{p}{)}
    \PY{n}{C\PYZus{}r} \PY{o}{=} \PY{n}{np}\PY{o}{.}\PY{n}{mean}\PY{p}{(}\PY{p}{[}\PY{n+nb}{len}\PY{p}{(}\PY{n}{kdtree}\PY{o}{.}\PY{n}{query\PYZus{}radius}\PY{p}{(}\PY{n}{point}\PY{o}{.}\PY{n}{reshape}\PY{p}{(}\PY{l+m+mi}{1}\PY{p}{,} \PY{o}{\PYZhy{}}\PY{l+m+mi}{1}\PY{p}{)}\PY{p}{,} \PY{n}{r}\PY{o}{=}\PY{n}{r}\PY{p}{)}\PY{p}{[}\PY{l+m+mi}{0}\PY{p}{]}\PY{p}{)} \PY{k}{for} \PY{n}{point} \PY{o+ow}{in} \PY{n}{embedds}\PY{p}{]}\PY{p}{)} \PY{o}{/} \PY{n}{N}
    \PY{k}{return} \PY{n}{C\PYZus{}r}

\PY{k}{def} \PY{n+nf}{calcular\PYZus{}dimension\PYZus{}gp\PYZus{}multivariado}\PY{p}{(}\PY{n}{datos}\PY{p}{,} \PY{n}{m}\PY{p}{,} \PY{n}{tau}\PY{p}{,} \PY{n}{r\PYZus{}vals}\PY{p}{,} \PY{n}{n\PYZus{}jobs}\PY{o}{=}\PY{o}{\PYZhy{}}\PY{l+m+mi}{1}\PY{p}{)}\PY{p}{:}
    \PY{n}{embedds} \PY{o}{=} \PY{n}{crear\PYZus{}embedds\PYZus{}multivariado}\PY{p}{(}\PY{n}{datos}\PY{p}{,} \PY{n}{m}\PY{p}{,} \PY{n}{tau}\PY{p}{)}
    \PY{n}{kdtree} \PY{o}{=} \PY{n}{KDTree}\PY{p}{(}\PY{n}{embedds}\PY{p}{)}
    \PY{n}{C\PYZus{}r\PYZus{}vals} \PY{o}{=} \PY{n}{Parallel}\PY{p}{(}\PY{n}{n\PYZus{}jobs}\PY{o}{=}\PY{n}{n\PYZus{}jobs}\PY{p}{)}\PY{p}{(}\PY{n}{delayed}\PY{p}{(}\PY{n}{calcular\PYZus{}correlacion\PYZus{}integral}\PY{p}{)}\PY{p}{(}\PY{n}{kdtree}\PY{p}{,} \PY{n}{embedds}\PY{p}{,} \PY{n}{r}\PY{p}{)} \PY{k}{for} \PY{n}{r} \PY{o+ow}{in} \PY{n}{r\PYZus{}vals}\PY{p}{)}
    \PY{n}{log\PYZus{}r} \PY{o}{=} \PY{n}{np}\PY{o}{.}\PY{n}{log}\PY{p}{(}\PY{n}{r\PYZus{}vals}\PY{p}{)}
    \PY{n}{log\PYZus{}C\PYZus{}r} \PY{o}{=} \PY{n}{np}\PY{o}{.}\PY{n}{log}\PY{p}{(}\PY{n}{C\PYZus{}r\PYZus{}vals}\PY{p}{)}
    \PY{n}{slope}\PY{p}{,} \PY{n}{intercept} \PY{o}{=} \PY{n}{np}\PY{o}{.}\PY{n}{polyfit}\PY{p}{(}\PY{n}{log\PYZus{}r}\PY{p}{,} \PY{n}{log\PYZus{}C\PYZus{}r}\PY{p}{,} \PY{l+m+mi}{1}\PY{p}{)}
    \PY{k}{return} \PY{n}{slope}
\end{Verbatim}
\end{tcolorbox}

    \begin{tcolorbox}[breakable, size=fbox, boxrule=1pt, pad at break*=1mm,colback=cellbackground, colframe=cellborder]
\prompt{In}{incolor}{13}{\boxspacing}
\begin{Verbatim}[commandchars=\\\{\}]
\PY{n}{m} \PY{o}{=} \PY{l+m+mi}{10}
\PY{n}{tau} \PY{o}{=} \PY{l+m+mi}{1}
\PY{n}{r\PYZus{}vals} \PY{o}{=} \PY{n}{np}\PY{o}{.}\PY{n}{logspace}\PY{p}{(}\PY{o}{\PYZhy{}}\PY{l+m+mi}{3}\PY{p}{,} \PY{l+m+mi}{0}\PY{p}{,} \PY{l+m+mi}{50}\PY{p}{)} 
\PY{n}{dimension\PYZus{}gp} \PY{o}{=} \PY{n}{calcular\PYZus{}dimension\PYZus{}gp\PYZus{}multivariado}\PY{p}{(}\PY{p}{[}\PY{n}{x}\PY{p}{,} \PY{n}{y}\PY{p}{]}\PY{p}{,} \PY{n}{m}\PY{p}{,} \PY{n}{tau}\PY{p}{,} \PY{n}{r\PYZus{}vals}\PY{p}{,} \PY{n}{n\PYZus{}jobs}\PY{o}{=}\PY{o}{\PYZhy{}}\PY{l+m+mi}{1}\PY{p}{)}
\PY{n+nb}{print}\PY{p}{(}\PY{l+s+s2}{\PYZdq{}}\PY{l+s+s2}{La dimensión de Grassberger\PYZhy{}Procaccia es:}\PY{l+s+s2}{\PYZdq{}}\PY{p}{,} \PY{n}{dimension\PYZus{}gp}\PY{p}{)}
\end{Verbatim}
\end{tcolorbox}

    \begin{Verbatim}[commandchars=\\\{\}]
La dimensión de Grassberger-Procaccia es: 1.63782763568
    \end{Verbatim}

    \hypertarget{interpretaciuxf3n}{%
\paragraph{Interpretación:}\label{interpretaciuxf3n}}

\begin{itemize}
\tightlist
\item
  La dimensión de Grassberger-Procaccia calculada es
  \textbf{1.63782763568}. Este valor sugiere que el atractor del sistema
  tiene una dimensión fractal intermedia, lo cual puede indicar un
  comportamiento caótico complejo pero en una dimensión razonable para
  sistemas caóticos típicos.
\end{itemize}

    \hypertarget{atractor-de-rossler}{%
\section{Atractor de Rossler}\label{atractor-de-rossler}}

    \begin{tcolorbox}[breakable, size=fbox, boxrule=1pt, pad at break*=1mm,colback=cellbackground, colframe=cellborder]
\prompt{In}{incolor}{10}{\boxspacing}
\begin{Verbatim}[commandchars=\\\{\}]
\PY{n}{valores\PYZus{}x} \PY{o}{=} \PY{n}{leer\PYZus{}col\PYZus{}csv}\PY{p}{(}\PY{l+s+s2}{\PYZdq{}}\PY{l+s+s2}{datosRossler.csv}\PY{l+s+s2}{\PYZdq{}}\PY{p}{,} \PY{l+s+s2}{\PYZdq{}}\PY{l+s+s2}{Valores x}\PY{l+s+s2}{\PYZdq{}}\PY{p}{)}
\PY{n}{valores\PYZus{}y} \PY{o}{=} \PY{n}{leer\PYZus{}col\PYZus{}csv}\PY{p}{(}\PY{l+s+s2}{\PYZdq{}}\PY{l+s+s2}{datosRossler.csv}\PY{l+s+s2}{\PYZdq{}}\PY{p}{,} \PY{l+s+s2}{\PYZdq{}}\PY{l+s+s2}{Valores y}\PY{l+s+s2}{\PYZdq{}}\PY{p}{)}
\PY{n}{valores\PYZus{}z} \PY{o}{=} \PY{n}{leer\PYZus{}col\PYZus{}csv}\PY{p}{(}\PY{l+s+s2}{\PYZdq{}}\PY{l+s+s2}{datosRossler.csv}\PY{l+s+s2}{\PYZdq{}}\PY{p}{,} \PY{l+s+s2}{\PYZdq{}}\PY{l+s+s2}{Valores z}\PY{l+s+s2}{\PYZdq{}}\PY{p}{)}
\PY{n}{valores\PYZus{}k} \PY{o}{=} \PY{n+nb}{range}\PY{p}{(}\PY{l+m+mi}{1}\PY{p}{,} \PY{n+nb}{len}\PY{p}{(}\PY{n}{valores\PYZus{}x}\PY{p}{)}\PY{o}{+}\PY{l+m+mi}{1}\PY{p}{)}
\PY{n}{graficar\PYZus{}3d}\PY{p}{(}\PY{n}{valores\PYZus{}x}\PY{p}{,} \PY{n}{valores\PYZus{}y}\PY{p}{,} \PY{n}{valores\PYZus{}z}\PY{p}{,} \PY{n}{titulo}\PY{o}{=}\PY{l+s+s2}{\PYZdq{}}\PY{l+s+s2}{Atractor de Rossler}\PY{l+s+s2}{\PYZdq{}}\PY{p}{)}
\end{Verbatim}
\end{tcolorbox}

    
    
    \hypertarget{probabiluxedstico}{%
\subsection{Probabilístico}\label{probabiluxedstico}}

    \begin{tcolorbox}[breakable, size=fbox, boxrule=1pt, pad at break*=1mm,colback=cellbackground, colframe=cellborder]
\prompt{In}{incolor}{ }{\boxspacing}
\begin{Verbatim}[commandchars=\\\{\}]
\PY{n}{rossler} \PY{o}{=} \PY{n}{DistribucionProbabilidad}\PY{p}{(}\PY{n}{valores\PYZus{}x}\PY{p}{,} \PY{n}{valores\PYZus{}y}\PY{p}{,} \PY{n}{valores\PYZus{}z}\PY{p}{)}
\PY{n}{rossler}\PY{o}{.}\PY{n}{mostrar\PYZus{}metricas}\PY{p}{(}\PY{p}{)}
\PY{n}{plotear\PYZus{}hist}\PY{p}{(}\PY{n}{valores\PYZus{}x}\PY{p}{,} \PY{l+s+s2}{\PYZdq{}}\PY{l+s+s2}{Histograma de Rossler X}\PY{l+s+s2}{\PYZdq{}}\PY{p}{,} \PY{l+s+s2}{\PYZdq{}}\PY{l+s+s2}{Valor}\PY{l+s+s2}{\PYZdq{}}\PY{p}{,} \PY{l+s+s2}{\PYZdq{}}\PY{l+s+s2}{Frecuencia}\PY{l+s+s2}{\PYZdq{}}\PY{p}{,} \PY{l+s+s2}{\PYZdq{}}\PY{l+s+s2}{sturges}\PY{l+s+s2}{\PYZdq{}}\PY{p}{)}
\PY{n}{plotear\PYZus{}hist}\PY{p}{(}\PY{n}{valores\PYZus{}y}\PY{p}{,} \PY{l+s+s2}{\PYZdq{}}\PY{l+s+s2}{Histograma de Rossler Y}\PY{l+s+s2}{\PYZdq{}}\PY{p}{,} \PY{l+s+s2}{\PYZdq{}}\PY{l+s+s2}{Valor}\PY{l+s+s2}{\PYZdq{}}\PY{p}{,} \PY{l+s+s2}{\PYZdq{}}\PY{l+s+s2}{Frecuencia}\PY{l+s+s2}{\PYZdq{}}\PY{p}{,} \PY{l+s+s2}{\PYZdq{}}\PY{l+s+s2}{sturges}\PY{l+s+s2}{\PYZdq{}}\PY{p}{)}
\PY{n}{plotear\PYZus{}hist}\PY{p}{(}\PY{n}{valores\PYZus{}z}\PY{p}{,} \PY{l+s+s2}{\PYZdq{}}\PY{l+s+s2}{Histograma de Rossler Z}\PY{l+s+s2}{\PYZdq{}}\PY{p}{,} \PY{l+s+s2}{\PYZdq{}}\PY{l+s+s2}{Valor}\PY{l+s+s2}{\PYZdq{}}\PY{p}{,} \PY{l+s+s2}{\PYZdq{}}\PY{l+s+s2}{Frecuencia}\PY{l+s+s2}{\PYZdq{}}\PY{p}{,} \PY{l+s+s2}{\PYZdq{}}\PY{l+s+s2}{sturges}\PY{l+s+s2}{\PYZdq{}}\PY{p}{)}
\PY{n}{graficar}\PY{p}{(}\PY{n}{valores\PYZus{}x}\PY{p}{,} \PY{n}{valores\PYZus{}k}\PY{p}{,} \PY{n}{width}\PY{o}{=}\PY{l+m+mi}{10}\PY{p}{,} \PY{n}{height}\PY{o}{=}\PY{l+m+mi}{7} \PY{p}{,}\PY{n}{titulo}\PY{o}{=}\PY{l+s+s2}{\PYZdq{}}\PY{l+s+s2}{Variable X vs k}\PY{l+s+s2}{\PYZdq{}}\PY{p}{)}
\PY{n}{graficar}\PY{p}{(}\PY{n}{valores\PYZus{}y}\PY{p}{,} \PY{n}{valores\PYZus{}k}\PY{p}{,} \PY{n}{width}\PY{o}{=}\PY{l+m+mi}{10}\PY{p}{,} \PY{n}{height}\PY{o}{=}\PY{l+m+mi}{7} \PY{p}{,}\PY{n}{titulo}\PY{o}{=}\PY{l+s+s2}{\PYZdq{}}\PY{l+s+s2}{Variable Y vs k}\PY{l+s+s2}{\PYZdq{}}\PY{p}{)}
\PY{n}{graficar}\PY{p}{(}\PY{n}{valores\PYZus{}z}\PY{p}{,} \PY{n}{valores\PYZus{}k}\PY{p}{,} \PY{n}{width}\PY{o}{=}\PY{l+m+mi}{10}\PY{p}{,} \PY{n}{height}\PY{o}{=}\PY{l+m+mi}{7} \PY{p}{,}\PY{n}{titulo}\PY{o}{=}\PY{l+s+s2}{\PYZdq{}}\PY{l+s+s2}{Variable Z vs k}\PY{l+s+s2}{\PYZdq{}}\PY{p}{)}
\end{Verbatim}
\end{tcolorbox}

    \begin{Verbatim}[commandchars=\\\{\}]
--- Métricas para Variable 1 ---
Media: 0.18485470681822816
Mediana: -0.22063136624253193
Moda: -9.28112168747234
Media Geométrica: nan
Rango: 21.07306785563107
Desviación Estándar: 4.998003092482866
Varianza: 24.980034912468287
Asimetría: 0.20554979315632138
Curtosis: -0.5972925304986041
Entropia: 11.512925464970223
Coeficiente de Variación: 27.03746730883902

--- Métricas para Variable 2 ---
Media: -0.908295039420532
Mediana: -0.8376306330690166
Moda: -11.090572417946863
Media Geométrica: nan
Rango: 19.023502319928667
Desviación Estándar: 4.742096688026963
Varianza: 22.48748099859629
Asimetría: -0.17152034988291723
Curtosis: -0.6646172866952118
Entropia: 11.512925464970223
Coeficiente de Variación: -5.220877008259666

--- Métricas para Variable 3 ---
Media: 0.9022493082024416
Mediana: 0.04085462120151002
Moda: 0.013365904330161932
Media Geométrica: 0.07601411137144386
Rango: 26.4706830221461
Desviación Estándar: 3.4708531990836273
Varianza: 12.046821929589049
Asimetría: 5.213854069701677
Curtosis: 27.985554192873423
Entropia: 11.512925464970223
Coeficiente de Variación: 3.846889288281789

Matriz de Correlación:
[[ 1.         -0.19435777  0.27536813]
 [-0.19435777  1.          0.09060812]
 [ 0.27536813  0.09060812  1.        ]]
    \end{Verbatim}

    \begin{Verbatim}[commandchars=\\\{\}]
/home/rodrigo/.local/lib/python3.10/site-packages/scipy/stats/\_stats\_py.py:197:
RuntimeWarning:

invalid value encountered in log

    \end{Verbatim}

    \begin{center}
    \adjustimage{max size={0.9\linewidth}{0.9\paperheight}}{analisisCaos2_files/analisisCaos2_99_2.png}
    \end{center}
    { \hspace*{\fill} \\}
    
    \begin{center}
    \adjustimage{max size={0.9\linewidth}{0.9\paperheight}}{analisisCaos2_files/analisisCaos2_99_3.png}
    \end{center}
    { \hspace*{\fill} \\}
    
    \begin{center}
    \adjustimage{max size={0.9\linewidth}{0.9\paperheight}}{analisisCaos2_files/analisisCaos2_99_4.png}
    \end{center}
    { \hspace*{\fill} \\}
    
    \begin{center}
    \adjustimage{max size={0.9\linewidth}{0.9\paperheight}}{analisisCaos2_files/analisisCaos2_99_5.png}
    \end{center}
    { \hspace*{\fill} \\}
    
    \begin{center}
    \adjustimage{max size={0.9\linewidth}{0.9\paperheight}}{analisisCaos2_files/analisisCaos2_99_6.png}
    \end{center}
    { \hspace*{\fill} \\}
    
    \begin{center}
    \adjustimage{max size={0.9\linewidth}{0.9\paperheight}}{analisisCaos2_files/analisisCaos2_99_7.png}
    \end{center}
    { \hspace*{\fill} \\}
    
    \begin{center}
    \adjustimage{max size={0.9\linewidth}{0.9\paperheight}}{analisisCaos2_files/analisisCaos2_99_8.png}
    \end{center}
    { \hspace*{\fill} \\}
    
    \hypertarget{anuxe1lisis-de-muxe9tricas-estaduxedsticas-para-el-modelo-del-atractor-de-rossler}{%
\subsection{Análisis de Métricas Estadísticas para el Modelo del
Atractor de
Rossler}\label{anuxe1lisis-de-muxe9tricas-estaduxedsticas-para-el-modelo-del-atractor-de-rossler}}

\hypertarget{anuxe1lisis-de-muxe9tricas-estaduxedsticas}{%
\subsubsection{Análisis de Métricas
Estadísticas}\label{anuxe1lisis-de-muxe9tricas-estaduxedsticas}}

\begin{itemize}
\item
  \textbf{Rango}: Los rangos amplios en todas las variables (Variable X:
  21.07, Variable Y: 19.02, Variable Z: 26.47) indican que cada variable
  explora un amplio espectro de valores. Esto es característico de
  sistemas caóticos donde los estados pueden cambiar dramáticamente a lo
  largo del tiempo.
\item
  \textbf{Desviación Estándar y Varianza}: Altas en todas las variables
  (por ejemplo, Variable X: Desviación Estándar de 5.00, Varianza de
  25.02), reflejan la considerable dispersión y la alta variabilidad de
  los datos. Esto subraya la naturaleza impredecible y sensible a las
  condiciones iniciales del sistema.
\item
  \textbf{Curtosis y Asimetría}: La curtosis elevada en la Variable Z
  (27.85) junto con una asimetría significativa (5.21) indica una
  distribución con colas pesadas y un pico agudo. Esto sugiere la
  presencia de comportamientos extremos y transiciones abruptas típicas
  en dinámicas caóticas.
\item
  \textbf{Entropía (11.9184 para todas las variables)}: Una entropía
  consistentemente alta a través de las variables muestra una gran
  diversidad en los estados del sistema, lo cual es un indicador de
  complejidad y caos.
\end{itemize}

\hypertarget{anuxe1lisis-de-correlaciuxf3n-y-gruxe1fica-de-dispersiuxf3n}{%
\subsubsection{Análisis de Correlación y Gráfica de
Dispersión}\label{anuxe1lisis-de-correlaciuxf3n-y-gruxe1fica-de-dispersiuxf3n}}

\begin{itemize}
\tightlist
\item
  \textbf{Matriz de Correlación}: La correlación entre las variables
  muestra valores bajos y mixtos (por ejemplo, 0.274 entre la Variable 1
  y la Variable 3), indicando que no hay una fuerte dependencia lineal
  entre ellas. Esto es esperado en sistemas caóticos donde las
  relaciones no son simples ni directamente proporcionales.
\end{itemize}

\hypertarget{conclusiuxf3n}{%
\subsubsection{Conclusión}\label{conclusiuxf3n}}

Las métricas estadísticas y las visualizaciones del atractor de Rossler
resaltan un sistema con extrema variabilidad y complejidad. La amplia
gama de valores explorados por las variables, junto con altas entropías
y patrones de dispersión característicos, confirman la naturaleza
caótica del atractor. Este análisis proporciona una comprensión profunda
de cómo las variables interactúan y evolucionan en el tiempo dentro de
este sistema dinámico.

    \hypertarget{cauxf3tico}{%
\subsection{Caótico}\label{cauxf3tico}}

    \hypertarget{exponentes-de-laypounov}{%
\subsubsection{Exponentes de Laypounov}\label{exponentes-de-laypounov}}

    \begin{tcolorbox}[breakable, size=fbox, boxrule=1pt, pad at break*=1mm,colback=cellbackground, colframe=cellborder]
\prompt{In}{incolor}{ }{\boxspacing}
\begin{Verbatim}[commandchars=\\\{\}]
\PY{k}{def} \PY{n+nf}{leer\PYZus{}datos\PYZus{}rossler}\PY{p}{(}\PY{n}{csv\PYZus{}path}\PY{p}{)}\PY{p}{:}
    \PY{n}{df} \PY{o}{=} \PY{n}{pd}\PY{o}{.}\PY{n}{read\PYZus{}csv}\PY{p}{(}\PY{n}{csv\PYZus{}path}\PY{p}{)}
    \PY{k}{return} \PY{n}{df}\PY{p}{[}\PY{l+s+s1}{\PYZsq{}}\PY{l+s+s1}{Valores x}\PY{l+s+s1}{\PYZsq{}}\PY{p}{]}\PY{o}{.}\PY{n}{values}\PY{p}{,} \PY{n}{df}\PY{p}{[}\PY{l+s+s1}{\PYZsq{}}\PY{l+s+s1}{Valores y}\PY{l+s+s1}{\PYZsq{}}\PY{p}{]}\PY{o}{.}\PY{n}{values}\PY{p}{,} \PY{n}{df}\PY{p}{[}\PY{l+s+s1}{\PYZsq{}}\PY{l+s+s1}{Valores z}\PY{l+s+s1}{\PYZsq{}}\PY{p}{]}\PY{o}{.}\PY{n}{values}

\PY{k}{def} \PY{n+nf}{jacobian\PYZus{}rossler}\PY{p}{(}\PY{n}{x}\PY{p}{,} \PY{n}{y}\PY{p}{,} \PY{n}{z}\PY{p}{,} \PY{n}{a}\PY{o}{=}\PY{l+m+mf}{0.1}\PY{p}{,} \PY{n}{b}\PY{o}{=}\PY{l+m+mf}{0.1}\PY{p}{,} \PY{n}{c}\PY{o}{=}\PY{l+m+mi}{14}\PY{p}{)}\PY{p}{:}
    \PY{k}{return} \PY{n}{np}\PY{o}{.}\PY{n}{array}\PY{p}{(}\PY{p}{[}
        \PY{p}{[}\PY{l+m+mi}{0}\PY{p}{,} \PY{o}{\PYZhy{}}\PY{l+m+mi}{1}\PY{p}{,} \PY{o}{\PYZhy{}}\PY{l+m+mi}{1}\PY{p}{]}\PY{p}{,}
        \PY{p}{[}\PY{l+m+mi}{1}\PY{p}{,} \PY{n}{a}\PY{p}{,} \PY{l+m+mi}{0}\PY{p}{]}\PY{p}{,}
        \PY{p}{[}\PY{n}{z}\PY{p}{,} \PY{l+m+mi}{0}\PY{p}{,} \PY{n}{x} \PY{o}{\PYZhy{}} \PY{n}{c}\PY{p}{]}
    \PY{p}{]}\PY{p}{)}

\PY{k}{def} \PY{n+nf}{calculate\PYZus{}lyapunov\PYZus{}exponents\PYZus{}rossler}\PY{p}{(}\PY{n}{x}\PY{p}{,} \PY{n}{y}\PY{p}{,} \PY{n}{z}\PY{p}{,} \PY{n}{a}\PY{o}{=}\PY{l+m+mf}{0.1}\PY{p}{,} \PY{n}{b}\PY{o}{=}\PY{l+m+mf}{0.1}\PY{p}{,} \PY{n}{c}\PY{o}{=}\PY{l+m+mi}{14}\PY{p}{,} \PY{n}{window\PYZus{}size}\PY{o}{=}\PY{l+m+mi}{100}\PY{p}{)}\PY{p}{:}
    \PY{n}{n} \PY{o}{=} \PY{n+nb}{len}\PY{p}{(}\PY{n}{x}\PY{p}{)}
    \PY{n}{perturbations} \PY{o}{=} \PY{n}{np}\PY{o}{.}\PY{n}{eye}\PY{p}{(}\PY{l+m+mi}{3}\PY{p}{)}
    \PY{n}{lyapunov\PYZus{}exponents} \PY{o}{=} \PY{n}{np}\PY{o}{.}\PY{n}{zeros}\PY{p}{(}\PY{p}{(}\PY{n}{n} \PY{o}{\PYZhy{}} \PY{n}{window\PYZus{}size}\PY{p}{,} \PY{l+m+mi}{3}\PY{p}{)}\PY{p}{)}
    
    \PY{k}{for} \PY{n}{i} \PY{o+ow}{in} \PY{n+nb}{range}\PY{p}{(}\PY{n}{window\PYZus{}size}\PY{p}{,} \PY{n}{n}\PY{p}{)}\PY{p}{:}
        \PY{n}{J} \PY{o}{=} \PY{n}{jacobian\PYZus{}rossler}\PY{p}{(}\PY{n}{x}\PY{p}{[}\PY{n}{i}\PY{o}{\PYZhy{}}\PY{l+m+mi}{1}\PY{p}{]}\PY{p}{,} \PY{n}{y}\PY{p}{[}\PY{n}{i}\PY{o}{\PYZhy{}}\PY{l+m+mi}{1}\PY{p}{]}\PY{p}{,} \PY{n}{z}\PY{p}{[}\PY{n}{i}\PY{o}{\PYZhy{}}\PY{l+m+mi}{1}\PY{p}{]}\PY{p}{,} \PY{n}{a}\PY{p}{,} \PY{n}{b}\PY{p}{,} \PY{n}{c}\PY{p}{)}
        \PY{n}{perturbations} \PY{o}{=} \PY{n}{J} \PY{o}{@} \PY{n}{perturbations}
        
        \PY{k}{if} \PY{n}{i} \PY{o}{\PYZpc{}} \PY{n}{window\PYZus{}size} \PY{o}{==} \PY{l+m+mi}{0}\PY{p}{:}
            \PY{n}{Q}\PY{p}{,} \PY{n}{R} \PY{o}{=} \PY{n}{np}\PY{o}{.}\PY{n}{linalg}\PY{o}{.}\PY{n}{qr}\PY{p}{(}\PY{n}{perturbations}\PY{p}{)}
            \PY{n}{perturbations} \PY{o}{=} \PY{n}{Q}
            \PY{n}{lyapunov\PYZus{}exponents}\PY{p}{[}\PY{n}{i} \PY{o}{\PYZhy{}} \PY{n}{window\PYZus{}size}\PY{p}{]} \PY{o}{=} \PY{n}{np}\PY{o}{.}\PY{n}{log}\PY{p}{(}\PY{n}{np}\PY{o}{.}\PY{n}{abs}\PY{p}{(}\PY{n}{np}\PY{o}{.}\PY{n}{diag}\PY{p}{(}\PY{n}{R}\PY{p}{)}\PY{p}{)}\PY{p}{)} \PY{o}{/} \PY{n}{window\PYZus{}size}
    
    \PY{k}{return} \PY{n}{lyapunov\PYZus{}exponents}
\end{Verbatim}
\end{tcolorbox}

    \begin{tcolorbox}[breakable, size=fbox, boxrule=1pt, pad at break*=1mm,colback=cellbackground, colframe=cellborder]
\prompt{In}{incolor}{ }{\boxspacing}
\begin{Verbatim}[commandchars=\\\{\}]
\PY{n}{x}\PY{p}{,} \PY{n}{y}\PY{p}{,} \PY{n}{z} \PY{o}{=} \PY{n}{leer\PYZus{}datos\PYZus{}rossler}\PY{p}{(}\PY{l+s+s2}{\PYZdq{}}\PY{l+s+s2}{datosRossler.csv}\PY{l+s+s2}{\PYZdq{}}\PY{p}{)}
\PY{n}{exponents} \PY{o}{=} \PY{n}{calculate\PYZus{}lyapunov\PYZus{}exponents\PYZus{}rossler}\PY{p}{(}\PY{n}{x}\PY{p}{,} \PY{n}{y}\PY{p}{,} \PY{n}{z}\PY{p}{)}
\PY{n}{plt}\PY{o}{.}\PY{n}{figure}\PY{p}{(}\PY{n}{figsize}\PY{o}{=}\PY{p}{(}\PY{l+m+mi}{12}\PY{p}{,} \PY{l+m+mi}{6}\PY{p}{)}\PY{p}{)}
\PY{n}{plt}\PY{o}{.}\PY{n}{plot}\PY{p}{(}\PY{n}{exponents}\PY{p}{[}\PY{p}{:}\PY{p}{,} \PY{l+m+mi}{0}\PY{p}{]}\PY{p}{,} \PY{n}{label}\PY{o}{=}\PY{l+s+s1}{\PYZsq{}}\PY{l+s+s1}{Exponente de Lyapunov \PYZdl{}}\PY{l+s+s1}{\PYZbs{}}\PY{l+s+s1}{lambda\PYZus{}1\PYZdl{}}\PY{l+s+s1}{\PYZsq{}}\PY{p}{)}
\PY{n}{plt}\PY{o}{.}\PY{n}{plot}\PY{p}{(}\PY{n}{exponents}\PY{p}{[}\PY{p}{:}\PY{p}{,} \PY{l+m+mi}{1}\PY{p}{]}\PY{p}{,} \PY{n}{label}\PY{o}{=}\PY{l+s+s1}{\PYZsq{}}\PY{l+s+s1}{Exponente de Lyapunov \PYZdl{}}\PY{l+s+s1}{\PYZbs{}}\PY{l+s+s1}{lambda\PYZus{}2\PYZdl{}}\PY{l+s+s1}{\PYZsq{}}\PY{p}{)}
\PY{n}{plt}\PY{o}{.}\PY{n}{plot}\PY{p}{(}\PY{n}{exponents}\PY{p}{[}\PY{p}{:}\PY{p}{,} \PY{l+m+mi}{2}\PY{p}{]}\PY{p}{,} \PY{n}{label}\PY{o}{=}\PY{l+s+s1}{\PYZsq{}}\PY{l+s+s1}{Exponente de Lyapunov \PYZdl{}}\PY{l+s+s1}{\PYZbs{}}\PY{l+s+s1}{lambda\PYZus{}3\PYZdl{}}\PY{l+s+s1}{\PYZsq{}}\PY{p}{)}
\PY{n}{plt}\PY{o}{.}\PY{n}{xlabel}\PY{p}{(}\PY{l+s+s1}{\PYZsq{}}\PY{l+s+s1}{Tiempo (en ventanas de }\PY{l+s+s1}{\PYZsq{}} \PY{o}{+} \PY{n+nb}{str}\PY{p}{(}\PY{n}{window\PYZus{}size}\PY{p}{)} \PY{o}{+} \PY{l+s+s1}{\PYZsq{}}\PY{l+s+s1}{)}\PY{l+s+s1}{\PYZsq{}}\PY{p}{)}
\PY{n}{plt}\PY{o}{.}\PY{n}{ylabel}\PY{p}{(}\PY{l+s+s1}{\PYZsq{}}\PY{l+s+s1}{Valor del Exponente de Lyapunov}\PY{l+s+s1}{\PYZsq{}}\PY{p}{)}
\PY{n}{plt}\PY{o}{.}\PY{n}{title}\PY{p}{(}\PY{l+s+s1}{\PYZsq{}}\PY{l+s+s1}{Espectro de Exponentes de Lyapunov del Atractor de Rössler}\PY{l+s+s1}{\PYZsq{}}\PY{p}{)}
\PY{n}{plt}\PY{o}{.}\PY{n}{legend}\PY{p}{(}\PY{p}{)}
\PY{n}{plt}\PY{o}{.}\PY{n}{grid}\PY{p}{(}\PY{k+kc}{True}\PY{p}{)}
\PY{n}{plt}\PY{o}{.}\PY{n}{show}\PY{p}{(}\PY{p}{)}
\end{Verbatim}
\end{tcolorbox}

    \begin{Verbatim}[commandchars=\\\{\}]
/tmp/ipykernel\_17541/1592744035.py:24: RuntimeWarning:

divide by zero encountered in log

    \end{Verbatim}

    \begin{center}
    \adjustimage{max size={0.9\linewidth}{0.9\paperheight}}{analisisCaos2_files/analisisCaos2_104_1.png}
    \end{center}
    { \hspace*{\fill} \\}
    
    \hypertarget{interpretaciuxf3n}{%
\subsubsection{Interpretación:}\label{interpretaciuxf3n}}

\begin{itemize}
\tightlist
\item
  **Exponente de Lyapunov \$ \lambda\_1 \$ (Azul):** Este exponente
  muestra una serie de valores positivos alrededor de 2.5 a 3.0, lo que
  indica que el sistema es altamente caótico.
\item
  **Exponente de Lyapunov \$ \lambda\_2 \$ (Naranja):** Este exponente
  se mantiene en torno a 1.5, lo que sugiere una dinámica caótica
  adicional en otra dirección.
\item
  **Exponente de Lyapunov \$ \lambda\_3 \$ (Verde):** Este exponente
  está alrededor de 1.0, lo cual es consistente con la presencia de caos
  en múltiples dimensiones en el sistema.
\end{itemize}

    \hypertarget{dimensiuxf3n-de-kaplan-yorke}{%
\subsubsection{Dimensión de
Kaplan-Yorke}\label{dimensiuxf3n-de-kaplan-yorke}}

    \begin{tcolorbox}[breakable, size=fbox, boxrule=1pt, pad at break*=1mm,colback=cellbackground, colframe=cellborder]
\prompt{In}{incolor}{ }{\boxspacing}
\begin{Verbatim}[commandchars=\\\{\}]
\PY{n}{averages} \PY{o}{=} \PY{n}{np}\PY{o}{.}\PY{n}{mean}\PY{p}{(}\PY{n}{exponents}\PY{p}{,} \PY{n}{axis}\PY{o}{=}\PY{l+m+mi}{1}\PY{p}{)}
\PY{n}{averages} \PY{o}{=} \PY{n}{averages}\PY{o}{.}\PY{n}{reshape}\PY{p}{(}\PY{o}{\PYZhy{}}\PY{l+m+mi}{1}\PY{p}{,} \PY{l+m+mi}{1}\PY{p}{)}
\PY{n}{kaplan\PYZus{}yorke\PYZus{}dimension} \PY{o}{=} \PY{n}{calcular\PYZus{}dimension\PYZus{}kaplan\PYZus{}yorke}\PY{p}{(}\PY{n}{averages}\PY{p}{)}
\PY{n+nb}{print}\PY{p}{(}\PY{l+s+s2}{\PYZdq{}}\PY{l+s+s2}{Dimensión de Kaplan\PYZhy{}Yorke:}\PY{l+s+s2}{\PYZdq{}}\PY{p}{,} \PY{n}{kaplan\PYZus{}yorke\PYZus{}dimension}\PY{p}{)}
\end{Verbatim}
\end{tcolorbox}

    \begin{Verbatim}[commandchars=\\\{\}]
Dimensión de Kaplan-Yorke: [98.]
    \end{Verbatim}

    \hypertarget{interpretaciuxf3n}{%
\paragraph{Interpretación:}\label{interpretaciuxf3n}}

\begin{itemize}
\tightlist
\item
  La dimensión de Kaplan-Yorke calculada es \textbf{98.0}. Este valor es
  bastante alto y sugiere que el sistema tiene un comportamiento
  complejo y un atractor de alta dimensión.
\end{itemize}

    \hypertarget{dimensiuxf3n-grassberger-procaccia}{%
\subsubsection{Dimensión
Grassberger-Procaccia}\label{dimensiuxf3n-grassberger-procaccia}}

    \begin{tcolorbox}[breakable, size=fbox, boxrule=1pt, pad at break*=1mm,colback=cellbackground, colframe=cellborder]
\prompt{In}{incolor}{14}{\boxspacing}
\begin{Verbatim}[commandchars=\\\{\}]
\PY{n}{m} \PY{o}{=} \PY{l+m+mi}{10}
\PY{n}{tau} \PY{o}{=} \PY{l+m+mi}{1}
\PY{n}{r\PYZus{}vals} \PY{o}{=} \PY{n}{np}\PY{o}{.}\PY{n}{logspace}\PY{p}{(}\PY{o}{\PYZhy{}}\PY{l+m+mi}{3}\PY{p}{,} \PY{l+m+mi}{0}\PY{p}{,} \PY{l+m+mi}{50}\PY{p}{)} 
\PY{n}{dimension\PYZus{}gp} \PY{o}{=} \PY{n}{calcular\PYZus{}dimension\PYZus{}gp\PYZus{}multivariado}\PY{p}{(}\PY{p}{[}\PY{n}{x}\PY{p}{,} \PY{n}{y}\PY{p}{,} \PY{n}{z}\PY{p}{]}\PY{p}{,} \PY{n}{m}\PY{p}{,} \PY{n}{tau}\PY{p}{,} \PY{n}{r\PYZus{}vals}\PY{p}{,} \PY{n}{n\PYZus{}jobs}\PY{o}{=}\PY{o}{\PYZhy{}}\PY{l+m+mi}{1}\PY{p}{)}
\PY{n+nb}{print}\PY{p}{(}\PY{l+s+s2}{\PYZdq{}}\PY{l+s+s2}{La dimensión de Grassberger\PYZhy{}Procaccia es:}\PY{l+s+s2}{\PYZdq{}}\PY{p}{,} \PY{n}{dimension\PYZus{}gp}\PY{p}{)}
\end{Verbatim}
\end{tcolorbox}

    \begin{Verbatim}[commandchars=\\\{\}]
La dimensión de Grassberger-Procaccia es: 1.2674972467934673
    \end{Verbatim}

    \hypertarget{interpretaciuxf3n}{%
\paragraph{Interpretación:}\label{interpretaciuxf3n}}

\begin{itemize}
\tightlist
\item
  La dimensión de Grassberger-Procaccia calculada es
  \textbf{1.2674972467934673}. Este valor sugiere que el atractor del
  sistema tiene una dimensión fractal baja, esto puede indicar que el
  atractor tiene una estructura fractal compleja pero en una dimensión
  efectiva más baja.
\end{itemize}

    \hypertarget{atractor-de-lorentz}{%
\subsubsection{Atractor de Lorentz}\label{atractor-de-lorentz}}

    \begin{tcolorbox}[breakable, size=fbox, boxrule=1pt, pad at break*=1mm,colback=cellbackground, colframe=cellborder]
\prompt{In}{incolor}{ }{\boxspacing}
\begin{Verbatim}[commandchars=\\\{\}]
\PY{n}{valores\PYZus{}x} \PY{o}{=} \PY{n}{leer\PYZus{}col\PYZus{}csv}\PY{p}{(}\PY{l+s+s2}{\PYZdq{}}\PY{l+s+s2}{datosLorentz.csv}\PY{l+s+s2}{\PYZdq{}}\PY{p}{,} \PY{l+s+s2}{\PYZdq{}}\PY{l+s+s2}{Valores x}\PY{l+s+s2}{\PYZdq{}}\PY{p}{)}
\PY{n}{valores\PYZus{}y} \PY{o}{=} \PY{n}{leer\PYZus{}col\PYZus{}csv}\PY{p}{(}\PY{l+s+s2}{\PYZdq{}}\PY{l+s+s2}{datosLorentz.csv}\PY{l+s+s2}{\PYZdq{}}\PY{p}{,} \PY{l+s+s2}{\PYZdq{}}\PY{l+s+s2}{Valores y}\PY{l+s+s2}{\PYZdq{}}\PY{p}{)}
\PY{n}{valores\PYZus{}z} \PY{o}{=} \PY{n}{leer\PYZus{}col\PYZus{}csv}\PY{p}{(}\PY{l+s+s2}{\PYZdq{}}\PY{l+s+s2}{datosLorentz.csv}\PY{l+s+s2}{\PYZdq{}}\PY{p}{,} \PY{l+s+s2}{\PYZdq{}}\PY{l+s+s2}{Valores z}\PY{l+s+s2}{\PYZdq{}}\PY{p}{)}
\PY{n}{valores\PYZus{}k} \PY{o}{=} \PY{n+nb}{range}\PY{p}{(}\PY{l+m+mi}{1}\PY{p}{,} \PY{n+nb}{len}\PY{p}{(}\PY{n}{valores\PYZus{}x}\PY{p}{)}\PY{o}{+}\PY{l+m+mi}{1}\PY{p}{)}
\PY{n}{graficar\PYZus{}3d}\PY{p}{(}\PY{n}{valores\PYZus{}x}\PY{p}{,} \PY{n}{valores\PYZus{}y}\PY{p}{,} \PY{n}{valores\PYZus{}z}\PY{p}{,} \PY{n}{titulo}\PY{o}{=}\PY{l+s+s2}{\PYZdq{}}\PY{l+s+s2}{Atractor de Lorentz}\PY{l+s+s2}{\PYZdq{}}\PY{p}{)}
\end{Verbatim}
\end{tcolorbox}

    
    
    \hypertarget{probabiluxedstico}{%
\subsection{Probabilístico}\label{probabiluxedstico}}

    \begin{tcolorbox}[breakable, size=fbox, boxrule=1pt, pad at break*=1mm,colback=cellbackground, colframe=cellborder]
\prompt{In}{incolor}{ }{\boxspacing}
\begin{Verbatim}[commandchars=\\\{\}]
\PY{n}{lorentz} \PY{o}{=} \PY{n}{DistribucionProbabilidad}\PY{p}{(}\PY{n}{valores\PYZus{}x}\PY{p}{,} \PY{n}{valores\PYZus{}y}\PY{p}{,} \PY{n}{valores\PYZus{}z}\PY{p}{)}
\PY{n}{lorentz}\PY{o}{.}\PY{n}{mostrar\PYZus{}metricas}\PY{p}{(}\PY{p}{)}
\PY{n}{plotear\PYZus{}hist}\PY{p}{(}\PY{n}{valores\PYZus{}x}\PY{p}{,} \PY{l+s+s2}{\PYZdq{}}\PY{l+s+s2}{Histograma de Lorentz X}\PY{l+s+s2}{\PYZdq{}}\PY{p}{,} \PY{l+s+s2}{\PYZdq{}}\PY{l+s+s2}{Valor}\PY{l+s+s2}{\PYZdq{}}\PY{p}{,} \PY{l+s+s2}{\PYZdq{}}\PY{l+s+s2}{Frecuencia}\PY{l+s+s2}{\PYZdq{}}\PY{p}{,} \PY{l+s+s2}{\PYZdq{}}\PY{l+s+s2}{sturges}\PY{l+s+s2}{\PYZdq{}}\PY{p}{)}
\PY{n}{plotear\PYZus{}hist}\PY{p}{(}\PY{n}{valores\PYZus{}y}\PY{p}{,} \PY{l+s+s2}{\PYZdq{}}\PY{l+s+s2}{Histograma de Lorentz Y}\PY{l+s+s2}{\PYZdq{}}\PY{p}{,} \PY{l+s+s2}{\PYZdq{}}\PY{l+s+s2}{Valor}\PY{l+s+s2}{\PYZdq{}}\PY{p}{,} \PY{l+s+s2}{\PYZdq{}}\PY{l+s+s2}{Frecuencia}\PY{l+s+s2}{\PYZdq{}}\PY{p}{,} \PY{l+s+s2}{\PYZdq{}}\PY{l+s+s2}{sturges}\PY{l+s+s2}{\PYZdq{}}\PY{p}{)}
\PY{n}{plotear\PYZus{}hist}\PY{p}{(}\PY{n}{valores\PYZus{}z}\PY{p}{,} \PY{l+s+s2}{\PYZdq{}}\PY{l+s+s2}{Histograma de Lorentz Z}\PY{l+s+s2}{\PYZdq{}}\PY{p}{,} \PY{l+s+s2}{\PYZdq{}}\PY{l+s+s2}{Valor}\PY{l+s+s2}{\PYZdq{}}\PY{p}{,} \PY{l+s+s2}{\PYZdq{}}\PY{l+s+s2}{Frecuencia}\PY{l+s+s2}{\PYZdq{}}\PY{p}{,} \PY{l+s+s2}{\PYZdq{}}\PY{l+s+s2}{sturges}\PY{l+s+s2}{\PYZdq{}}\PY{p}{)}
\PY{n}{graficar}\PY{p}{(}\PY{n}{valores\PYZus{}x}\PY{p}{,} \PY{n}{valores\PYZus{}k}\PY{p}{,} \PY{n}{width}\PY{o}{=}\PY{l+m+mi}{10}\PY{p}{,} \PY{n}{height}\PY{o}{=}\PY{l+m+mi}{7} \PY{p}{,}\PY{n}{titulo}\PY{o}{=}\PY{l+s+s2}{\PYZdq{}}\PY{l+s+s2}{Variable X vs k}\PY{l+s+s2}{\PYZdq{}}\PY{p}{)}
\PY{n}{graficar}\PY{p}{(}\PY{n}{valores\PYZus{}y}\PY{p}{,} \PY{n}{valores\PYZus{}k}\PY{p}{,} \PY{n}{width}\PY{o}{=}\PY{l+m+mi}{10}\PY{p}{,} \PY{n}{height}\PY{o}{=}\PY{l+m+mi}{7} \PY{p}{,}\PY{n}{titulo}\PY{o}{=}\PY{l+s+s2}{\PYZdq{}}\PY{l+s+s2}{Variable Y vs k}\PY{l+s+s2}{\PYZdq{}}\PY{p}{)}
\PY{n}{graficar}\PY{p}{(}\PY{n}{valores\PYZus{}z}\PY{p}{,} \PY{n}{valores\PYZus{}k}\PY{p}{,} \PY{n}{width}\PY{o}{=}\PY{l+m+mi}{10}\PY{p}{,} \PY{n}{height}\PY{o}{=}\PY{l+m+mi}{7} \PY{p}{,}\PY{n}{titulo}\PY{o}{=}\PY{l+s+s2}{\PYZdq{}}\PY{l+s+s2}{Variable Z vs k}\PY{l+s+s2}{\PYZdq{}}\PY{p}{)}
\end{Verbatim}
\end{tcolorbox}

    \begin{Verbatim}[commandchars=\\\{\}]
--- Métricas para Variable 1 ---
Media: -0.8837743511144559
Mediana: -1.2706424157711937
Moda: -18.81364430507488
Media Geométrica: nan
Rango: 38.56871649435135
Desviación Estándar: 7.964705996176799
Varianza: 63.436541605534664
Asimetría: 0.2116758898459656
Curtosis: -0.7767563835465703
Entropia: 11.512925464970223
Coeficiente de Variación: -9.01214884334803

--- Métricas para Variable 2 ---
Media: -0.8926042312278303
Mediana: -1.1269238369230985
Moda: -25.60801161362229
Media Geométrica: nan
Rango: 53.02582318365201
Desviación Estándar: 8.981911993104438
Varianza: 80.67474305187335
Asimetría: 0.21541672236583292
Curtosis: -0.2165588403431289
Entropia: 11.512925464970223
Coeficiente de Variación: -10.06259177233485

--- Métricas para Variable 3 ---
Media: 23.943247724589817
Mediana: 23.718936504537208
Moda: 0.8609231341947348
Media Geométrica: 22.237786636601825
Rango: 47.54539290506558
Desviación Estándar: 8.350685729826283
Varianza: 69.73395215832431
Asimetría: 0.07697925163258434
Curtosis: -0.6133079795441914
Entropia: 11.512925464970223
Coeficiente de Variación: 0.3487699674614356

Matriz de Correlación:
[[ 1.          0.88597276 -0.03967717]
 [ 0.88597276  1.         -0.03792969]
 [-0.03967717 -0.03792969  1.        ]]
    \end{Verbatim}

    \begin{Verbatim}[commandchars=\\\{\}]
/home/rodrigo/.local/lib/python3.10/site-packages/scipy/stats/\_stats\_py.py:197:
RuntimeWarning:

invalid value encountered in log

    \end{Verbatim}

    \begin{center}
    \adjustimage{max size={0.9\linewidth}{0.9\paperheight}}{analisisCaos2_files/analisisCaos2_115_2.png}
    \end{center}
    { \hspace*{\fill} \\}
    
    \begin{center}
    \adjustimage{max size={0.9\linewidth}{0.9\paperheight}}{analisisCaos2_files/analisisCaos2_115_3.png}
    \end{center}
    { \hspace*{\fill} \\}
    
    \begin{center}
    \adjustimage{max size={0.9\linewidth}{0.9\paperheight}}{analisisCaos2_files/analisisCaos2_115_4.png}
    \end{center}
    { \hspace*{\fill} \\}
    
    \begin{center}
    \adjustimage{max size={0.9\linewidth}{0.9\paperheight}}{analisisCaos2_files/analisisCaos2_115_5.png}
    \end{center}
    { \hspace*{\fill} \\}
    
    \begin{center}
    \adjustimage{max size={0.9\linewidth}{0.9\paperheight}}{analisisCaos2_files/analisisCaos2_115_6.png}
    \end{center}
    { \hspace*{\fill} \\}
    
    \begin{center}
    \adjustimage{max size={0.9\linewidth}{0.9\paperheight}}{analisisCaos2_files/analisisCaos2_115_7.png}
    \end{center}
    { \hspace*{\fill} \\}
    
    \begin{center}
    \adjustimage{max size={0.9\linewidth}{0.9\paperheight}}{analisisCaos2_files/analisisCaos2_115_8.png}
    \end{center}
    { \hspace*{\fill} \\}
    
    \hypertarget{anuxe1lisis-de-muxe9tricas-estaduxedsticas-para-el-modelo-del-atractor-de-lorenz}{%
\subsection{Análisis de Métricas Estadísticas para el Modelo del
Atractor de
Lorenz}\label{anuxe1lisis-de-muxe9tricas-estaduxedsticas-para-el-modelo-del-atractor-de-lorenz}}

\hypertarget{anuxe1lisis-de-muxe9tricas-estaduxedsticas}{%
\subsubsection{Análisis de Métricas
Estadísticas}\label{anuxe1lisis-de-muxe9tricas-estaduxedsticas}}

\begin{itemize}
\item
  \textbf{Rango}: Los rangos muy amplios para todas las variables
  (Variable X: 38.57, Variable Y: 53.03, Variable Z: 47.55) indican que
  cada variable explora extensivamente su espacio de estado. Esto
  subraya la gran variabilidad y la capacidad del sistema para transitar
  por estados muy distintos, lo cual es característico de la dinámica
  caótica.
\item
  \textbf{Desviación Estándar y Varianza}: Altas en todas las variables
  (Variable X: Desviación Estándar de 7.98, Varianza de 63.72; Variable
  Y: Desviación Estándar de 9.03, Varianza de 81.57), reflejan una
  considerable dispersión de los datos y subrayan la alta variabilidad y
  la naturaleza impredecible del sistema.
\item
  \textbf{Curtosis y Asimetría}: La curtosis ligeramente negativa en
  todas las variables indica una distribución más plana que la normal,
  lo que sugiere un perfil de distribución con colas pesadas. La
  asimetría cercana a cero sugiere una simetría relativa en la
  distribución de los valores alrededor de la media.
\end{itemize}

\hypertarget{anuxe1lisis-de-correlaciuxf3n-y-gruxe1fica-de-dispersiuxf3n}{%
\subsubsection{Análisis de Correlación y Gráfica de
Dispersión}\label{anuxe1lisis-de-correlaciuxf3n-y-gruxe1fica-de-dispersiuxf3n}}

\begin{itemize}
\item
  \textbf{Matriz de Correlación}: La correlación notablemente alta entre
  las variables X y Y (0.883) sugiere una fuerte relación lineal entre
  estas dimensiones del sistema, lo cual es interesante dado que el
  atractor de Lorenz tiende a mostrar una estructura de alas de mariposa
  simétrica que podría explicar esta correlación. Las correlaciones
  cercanas a cero con la Variable Z indican que esta dimensión opera más
  independientemente, lo cual es característico de sistemas con
  dinámicas multidimensionales complejas.
\item
  \textbf{Gráfica de Dispersión}: Las visualizaciones muestran claros
  patrones de atractor extraño con estructuras en forma de mariposa y
  anillos concéntricos que son típicos del atractor de Lorenz. Estos
  patrones son indicativos de la dinámica no lineal y recurrente del
  sistema y proporcionan una confirmación visual de la naturaleza
  caótica y compleja del atractor.
\end{itemize}

\hypertarget{conclusiuxf3n}{%
\subsubsection{Conclusión}\label{conclusiuxf3n}}

Las métricas y las visualizaciones para el atractor de Lorenz ilustran
un sistema con una variabilidad extremadamente alta y relaciones
complejas entre sus variables. Los amplios rangos y las altas
desviaciones estándar, junto con las visualizaciones que muestran
patrones distintivos de atractores extraños, confirman la naturaleza
dinámica y caótica del sistema. Este análisis destaca cómo el atractor
de Lorenz continúa siendo un ejemplo fascinante de caos determinista en
sistemas dinámicos.

    \hypertarget{cauxf3tico}{%
\subsection{Caótico}\label{cauxf3tico}}

    \hypertarget{exponentes-de-laypounov}{%
\subsubsection{Exponentes de Laypounov}\label{exponentes-de-laypounov}}

    \begin{tcolorbox}[breakable, size=fbox, boxrule=1pt, pad at break*=1mm,colback=cellbackground, colframe=cellborder]
\prompt{In}{incolor}{ }{\boxspacing}
\begin{Verbatim}[commandchars=\\\{\}]
\PY{k}{def} \PY{n+nf}{leer\PYZus{}datos\PYZus{}lorenz}\PY{p}{(}\PY{n}{csv\PYZus{}path}\PY{p}{)}\PY{p}{:}
    \PY{n}{df} \PY{o}{=} \PY{n}{pd}\PY{o}{.}\PY{n}{read\PYZus{}csv}\PY{p}{(}\PY{n}{csv\PYZus{}path}\PY{p}{)}
    \PY{k}{return} \PY{n}{df}\PY{p}{[}\PY{l+s+s1}{\PYZsq{}}\PY{l+s+s1}{Valores x}\PY{l+s+s1}{\PYZsq{}}\PY{p}{]}\PY{o}{.}\PY{n}{values}\PY{p}{,} \PY{n}{df}\PY{p}{[}\PY{l+s+s1}{\PYZsq{}}\PY{l+s+s1}{Valores y}\PY{l+s+s1}{\PYZsq{}}\PY{p}{]}\PY{o}{.}\PY{n}{values}\PY{p}{,} \PY{n}{df}\PY{p}{[}\PY{l+s+s1}{\PYZsq{}}\PY{l+s+s1}{Valores z}\PY{l+s+s1}{\PYZsq{}}\PY{p}{]}\PY{o}{.}\PY{n}{values}

\PY{k}{def} \PY{n+nf}{jacobian\PYZus{}lorenz}\PY{p}{(}\PY{n}{x}\PY{p}{,} \PY{n}{y}\PY{p}{,} \PY{n}{z}\PY{p}{,} \PY{n}{sigma}\PY{o}{=}\PY{l+m+mi}{10}\PY{p}{,} \PY{n}{rho}\PY{o}{=}\PY{l+m+mi}{28}\PY{p}{,} \PY{n}{beta}\PY{o}{=}\PY{l+m+mi}{8}\PY{o}{/}\PY{l+m+mi}{3}\PY{p}{)}\PY{p}{:}
    \PY{k}{return} \PY{n}{np}\PY{o}{.}\PY{n}{array}\PY{p}{(}\PY{p}{[}
        \PY{p}{[}\PY{o}{\PYZhy{}}\PY{n}{sigma}\PY{p}{,} \PY{n}{sigma}\PY{p}{,} \PY{l+m+mi}{0}\PY{p}{]}\PY{p}{,}
        \PY{p}{[}\PY{n}{rho} \PY{o}{\PYZhy{}} \PY{n}{z}\PY{p}{,} \PY{o}{\PYZhy{}}\PY{l+m+mi}{1}\PY{p}{,} \PY{o}{\PYZhy{}}\PY{n}{x}\PY{p}{]}\PY{p}{,}
        \PY{p}{[}\PY{n}{y}\PY{p}{,} \PY{n}{x}\PY{p}{,} \PY{o}{\PYZhy{}}\PY{n}{beta}\PY{p}{]}
    \PY{p}{]}\PY{p}{)}

\PY{k}{def} \PY{n+nf}{calculate\PYZus{}lyapunov\PYZus{}exponents\PYZus{}lorenz}\PY{p}{(}\PY{n}{x}\PY{p}{,} \PY{n}{y}\PY{p}{,} \PY{n}{z}\PY{p}{,} \PY{n}{sigma}\PY{o}{=}\PY{l+m+mi}{10}\PY{p}{,} \PY{n}{rho}\PY{o}{=}\PY{l+m+mi}{28}\PY{p}{,} \PY{n}{beta}\PY{o}{=}\PY{l+m+mi}{8}\PY{o}{/}\PY{l+m+mi}{3}\PY{p}{,} \PY{n}{window\PYZus{}size}\PY{o}{=}\PY{l+m+mi}{100}\PY{p}{)}\PY{p}{:}
    \PY{n}{n} \PY{o}{=} \PY{n+nb}{len}\PY{p}{(}\PY{n}{x}\PY{p}{)}
    \PY{n}{perturbations} \PY{o}{=} \PY{n}{np}\PY{o}{.}\PY{n}{eye}\PY{p}{(}\PY{l+m+mi}{3}\PY{p}{)}
    \PY{n}{lyapunov\PYZus{}exponents} \PY{o}{=} \PY{n}{np}\PY{o}{.}\PY{n}{zeros}\PY{p}{(}\PY{p}{(}\PY{n}{n} \PY{o}{\PYZhy{}} \PY{n}{window\PYZus{}size}\PY{p}{,} \PY{l+m+mi}{3}\PY{p}{)}\PY{p}{)}
    
    \PY{k}{for} \PY{n}{i} \PY{o+ow}{in} \PY{n+nb}{range}\PY{p}{(}\PY{n}{window\PYZus{}size}\PY{p}{,} \PY{n}{n}\PY{p}{)}\PY{p}{:}
        \PY{n}{J} \PY{o}{=} \PY{n}{jacobian\PYZus{}lorenz}\PY{p}{(}\PY{n}{x}\PY{p}{[}\PY{n}{i}\PY{o}{\PYZhy{}}\PY{l+m+mi}{1}\PY{p}{]}\PY{p}{,} \PY{n}{y}\PY{p}{[}\PY{n}{i}\PY{o}{\PYZhy{}}\PY{l+m+mi}{1}\PY{p}{]}\PY{p}{,} \PY{n}{z}\PY{p}{[}\PY{n}{i}\PY{o}{\PYZhy{}}\PY{l+m+mi}{1}\PY{p}{]}\PY{p}{,} \PY{n}{sigma}\PY{p}{,} \PY{n}{rho}\PY{p}{,} \PY{n}{beta}\PY{p}{)}
        
        \PY{n}{perturbations} \PY{o}{=} \PY{n}{J} \PY{o}{@} \PY{n}{perturbations}
        
        \PY{k}{if} \PY{n}{i} \PY{o}{\PYZpc{}} \PY{n}{window\PYZus{}size} \PY{o}{==} \PY{l+m+mi}{0}\PY{p}{:}
            \PY{n}{Q}\PY{p}{,} \PY{n}{R} \PY{o}{=} \PY{n}{np}\PY{o}{.}\PY{n}{linalg}\PY{o}{.}\PY{n}{qr}\PY{p}{(}\PY{n}{perturbations}\PY{p}{)}
            \PY{n}{perturbations} \PY{o}{=} \PY{n}{Q}
            \PY{n}{lyapunov\PYZus{}exponents}\PY{p}{[}\PY{n}{i} \PY{o}{\PYZhy{}} \PY{n}{window\PYZus{}size}\PY{p}{]} \PY{o}{=} \PY{n}{np}\PY{o}{.}\PY{n}{log}\PY{p}{(}\PY{n}{np}\PY{o}{.}\PY{n}{abs}\PY{p}{(}\PY{n}{np}\PY{o}{.}\PY{n}{diag}\PY{p}{(}\PY{n}{R}\PY{p}{)}\PY{p}{)}\PY{p}{)} \PY{o}{/} \PY{n}{window\PYZus{}size}
    
    \PY{k}{return} \PY{n}{lyapunov\PYZus{}exponents}
\end{Verbatim}
\end{tcolorbox}

    \begin{tcolorbox}[breakable, size=fbox, boxrule=1pt, pad at break*=1mm,colback=cellbackground, colframe=cellborder]
\prompt{In}{incolor}{ }{\boxspacing}
\begin{Verbatim}[commandchars=\\\{\}]
\PY{n}{csv\PYZus{}path} \PY{o}{=} \PY{l+s+s2}{\PYZdq{}}\PY{l+s+s2}{datosLorentz.csv}\PY{l+s+s2}{\PYZdq{}}
\PY{n}{x}\PY{p}{,} \PY{n}{y}\PY{p}{,} \PY{n}{z} \PY{o}{=} \PY{n}{leer\PYZus{}datos\PYZus{}lorenz}\PY{p}{(}\PY{n}{csv\PYZus{}path}\PY{p}{)}
\PY{n}{exponents} \PY{o}{=} \PY{n}{calculate\PYZus{}lyapunov\PYZus{}exponents\PYZus{}lorenz}\PY{p}{(}\PY{n}{x}\PY{p}{,} \PY{n}{y}\PY{p}{,} \PY{n}{z}\PY{p}{)}

\PY{n}{plt}\PY{o}{.}\PY{n}{figure}\PY{p}{(}\PY{n}{figsize}\PY{o}{=}\PY{p}{(}\PY{l+m+mi}{12}\PY{p}{,} \PY{l+m+mi}{6}\PY{p}{)}\PY{p}{)}
\PY{n}{plt}\PY{o}{.}\PY{n}{plot}\PY{p}{(}\PY{n}{exponents}\PY{p}{[}\PY{p}{:}\PY{p}{,} \PY{l+m+mi}{0}\PY{p}{]}\PY{p}{,} \PY{n}{label}\PY{o}{=}\PY{l+s+s1}{\PYZsq{}}\PY{l+s+s1}{Exponente de Lyapunov \PYZdl{}}\PY{l+s+s1}{\PYZbs{}}\PY{l+s+s1}{lambda\PYZus{}1\PYZdl{}}\PY{l+s+s1}{\PYZsq{}}\PY{p}{)}
\PY{n}{plt}\PY{o}{.}\PY{n}{plot}\PY{p}{(}\PY{n}{exponents}\PY{p}{[}\PY{p}{:}\PY{p}{,} \PY{l+m+mi}{1}\PY{p}{]}\PY{p}{,} \PY{n}{label}\PY{o}{=}\PY{l+s+s1}{\PYZsq{}}\PY{l+s+s1}{Exponente de Lyapunov \PYZdl{}}\PY{l+s+s1}{\PYZbs{}}\PY{l+s+s1}{lambda\PYZus{}2\PYZdl{}}\PY{l+s+s1}{\PYZsq{}}\PY{p}{)}
\PY{n}{plt}\PY{o}{.}\PY{n}{plot}\PY{p}{(}\PY{n}{exponents}\PY{p}{[}\PY{p}{:}\PY{p}{,} \PY{l+m+mi}{2}\PY{p}{]}\PY{p}{,} \PY{n}{label}\PY{o}{=}\PY{l+s+s1}{\PYZsq{}}\PY{l+s+s1}{Exponente de Lyapunov \PYZdl{}}\PY{l+s+s1}{\PYZbs{}}\PY{l+s+s1}{lambda\PYZus{}3\PYZdl{}}\PY{l+s+s1}{\PYZsq{}}\PY{p}{)}
\PY{n}{plt}\PY{o}{.}\PY{n}{xlabel}\PY{p}{(}\PY{l+s+s1}{\PYZsq{}}\PY{l+s+s1}{Tiempo (en ventanas de }\PY{l+s+s1}{\PYZsq{}} \PY{o}{+} \PY{n+nb}{str}\PY{p}{(}\PY{n}{window\PYZus{}size}\PY{p}{)} \PY{o}{+} \PY{l+s+s1}{\PYZsq{}}\PY{l+s+s1}{)}\PY{l+s+s1}{\PYZsq{}}\PY{p}{)}
\PY{n}{plt}\PY{o}{.}\PY{n}{ylabel}\PY{p}{(}\PY{l+s+s1}{\PYZsq{}}\PY{l+s+s1}{Valor del Exponente de Lyapunov}\PY{l+s+s1}{\PYZsq{}}\PY{p}{)}
\PY{n}{plt}\PY{o}{.}\PY{n}{title}\PY{p}{(}\PY{l+s+s1}{\PYZsq{}}\PY{l+s+s1}{Espectro de Exponentes de Lyapunov del Atractor de Lorenz}\PY{l+s+s1}{\PYZsq{}}\PY{p}{)}
\PY{n}{plt}\PY{o}{.}\PY{n}{legend}\PY{p}{(}\PY{p}{)}
\PY{n}{plt}\PY{o}{.}\PY{n}{grid}\PY{p}{(}\PY{k+kc}{True}\PY{p}{)}
\PY{n}{plt}\PY{o}{.}\PY{n}{show}\PY{p}{(}\PY{p}{)}
\end{Verbatim}
\end{tcolorbox}

    \begin{Verbatim}[commandchars=\\\{\}]
/tmp/ipykernel\_17541/2811961212.py:25: RuntimeWarning:

divide by zero encountered in log

    \end{Verbatim}

    \begin{center}
    \adjustimage{max size={0.9\linewidth}{0.9\paperheight}}{analisisCaos2_files/analisisCaos2_120_1.png}
    \end{center}
    { \hspace*{\fill} \\}
    
    \hypertarget{interpretaciuxf3n}{%
\paragraph{Interpretación:}\label{interpretaciuxf3n}}

\begin{itemize}
\tightlist
\item
  **Exponente de Lyapunov \$ \lambda\_1 \$ (Azul):** Este exponente
  muestra una serie de valores positivos alrededor de 2.5 a 3.0, lo que
  indica que el sistema es altamente caótico.
\item
  **Exponente de Lyapunov \$ \lambda\_2 \$ (Naranja):** Este exponente
  se mantiene alrededor de 1.5, lo que sugiere una dinámica caótica
  adicional en otra dirección.
\item
  **Exponente de Lyapunov \$ \lambda\_3 \$ (Verde):** Este exponente se
  mantiene en torno a 1.0, lo cual es consistente con la presencia de
  caos en múltiples dimensiones en el sistema.
\end{itemize}

    \hypertarget{dimensiuxf3n-de-kaplan-yorke}{%
\subsubsection{Dimensión de
Kaplan-Yorke}\label{dimensiuxf3n-de-kaplan-yorke}}

    \begin{tcolorbox}[breakable, size=fbox, boxrule=1pt, pad at break*=1mm,colback=cellbackground, colframe=cellborder]
\prompt{In}{incolor}{ }{\boxspacing}
\begin{Verbatim}[commandchars=\\\{\}]
\PY{n}{averages} \PY{o}{=} \PY{n}{np}\PY{o}{.}\PY{n}{mean}\PY{p}{(}\PY{n}{exponents}\PY{p}{,} \PY{n}{axis}\PY{o}{=}\PY{l+m+mi}{1}\PY{p}{)}
\PY{n}{averages} \PY{o}{=} \PY{n}{averages}\PY{o}{.}\PY{n}{reshape}\PY{p}{(}\PY{o}{\PYZhy{}}\PY{l+m+mi}{1}\PY{p}{,} \PY{l+m+mi}{1}\PY{p}{)}
\PY{n}{kaplan\PYZus{}yorke\PYZus{}dimension} \PY{o}{=} \PY{n}{calcular\PYZus{}dimension\PYZus{}kaplan\PYZus{}yorke}\PY{p}{(}\PY{n}{averages}\PY{p}{)}
\PY{n+nb}{print}\PY{p}{(}\PY{l+s+s2}{\PYZdq{}}\PY{l+s+s2}{Dimensión de Kaplan\PYZhy{}Yorke:}\PY{l+s+s2}{\PYZdq{}}\PY{p}{,} \PY{n}{kaplan\PYZus{}yorke\PYZus{}dimension}\PY{p}{)}
\end{Verbatim}
\end{tcolorbox}

    \begin{Verbatim}[commandchars=\\\{\}]
Dimensión de Kaplan-Yorke: [1098.]
    \end{Verbatim}

    \hypertarget{interpretaciuxf3n}{%
\subsubsection{Interpretación:}\label{interpretaciuxf3n}}

\begin{itemize}
\tightlist
\item
  La dimensión de Kaplan-Yorke calculada es \textbf{1098.0}. Este valor
  es bastante alto y sugiere que el sistema tiene un comportamiento
  complejo y un atractor de alta dimensión.
\end{itemize}

    \hypertarget{dimensiuxf3n-grassberger-procaccia}{%
\subsubsection{Dimensión
Grassberger-Procaccia}\label{dimensiuxf3n-grassberger-procaccia}}

    \begin{tcolorbox}[breakable, size=fbox, boxrule=1pt, pad at break*=1mm,colback=cellbackground, colframe=cellborder]
\prompt{In}{incolor}{15}{\boxspacing}
\begin{Verbatim}[commandchars=\\\{\}]
\PY{n}{m} \PY{o}{=} \PY{l+m+mi}{10}
\PY{n}{tau} \PY{o}{=} \PY{l+m+mi}{1}
\PY{n}{r\PYZus{}vals} \PY{o}{=} \PY{n}{np}\PY{o}{.}\PY{n}{logspace}\PY{p}{(}\PY{o}{\PYZhy{}}\PY{l+m+mi}{3}\PY{p}{,} \PY{l+m+mi}{0}\PY{p}{,} \PY{l+m+mi}{50}\PY{p}{)} 
\PY{n}{dimension\PYZus{}gp} \PY{o}{=} \PY{n}{calcular\PYZus{}dimension\PYZus{}gp\PYZus{}multivariado}\PY{p}{(}\PY{p}{[}\PY{n}{x}\PY{p}{,} \PY{n}{y}\PY{p}{,} \PY{n}{z}\PY{p}{]}\PY{p}{,} \PY{n}{m}\PY{p}{,} \PY{n}{tau}\PY{p}{,} \PY{n}{r\PYZus{}vals}\PY{p}{,} \PY{n}{n\PYZus{}jobs}\PY{o}{=}\PY{o}{\PYZhy{}}\PY{l+m+mi}{1}\PY{p}{)}
\PY{n+nb}{print}\PY{p}{(}\PY{l+s+s2}{\PYZdq{}}\PY{l+s+s2}{La dimensión de Grassberger\PYZhy{}Procaccia es:}\PY{l+s+s2}{\PYZdq{}}\PY{p}{,} \PY{n}{dimension\PYZus{}gp}\PY{p}{)}
\end{Verbatim}
\end{tcolorbox}

    \begin{Verbatim}[commandchars=\\\{\}]
La dimensión de Grassberger-Procaccia es: 2.648790926345455
    \end{Verbatim}

    \hypertarget{interpretaciuxf3n}{%
\paragraph{Interpretación:}\label{interpretaciuxf3n}}

\begin{itemize}
\tightlist
\item
  La dimensión de Grassberger-Procaccia calculada es
  \textbf{2.648790926345455}. Este valor sugiere que el atractor del
  sistema tiene una dimensión fractal intermedia, lo cual puede indicar
  un comportamiento caótico complejo pero en una dimensión razonable
  para sistemas caóticos típicos.
\end{itemize}


    % Add a bibliography block to the postdoc
    
    
    
\end{document}
