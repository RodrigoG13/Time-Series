\documentclass[11pt]{article}

\usepackage[utf8]{inputenc}

% This is the preamble, load any packages you're going to use here
\usepackage{physics} % provides lots of nice features and commands often used in physics, it also loads some other packages (like AMSmath)
\usepackage{siunitx} % typesets numbers with units very nicely
\usepackage[spanish]{babel}

\usepackage[margin=1.3in,letterpaper]{geometry}


% Establecer Latin Modern Sans Serif como la fuente predeterminada
\renewcommand*\familydefault{\sfdefault} % Cambia la fuente predeterminada a sans-serif
\usepackage[T1]{fontenc} % Codificación de la fuente
\usepackage{lmodern} % Carga la fuente Latin Modern
\renewcommand{\abstractname}{Objetivo}

\usepackage{enumitem}

\usepackage{changepage}

\usepackage{graphicx}

\usepackage{tabularx}

\setlength{\parskip}{8pt}

\usepackage{fancyhdr} % Paquete para personalizar encabezado y pie de página
\pagestyle{fancy} % Establece que personalizaremos el pie de pagina y el encabezado
\setlength{\headheight}{13.59999pt} % Establece la altura del encabezado
\fancyhead[R]{\textcolor{darkBlue}{}} % Encabezado derecho
\fancyhead[L]{\textit{\textcolor{darkBlue}{Series de Tiempo}}} % Encabezado izquierdo
\fancyfoot[L]{\textit{\textcolor{darkBlue}{Centro de Investigación en Computación}}} % Pie de página izquierdo 
\fancyfoot[R]{\textcolor{darkBlue}{\thepage}} % Pie de página  derecho
\fancyfoot[C]{} % Elimina la nueración central de páginas en el pie de página
\renewcommand{\headrulewidth}{0.5pt} % Grosor de la linea de encabezado
\renewcommand{\footrulewidth}{0.5pt} % Grosor de la linea de pie de página

\usepackage{xcolor}
\definecolor{darkBlue}{rgb}{0,0,0.31}
%\definecolor{darkBlue}{rgb}{0,0,0.5}
\definecolor{munsell}{rgb}{0.0, 0.5, 0.69}
\definecolor{indigo}{rgb}{0.0, 0.25, 0.42}
\renewcommand{\footrulewidth}{2pt}
\renewcommand{\footrule}{\hbox to\headwidth{\color{darkBlue}\leaders\hrule height \footrulewidth\hfill}}

\usepackage{colortbl}

\usepackage{titlesec}
\titleformat{\section}
{\normalfont\Large\bfseries\color{darkBlue}}{\thesection.}{1em}{}

\usepackage{tabularx}

\usepackage{textcomp}

\usepackage{titling}

\usepackage{listings}

\definecolor{codegreen}{rgb}{0,0.6,0}
\definecolor{codegray}{rgb}{0.5,0.5,0.5}
\definecolor{codepurple}{rgb}{0.58,0,0.82}
\definecolor{backcolour}{rgb}{0.95,0.95,0.92}

\usepackage{amsmath, amssymb}


\definecolor{codeblue}{rgb}{0.25,0.5,0.5}
\definecolor{backcolour}{rgb}{0.95,0.95,0.92}
\definecolor{commentblue}{rgb}{0.3,0.3,0.6}
\definecolor{keywordblue}{rgb}{0.2,0.2,0.7}
\definecolor{stringblue}{rgb}{0.15,0.2,0.9}

\lstdefinestyle{bluepythonstyle}{
	language=Python,
	basicstyle=\ttfamily\small,
	commentstyle=\color{commentblue},
	keywordstyle=\color{keywordblue},
	numberstyle=\tiny\color{codeblue},
	stringstyle=\color{stringblue},
	backgroundcolor=\color{backcolour},
	breaklines=true,
	captionpos=b,
	abovecaptionskip=1\baselineskip,
	showstringspaces=false,
	frame=lines,
	numbers=left,
	xleftmargin=\parindent,
	tabsize=4
}
\lstset{style=bluepythonstyle}

\definecolor{codegreen}{rgb}{0,0.6,0}
\definecolor{codegray}{rgb}{0.5,0.5,0.5}
\definecolor{codepurple}{rgb}{0.58,0,0.82}



\begin{document}
	
	

\title{\LARGE \textcolor{darkBlue}{Métricas para distribuciones de Probabilidad}}
\author{Rodrigo Gerardo Trejo Arriaga}
\date{\today}

\maketitle

\thispagestyle{fancy} 

\section*{Distribución Normal}
La distribución normal se caracteriza por dos parámetros: la media \(\mu\) y la desviación estándar \(\sigma\). Estos se calculan como sigue:
\begin{itemize}
	\item \textbf{Media} (\(\mu\)): La media de una población se calcula como:
	\[
	\mu = \frac{1}{N} \sum_{i=1}^{N} x_i
	\]
	donde \(N\) es el número total de valores en el conjunto de datos y \(x_i\) es cada valor individual.
	
	\item \textbf{Desviación estándar} (\(\sigma\)): La desviación estándar se calcula usando la fórmula:
	\[
	\sigma = \sqrt{\frac{1}{N} \sum_{i=1}^{N} (x_i - \mu)^2}
	\]
	donde \(\mu\) es la media del conjunto de datos, \(N\) es el número total de valores, y \(x_i\) es cada valor individual.
\end{itemize}

\subsection*{Tendencia Central}
Para una distribución normal con media \(\mu\) y desviación estándar \(\sigma\), las métricas son:
\begin{itemize}
	\item \textbf{Media}: \(\mu\)
	\item \textbf{Mediana}: \(\mu\)
	\item \textbf{Moda}: \(\mu\)
\end{itemize}

\subsection*{Dispersión}
Para una distribución normal con media \(\mu\) y desviación estándar \(\sigma\):
\begin{itemize}
	\item \textbf{Rango}: No definido específicamente ya que teóricamente es \((-\infty, \infty)\). En el caso discreto, diferencia entre el valor máximo y mínimo observado.
	\item \textbf{Varianza} (\(\sigma^2\)): \(\sigma^2\)
	\item \textbf{Desviación estándar} (\(\sigma\)): \(\sigma\)
	\item \textbf{Coeficiente de variación} (\(CV\)): \(\frac{\sigma}{\mu}\) (para \(\mu \neq 0\))
\end{itemize}

\subsection*{Posición Relativa}
La posición relativa dentro de una distribución normal puede determinarse mediante la función de distribución acumulativa (CDF). Los percentiles y cuartiles se calculan como sigue:
\begin{itemize}
	\item \textbf{Percentil} (\(P\)): Un valor \(x\) que cumple con la condición de que \(P\)% de los datos son menores o iguales a \(x\) se encuentra mediante la inversa de la CDF (función cuantil) de la distribución normal estándar, ajustada por \(\mu\) y \(\sigma\).
	\[
	x = \mu + Z(P) \cdot \sigma
	\]
	donde \(Z(P)\) es el valor \(Z\) para el percentil \(P\) de la distribución normal estándar.
	
	\item \textbf{Cuartil}: Los cuartiles son casos especiales de percentiles; por ejemplo, el primer cuartil (25\%) es \(Q1 = \mu + Z(0.25) \cdot \sigma\), el segundo cuartil (mediana, 50\%) es \(Q2 = \mu\), y el tercer cuartil (75\%) es \(Q3 = \mu + Z(0.75) \cdot \sigma\).
\end{itemize}

Para calcular cualquier percentil \(P\) en una distribución normal, se utiliza la función de distribución acumulativa (CDF) inversa ajustada por la media \(\mu\) y la desviación estándar \(\sigma\):
\begin{itemize}
	\item \textbf{Percentil} \(P\): 
	\[
	x_P = \mu + Z(P) \cdot \sigma
	\]
	Donde \(x_P\) es el valor correspondiente al percentil \(P\) y \(Z(P)\) es el valor de la puntuación Z para el percentil \(P\) en la distribución normal estándar.
\end{itemize}
Los cuartiles son casos específicos de percentiles (25\%, 50\%, 75\%, 100\%), y se calculan con la misma fórmula ajustando \(P\) para 0.25, 0.5, 0.75 y 1.

\subsection*{Forma}
La distribución normal es simétrica, por lo que sus métricas de forma son:
\begin{itemize}
	\item \textbf{Asimetría (Skewness)}: 0, indicando una distribución perfectamente simétrica.
\end{itemize}

\textit{Ver la sección de notas.}

\subsection*{Curtosis y Entropía}
Para una distribución normal con media \(\mu\) y desviación estándar \(\sigma\):
\begin{itemize}
	\item \textbf{Curtosis}: 3 (La curtosis se mide a menudo relativa a una distribución normal, que tiene una curtosis de 3. Por lo tanto, el exceso de curtosis sería 0).
	\item \textbf{Entropía}: \(\frac{1}{2} \ln(2 \pi e \sigma^2)\), donde \(e\) es la base del logaritmo natural.
\end{itemize}



\section*{Distribución Gamma}
La distribución gamma se caracteriza por los parámetros de forma \(k\) y escala \(\theta\). La estimación de estos parámetros no se realiza mediante una fórmula directa, sino a través de métodos estadísticos avanzados como la estimación de máxima verosimilitud o el ajuste de momentos.
\begin{itemize}
	\item \textbf{Nota}: Los parámetros \(k\) y \(\theta\) suelen estimarse a partir de los datos utilizando técnicas estadísticas complejas que van más allá de cálculos directos.
\end{itemize}

\subsection*{Tendencia Central}
Para una distribución gamma con parámetros de forma \(k\) y escala \(\theta\):
\begin{itemize}
	\item \textbf{Media}: \(k\theta\)
	\item \textbf{Mediana}: La mediana para la distribución gamma no tiene una expresión simple cerrada.
	\item \textbf{Moda}: Para \(k > 1\), \(\text{Moda} = (k - 1)\theta\). Para \(k \leq 1\), la moda no está bien definida.
\end{itemize}

\subsection*{Dispersión}
Para una distribución gamma con parámetros de forma \(k\) y escala \(\theta\):
\begin{itemize}
	\item \textbf{Rango}: \((0, \infty)\) para \(k, \theta > 0\). En el caso discreto, diferencia entre el valor máximo y mínimo observado.
	\item \textbf{Varianza}: \(k\theta^2\)
	\item \textbf{Desviación estándar}: \(\sqrt{k}\theta\)
	\item \textbf{Coeficiente de variación}: \(\frac{\sqrt{k}\theta}{k\theta} = \frac{1}{\sqrt{k}}\)
\end{itemize}

\subsection*{Posición Relativa}
Los percentiles y cuartiles de la distribución gamma también se basan en su CDF, pero no hay fórmulas simples para su cálculo directo y generalmente se requieren métodos numéricos:
\begin{itemize}
	\item \textbf{Percentil y Cuartil}: Se determinan a través de la inversa de la CDF de la distribución gamma, que generalmente se calcula mediante software estadístico debido a la complejidad matemática de la función.
\end{itemize}

\begin{lstlisting}
	import scipy.stats as stats
	
	# Parametros de la distribucion Gamma: forma (k) y escala (theta)
	k = 2.0  # Parametro de forma
	theta = 3.0  # Parametro de escala
	
	# Calcular cuartiles
	cuartil_1 = stats.gamma.ppf(0.25, k, scale=theta)
	cuartil_2 = stats.gamma.ppf(0.50, k, scale=theta)  # Mediana
	cuartil_3 = stats.gamma.ppf(0.75, k, scale=theta)
	cuartil_4 = stats.gamma.ppf(1.00, k, scale=theta)  # Maximo
	
	print("Cuartil 1 (Q1):", cuartil_1)
	print("Cuartil 2 (Mediana, Q2):", cuartil_2)
	print("Cuartil 3 (Q3):", cuartil_3)
	print("Cuartil 4 (Maximo):", cuartil_4)
	
	# Calcular un percentil especifico, por ejemplo, el percentil 90
	percentil_90 = stats.gamma.ppf(0.90, k, scale=theta)
	print("Percentil 90:", percentil_90)
	
	# Para calcular cualquier otro percentil, cambia el 0.90 del ejemplo anterior
	# por el valor correspondiente.
\end{lstlisting}

Este código utiliza la función \texttt{ppf} (Percent Point Function, la inversa de la CDF) de la distribución gamma en SciPy para calcular los cuartiles y percentiles.


\subsection*{Forma}
La asimetría de la distribución gamma depende de su parámetro de forma \(k\):
\begin{itemize}
	\item \textbf{Asimetría (Skewness)}: \(\frac{2}{\sqrt{k}}\), indicando que la distribución puede ser asimétrica dependiendo del valor de \(k\). Para valores grandes de \(k\), la distribución se aproxima a una forma simétrica.
\end{itemize}

\subsection*{Curtosis y Entropía}
Para una distribución gamma con parámetros de forma \(k\) y escala \(\theta\):
\begin{itemize}
	\item \textbf{Curtosis}: \(\frac{6}{k}\)
	\item \textbf{Entropía}: \(k + \ln(\theta) + \ln(\Gamma(k)) + (1-k) \psi(k)\), donde \(\Gamma(k)\) es la función gamma y \(\psi(k)\) es la función digamma.
\end{itemize}



\section*{Distribución Uniforme}
La distribución uniforme está definida por dos parámetros: \(a\) y \(b\), que son los límites inferior y superior de la distribución, respectivamente.
\begin{itemize}
	\item \textbf{Límite inferior} (\(a\)): El valor mínimo en el rango de valores posibles.
	\item \textbf{Límite superior} (\(b\)): El valor máximo en el rango de valores posibles.
\end{itemize}

\subsection*{Tendencia Central}
Para una distribución uniforme definida entre \(a\) y \(b\):
\begin{itemize}
	\item \textbf{Media}: \(\frac{a + b}{2}\)
	\item \textbf{Mediana}: \(\frac{a + b}{2}\)
	\item \textbf{Moda}: No está bien definida ya que todos los valores dentro del intervalo \([a, b]\) son igualmente probables.
\end{itemize}

\subsection*{Dispersión}

Para una distribución uniforme definida entre \(a\) y \(b\):
\begin{itemize}
	\item \textbf{Rango}: \(b - a\)
	\item \textbf{Varianza}: \(\frac{(b - a)^2}{12}\)
	\item \textbf{Desviación estándar}: \(\sqrt{\frac{(b - a)^2}{12}}\)
	\item \textbf{Coeficiente de variación}: \(\frac{\sqrt{\frac{(b - a)^2}{12}}}{\frac{a + b}{2}}\) (para \(a, b > 0\))
\end{itemize}

\subsection*{Posición Relativa}
Para la distribución uniforme, los percentiles y cuartiles se pueden calcular de forma más directa debido a la naturaleza uniforme de la distribución:
\begin{itemize}
	\item \textbf{Percentil} (\(P\)): Dado un percentil \(P\) (como fracción, por ejemplo, 0.25 para el 25\%), el valor correspondiente \(x\) es:
	\[
	x = a + P \cdot (b - a)
	\]
	
	\item \textbf{Cuartil}: De manera similar, cada cuartil se calcula ajustando \(P\) para 0.25, 0.5, y 0.75 respectivamente.
\end{itemize}

Los percentiles y cuartiles se calculan directamente de la forma:
\begin{itemize}
	\item \textbf{Percentil} \(P\): 
	\[
	x_P = a + P \cdot (b - a)
	\]
	Donde \(x_P\) es el valor correspondiente al percentil \(P\) (como fracción, por ejemplo, 0.25 para el 25\%).
\end{itemize}
Para los cuartiles, \(P\) se ajusta para 0.25, 0.5, 0.75 y 1, respectivamente.


\subsection*{Forma}
La distribución uniforme es simétrica entre \(a\) y \(b\), por lo que sus métricas de forma son:
\begin{itemize}
	\item \textbf{Asimetría (Skewness)}: 0, reflejando su simetría perfecta.
\end{itemize}

\textit{Ver la sección de notas.}


\subsection*{Curtosis y Entropía}
Para una distribución uniforme definida entre \(a\) y \(b\):
\begin{itemize}
	\item \textbf{Curtosis}: \(-\frac{6}{5}\) (o el exceso de curtosis es \(-\frac{6}{5}\), ya que la curtosis de una distribución uniforme perfecta es menor en comparación con una normal).
	\item \textbf{Entropía}: \(\ln(b - a)\)
\end{itemize}



\section*{Notas}

La asimetría (skewness) de un conjunto de datos se puede calcular utilizando la siguiente fórmula:

\[ \text{Skewness} = \frac{n}{(n-1)(n-2)} \sum \left(\frac{x_i - \bar{x}}{s}\right)^3 \]

donde \(n\) es el número de observaciones, \(x_i\) son los valores individuales, \(\bar{x}\) es la media del conjunto de datos, y \(s\) es la desviación estándar.

Este cálculo puede realizarse fácilmente en Python para conjuntos de datos tanto normalmente como uniformemente distribuidos, utilizando bibliotecas como SciPy o Pandas. A continuación se muestra un ejemplo de código en Python para calcular la asimetría de un conjunto de datos:

\begin{verbatim}
	import numpy as np
	from scipy.stats import skew
	
	# Ejemplo de conjunto de datos
	data = np.array([1, 2, 3, 4, 5, 6, 7, 8, 9, 10])
	
	# Cálculo de la asimetría
	data_skewness = skew(data)
	
	print("Asimetría del conjunto de datos:", data_skewness)
\end{verbatim}






\end{document}